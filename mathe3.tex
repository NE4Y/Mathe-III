\documentclass[a4paper, openany]{book}
\usepackage{titlesec}
\usepackage[utf8x]{inputenc}
\usepackage[document]{ragged2e}
\usepackage{enumitem}
\usepackage{polynom}
\usepackage{amsmath, amssymb, amstext, amsfonts, mathrsfs}
\usepackage{wasysym}
\usepackage{graphicx}
\renewcommand{\chaptername}{}
\renewcommand{\contentsname}{Inhaltsverzeichnis}
\titleformat{\chapter}
  {\Large\bfseries} % format
  {}                % label
  {0pt}             % sep
  {\huge}           % before-code

\author{Steffen Lindner}
\title{\vspace{-2cm}Mathe III Skript WS 15/16}

\begin{document}
\maketitle

\tableofcontents

\chapter{Algebraische Strukturen}
\section{Definition Verknüpfung}

Sei X $\neq \emptyset$ Menge. Eine \underline{Verknüpfung} auf X ist Abb. 
$\begin{cases}
    X \times X \rightarrow x \\
    (a,b) \mapsto a \ast b
\end{cases}$

$\ast$ ist Platzhalter für andere Verknüpfungssymbole, die in speziellen Beispielen auftreten können.

\section{Beispiel}
\begin{enumerate}[label=(\alph*)]
  \item Addition + und Multiplikation $\cdot$ sind Verknüpfungen auf $\mathbb{N}$, $\mathbb{Z}$, $\mathbb{Q}$, $\mathbb{R}$, $\mathbb{C}$.
  
  Multiplikation ist \underline{keine} Verknüpfung auf der Menge der \underline{negativen} ganzen Zahlen.
  
  \item Division ist \underline{keine} Verknüpfung auf $\mathbb{N}$, $\mathbb{Z}$.
  
  Division ist Verknüpfung auf $\mathbb{Q} \setminus \lbrace 0 \rbrace$, $\mathbb{R} \setminus \lbrace 0 \rbrace$.
  
  \item $\mathbb{Z}_n$ := $\lbrace$ 0,1,...,n-1 $\rbrace$ (n $\in \mathbb{N})$
  
  a $\oplus$ b := (a+b) mod n $\in \mathbb{Z}_n$
  
  a $\odot$ b := (a $\cdot$ b) mod n $\in \mathbb{Z}_n$
  
  Verknüpfungen auf $\mathbb{Z}_n$
  
  \item M Menge, X = Menge aller Abb. $M \rightarrow M$
  
  Verknüpfung auf X: Hintereinanderausführung von Abb. $\circ$
  
  f, g : $M \rightarrow M$, so f $\circ$ g : $M \rightarrow M$

  (f $\circ$ g) (m) = f(g(m)) $\in$ M
  
  Im Allgemeinen ist $g \circ f \neq f \circ g$
  
  \item X = $\lbrace 0, 1 \rbrace$
  
  2-stellige aussagenlogische Junktoren wie $\vee, \wedge, XOR, \Rightarrow, ...$ liefern Verknüpfungen auf X.
  
  0 $\vee$ 0 = 0, 1 $\vee$ 0 = 1 
  
  0 $\wedge$ 0 = 0, 1 $\wedge$ 0 = 0, 1 $\wedge$ 1 = 1 (= Multiplikation)
  
  0 XOR 0 = 0, 1 XOR 0 = 1, 1 XOR 1 = 0 (= Addition mod 2)
  
  \item X = $M_n$ ($\mathbb{R}$) = Menge der n $\times$ n - Matrizen über $\mathbb{R}$
  
  Matrizenaddition ist Verknüpfung auf X. Matrizenmultiplikation ist Verknüpfung auf X.
  
  \item M Menge, X = Menge aller endlichen Folgen von Elementen aus M ("Wörter" über M).
  
  Verknüpfung: Hintereinanderausführung zweier Folgen, \underline{Konkatenation}.
  
  M = $\lbrace 0, 1 \rbrace$
  
  $w_1 = 1101$, $w_2 = 001$, $w_1w_2=1101001$, $w_2w_1 = 0011101$
\end{enumerate}

\section{Definition Halbgruppe, Gruppe, Monoid}

Sei X $\neq \emptyset$ eine Menge mit Verknüpfung $\ast$.

\begin{enumerate}[label=(\alph*)]
  \item X, genauer (X, $\ast$) ist \underline{Halbgruppe}, falls (a $\ast$ b) $\ast$ c = a $\ast$ (b $\ast$ c) für alle a,b,c $\in X$ (\underline{Assoziativgesetz})
  
  \item (X, $\ast$) heißt \underline{Monoid}, falls (X, $\ast$) Halbgruppe ist und ein e $\in X$ existiert mit e $\ast$ a, a $\ast$ e = a f.a a $\in X$.
  
  e heißt \underline{neutrales Element}. (Später: es ist eindeutig bestimmt)
  
  \item Sei (X, $\ast$) ein Monoid.
  
  Ein Element a $\in X$ heißt \underline{invertierbar}, falls b $\in X$ existiert (abhängig von a) mit a $\ast$ b = b $\ast$ a = e.
  
  b heißt \underline{inverses Element} (das \underline{Inverse}) zu a. (Später: Wenn b existiert, so ist es eindeutig).
  
  \item Monoid (X, $\ast$) heißt \underline{Gruppe}, falls jedes Element in x bezüglich $\ast$ invertierbar ist.
  
  \item Halbgruppe, Monoid, Gruppe (X, $\ast$) heißt \underline{kommutativ} (oder abelsch), falls a $\ast$ b = b $\ast$ a, für alle a,b $\in X$ (Kommutativgesetz)
\end{enumerate}

\section{Bemerkung}
In Halbgruppe liefert jede sinvolle Klammerung eines `Produktes' mit endlich vielen Faktoren das gleiche Element.

\par \medskip
n = 4 

(a $\ast$ (b $\ast$ c)) $\ast$ d = ((a $\ast$ b) $\ast$ c) $\ast$ d = (a $\ast$ b) $\ast$ (c $\ast$ d) = a $\ast$ (b $\ast$ (c $\ast$ d)) = a $\ast$ ((b $\ast$ c) $\ast d$) (Assoziativgesetz)

\par \medskip

Klammern werden meist weggelassen:

$a^n$ = a $\ast$ ... $\ast$ a  ``Potenzen'' eindeutig definiert. ($n \in \mathbb{N}$)

\section{Proposition}
\begin{enumerate}[label=(\alph*)]
  \item In einem Monoid (X, $\ast$) ist das neutrale Element eindeutig bestimmt.
  \item Ist (X, $\ast$) Monoid und ist $a \in X$ invertierbar, so ist das Inverse zu a eindeutig bestimmt.

  Beziehung.: $a^{-1}$
  \item Ist (X, $\ast$) Monoid und wenn $a,b \in X$ invertierbar sind, so auch a $\ast$ b und (a $\ast$ b)$^{-1}$ = $b^{-1} \ast a^{-1}$

  \item Die Menge der invertierbaren Elemente in einem Monoid (X, $\ast$) bilden bezüglich $\ast$ eine Gruppe.
\end{enumerate}

\subsection{Beweis}

\begin{enumerate}[label=(\alph*)]
  \item Angenommen $e_1, e_2$ sind neutrale Elemente. Dann:

  \begin{equation}
    e_1 = e_1 \ast e_2 = e_2 = e_1 \ast e_2
  \end{equation}

  \item Angenommen a hat 2 inverse Elemente $b_1, b_2$; also 

  \begin{equation}
    \begin{split}
      a \ast b_1 = e, b_2 \ast a = e \\
      b_1 = e \ast b_1 = (b_2 \ast a) \ast b_1 \\
      = b_2 \ast (a \ast b_1) = b_2 \ast e = b_2
    \end{split}
  \end{equation}

  \item (a $\ast$ b) $\ast$ ($b^{-1} \ast a^{-1}$) 

  = $a \ast e \ast a^{-1} = a \ast a^{-1} = e$

  Analog: $(b^{-1} \ast a^{-1}) \ast (a \ast b) = e$

  Also: $(a \ast b)^{-1} = b^{-1} \ast a^{-1}$

  \item I = Menge der invertierten Elemente in (X, $\ast$)

  $e \in I$, dann $e \ast e = e$, d.h. $e^{-1} = e$.

  $\ast$ ist Verknüpfung auf I. Z.z. $a,b \in I \Rightarrow a \ast b \in J$

  Folgt aus c). 

  Assoziativgesetz gilt in I

  $a \in I \Rightarrow a^{-1} \in I$, denn $(a^{-1})^{-1} = a$
  \end{enumerate}

  \subsection{Bemerkung}

  Multiplikation mit $a^{-1}$ macht Multiplikation mit a (Verkn.) rückgängig: 

  \begin{equation}
    \begin{split}
      (b \ast a) \ast a^{-1} = b \ast (a \ast a^{1}) = b \ast e = b \\
      a^{-1} \ast (a \ast b) = b
    \end{split}
  \end{equation}

  \section{Beispiel}

  \begin{enumerate}[label=(\alph*)]
    \item $\mathbb{N}, \mathbb{Z}, \mathbb{Q}, \mathbb{R}, \mathbb{C}$ sind Halbgruppen bezüglich +

    $\mathbb{Z}, \mathbb{Q}, \mathbb{R}, \mathbb{C}$ sind bezüglich + Monoide, neutrales Element 0.

    $\mathbb{N} = \lbrace 1,2,... \rbrace$ ist kein Monoid bezügich +, aber $\mathbb{N}_0$

    $\mathbb{Z}, \mathbb{Q}, \mathbb{R}, \mathbb{C}$ sind Gruppen bezüglich +, inverses Element zu a : -a

    $\mathbb{N}_0$ ist keine Gruppe bezüglich. + 

    Invertierte Elemente in $\mathbb{N}_0: \lbrace 0 \rbrace$.

    \item $\mathbb{N}, \mathbb{Z}, \mathbb{Q}, \mathbb{R}, \mathbb{C}$ sind Monoide bezüglich $\cdot$ (Multiplikation) (neutrales Element: 1)

    Keine Gruppen (da 0 nicht invertierbar ist).

    $\mathbb{Q} \setminus \lbrace 0 \rbrace, \mathbb{R} \setminus \lbrace 0 \rbrace, \mathbb{C} \setminus \lbrace 0 \rbrace$ Gruppen.

    Intervierbare Elemente in $\mathbb{Z}: \lbrace 1, -1 \rbrace$ (Gruppe bezüglich Multiplikation)

    \item M Menge. 

    X = Menge aller Abbildungen M $\rightarrow$ M mit Hintereinanderausführung $\circ$ als Verknüpfung.

    Monoid, neutrales Element $id_M$

    \begin{equation}
      f \circ id_M = f = id_M \circ f
    \end{equation}

    Invertierbar sind genau die bijektiven Abbildungen M $\rightarrow$ M

    Inverse = Umkehrabbildung

    $f: M \rightarrow M$ bijektiv:

    \begin{equation}
      f \circ f^{-1} = f^{-1} \circ f = id_M
    \end{equation}

    1.5.d): Die bijektiven Abbildungen M $\rightarrow$ M bilden bezüglich $\circ$ eine Gruppe.

    \item M Menge, z.B. $\lbrace 0, 1 \rbrace$

    X  = Menge aller endlichen Folgen über M.

    Verknüpfung: Konkatenation (Hintereinanderausführung).

    $\rightarrow$ Halbgruppe

    Nimmt man die leere Folge hinzu, so ist sie das neutrale Element (einziges invertierbares Element).

    Dann: Monoid.

    \item $M_n$($\mathbb{R}$) = Menge der $n \times n$ - Matrizen über $\mathbb{R}$

    Addition: neutrales Element Nullmatrix 

    Inverses zu A ist -A.

    $\rightarrow$ Gruppe

    Multiplikation: $(A \cdot B) \cdot C = A \cdot (B \cdot C)$ 

    $\rightarrow$ Halbgruppe

    Neutrales Element: Einheitsmatrix

    \item $n \in \mathbb{N}$

    $\mathbb{Z}_n =  \{0,1,...,n-1\}$, Verknüpfung $\oplus$

    a $\oplus$ b := a + b mod n

    ($\mathbb{Z}_n$, $\oplus$) ist gruppe, Assoziativgesetz: a,b,c $\in \mathbb{Z}_n$.

    \begin{equation}
      \begin{split}
        (a \oplus b) \oplus c = ((a+b mod n) + c) \ mod \ n \\
        = ((a+b)+c) \ mod \ n \\
        = (a+(b+c))\  mod \ n \\
        = (a+(b+c) \ mod \ n) \ mod \ n \\
        = (a+(b \oplus c)) \ mod \ n \\
        = a \oplus (b \oplus c)
      \end{split}
    \end{equation}

    0 ist neutrales Element bezüglich $\oplus$.

    0 ist sein eigenes Inverses.

    $1 \leq i \le n-1: n-i \in \mathbb{Z}_n$ Inverses zu i.

    \begin{equation}
      \begin{split}
        i \oplus (n-i) \\
        = (i+(n-i)) \ mod \ n \\
        = n \ mod\  n \\
        = 0
      \end{split}
    \end{equation}

    \item $n \in \mathbb{N}, \mathbb{Z}_n$

    Verknüpfung $\odot$ = a $\cdot$ b mod n, $n > 1$

    ($\mathbb{Z}_n$, $\odot$) ist Monoid, Assozisativgesetz wie bei $\oplus$.

    1 ist neutrales Element bezüglich $\odot$. 

    Keine Gruppe bezüglich $\odot$, denn z.B. hat 0 kein Inverses.

  \end{enumerate}

  \section{Satz}

  Sei $n \in \mathbb{N}, n > 1$

  \begin{enumerate}[label=(\alph*)]
    \item Die Elemente in ($\mathbb{Z}_n$, $\odot$), die invertierbar bezüglich $\odot$ sind, sind genau diejenigen $a \in \mathbb{Z}_n$ mit ggT(a, n) = 1.

    Für solche a bestimmt man das Inverse folgendermaßen:

    Bestimme $s, t \in \mathbb{Z}$ mit:

    \begin{equation}
      s \cdot a + t \cdot n = 1
    \end{equation}

    (Erweiterter Euklidischer Algorithmus, Mathe I)

    Dann ist $a^{-1} = s \ mod \ n$

    \item $\mathbb{Z}_{n}^{\ast} := \{a \in \mathbb{Z}_n : ggT(a,n) = 1\}$ ist Gruppe bezüglich $\odot$.

    $|\mathbb{Z}_{n}^{\ast}| =: \varphi(n)$ \underline{Eulersche $\varphi$-Funktion} (L. Euler, 1707-1783).

    \item Ist p eine primzahl, so ist ($\mathbb{Z}_p \setminus \{0\}, \odot$) eine Gruppe. Beweis folgt aus b).
  \end{enumerate}

  \subsection{Beweis}

  \begin{enumerate}[label=(\alph*)]
    \item Angenommen $a \in \mathbb{Z_n}$ invertierbar bezüglich $\odot$.

    Das heißt es existiert $b \in \mathbb{Z}_n$ mit $a \odot$ b = 1. 

    a $\cdot$ b mod n = 1, das heißt es existiert $k \in \mathbb{Z}$ mit:

    \begin{equation}
      a \cdot b = 1 + k \cdot n, 1 = a \cdot b - k \cdot n
    \end{equation}

    Sei d = ggT(a, n).

    \begin{equation}
      \begin{split}
        d|a \rightarrow d | a \cdot b \\
        d|n \rightarrow d | k \cdot n \\
        \Rightarrow d|a \cdot b - k \dot n = 1 \\
        \Rightarrow d = 1.  ggT(a, n) = 1.
      \end{split}
    \end{equation}

    Umgekehrt sei $a \in \mathbb{Z}_n$ mit ggT(a,n) = 1.

    EEA (Erweiterter Euklidischer Algorithmus) liefert $s, t \in \mathbb{Z}$ mit $s \cdot a + t \cdot n = 1$.

    \begin{equation}
      \begin{split}
        (s \ mod \ n) \odot a \\
        = ((s \ mod \ n) \cdot a) \ mod \ n \\
        = (s \cdot a) \ mod \ n \\
        = (1-t \cdot n) \ mod \ n \\
        = (1-(t \cdot n) \ mod \ n) \ mod \ n \\
        = 1 \ mod \ n  \\
        = 1.
      \end{split}
    \end{equation}

    \item 1.5.d) und Teil a)

  \end{enumerate}

  \section{Beispiel}

  n = 24, a = 7 ist invertierbar in ($\mathbb{Z}_{24}, \odot$). 

  EEA:

  1 = (-2) $\cdot$ 24 + 7 $\cdot$ 7.

  $a^{-1}$ = 7 mod 24 = 7 = a.

  \section{Beispiel symmetrische Gruppe}

  Sei M = $\{1,...,n\}$.

  Die Menge der bijektiven Abbildungen auf M (\underline{Permutation}) bilden nach 1.6.c) eine Gruppe bezüglich der Hintereinanderausführung $\circ$.

  \par \medskip

  Bezeichnung: $S_n$, \underline{symmetrische Gruppe vom Grad n}.

  Es ist $|S_n|$ = $n!$.

  Zum Beispiel $\pi$ = $\begin{pmatrix}
                          1 & 2 & 3  \\
                          3 & 2 & 1 
                        \end{pmatrix} \in S_3$,  $\pi^{-1}$ = $\begin{pmatrix}
                                                               1 & 2 & 3 \\
                                                               3 & 2 & 1 
                                                            \end{pmatrix}$ = $\pi$

  
  $\rho$ = $\begin{pmatrix} 1 & 2 & 3 \\ 3 & 1 & 2 \end{pmatrix}$ $\in S_3$, $\rho^{-1}$ = $\begin{pmatrix} 1 & 2 & 3 \\ 2 & 3 & 1 \end{pmatrix}$

  $\Rightarrow$ $\rho \circ \rho^{-1} = id$

  \par \medskip

  $\pi \circ \rho = \begin{pmatrix}1 & 2 & 3 \\ 1 & 3 & 2 \end{pmatrix}$ $\neq$ $\rho \circ \pi = \begin{pmatrix}1 & 2 & 3 \\ 2 & 1 & 3 \end{pmatrix}$

  $S_n$ ist für $n \ge 3$ nicht abelsch (nicht kommutativ).

  \section{Satz (Gleichungslösen in Gruppen)}

  Sei (G, $\cdot$) eine Gruppe, $a,b \in G$. (In allgemeinen Gruppen schreibt man Verknüpfung oft als $\cdot$ statt $\ast$, oft auch $ab$ statt $a \cdot b$).

  \begin{enumerate}[label=(\alph*)]
    \item Es gibt genau ein $x \in G$ mit $ax = b$ (nämlich $x = a^{-1} \cdot b$).

    [''Teilen'' durch a von links = Multiplikation von links mit $a^{-1}$]

    \item Es gibt genau ein $y \in G$ mit $ya = b$ (nämlich $y = ba^{-1}$).

    \item Ist $ax = bx$ für ein $x \in G$ so ist $a = b$.

    Ist $ya = yb$ für ein $y \in G$ so ist $a=b$.
  \end{enumerate}

  \subsection{Beweis}

  \begin{enumerate}[label=(\alph*)]
    \item Setze $x = a^{-1} \cdot b \in G$.

    \begin{equation}
      a \cdot (a^{-1} \cdot b) = (a \cdot a^{-1}) \cdot b = e \cdot b = b.
    \end{equation}

    \textbf{Eindeutigkeit:} Sei $x \in G$ mit $ax = b$.

    Multipliziere beide Seiten mit $a^{-1}$.

    \begin{equation}
      x = (a^{-1}a)x = a^{-1}(ax) = a^{-1}b
    \end{equation}

    \item analog

    \item $ax = bx$, multiplikation mit $x^{-1}$ von rechts.

    Dann $a=b$.
  \end{enumerate}

  \section{Beispiel}

  \begin{enumerate}[label=(\alph*)]
    \item Suche Permutation $\xi \in S_3$ mit $\rho \circ \xi = \pi$ (vgl. 1.9)

    1.10.a):

    $\xi = \rho^{-1} \circ \pi$ = $\begin{pmatrix}1 & 2 & 3 \\ 2 & 3 & 1 \end{pmatrix}$ $\circ$ $\begin{pmatrix} 1 & 2 & 3 \\ 3 & 2 & 1 \end{pmatrix}$ = $\begin{pmatrix}1 & 2 & 3 \\ 1 & 3 & 2 \end{pmatrix}$

    \item 1.10.c) gilt in Monoiden, die keine Gruppen sind, im Allgemienen nicht:

    \textbf{Beispiel:} ($\mathbb{Z}_6, \odot$)

    $2 \odot 3 = 0 = 4 \odot 3$, aber $2 \neq 4$.
  \end{enumerate}

   \section{Definition Ring}

  \begin{enumerate}[label=(\alph*)]
    \item  R $\neq \emptyset$ Menge mit 2 Verknüpfungen + und $\cdot$ heißt \underline{Ring}, falls gilt:
   
      \begin{enumerate}
        \item (R, +) ist kommutative Gruppe (neutrales Element: 0, \underline{Nullelement}, Inverses zu a: -a, b+(-a) =: b-a)
        \item (R, $\cdot$) ist Halbgruppe
        \item (a+b) $\cdot$ c = a $\cdot$ c + b $\cdot$ c und a $\cdot$ (b+c) = a $\cdot$ b + a $\cdot$ c ($\cdot$ vor +), für alle $a,b,c \in R$.

        \underline{Distributivgesetz}
      \end{enumerate}

    \item Ring heißt \underline{kommutativer Ring} falls (R, $\cdot$) kommutative Halbgruppe ist.
    \item Ring R heißt \underline{Ring mit Eins}, falls (R, $\cdot$) Monoid mit neutralem Element 1 $\neq$ 0 (Einselement, \underline{Eins})
  \end{enumerate}

  \section{Beispiele zu Ringen}

  \begin{enumerate}[label=(\alph*)]
    \item ($\mathbb{Z},+, \cdot$) ist kommutativer Ring mit 1, invertierbare Elemente bezüglich $\cdot$ sind 1 und -1.
    \item ($\mathbb{Q},+, \cdot$), ($\mathbb{R}, +, \cdot$), ($\mathbb{C}, +, \cdot$) sind kommutative Ringe mit Eins. Alle Elemente $\neq$ 0 sind invetierbar bezüglich $\cdot$.
    \item $n \in \mathbb{N}, n > 1$. $\mathbb{Z}_n = \{0, ..., n-1\}$

    ($\mathbb{Z}, \oplus, \odot$) ist kommutativer Ring mit Eins: 

    \textbf{Beweis:} Wegen 1.6.f),g) sind nur die Distributivgesetze zu zeigen:

    \begin{equation}
      \begin{split}
        (a \oplus b) \odot c = ((a \oplus b) \cdot c) \mod n \\
        = (((a+b) \mod n) \cdot c) \mod n \\
        \underbrace{=}_{Mathe \ I} ((a+b) \cdot c) \mod n \\
        = (a \cdot c + b \cdot c) \mod n \\
        \underbrace{=}_{Mathe \ I} ((a \cdot c) \mod n + (b \cdot c) \mod n) \mod n \\
        = a \odot c \oplus b \odot c
      \end{split}
    \end{equation}

    \item $M_n(\mathbb{R})$, $n \times n$ - Matrizen über $\mathbb{R}$, mit Matrizenaddition + und Multiplikaiton $\cdot$, ist Ring mit Eins. (Folgt aus Rechenregeln für Matrizen, Mathe II)

    Eins: $E_n$, $n \times n$ - Einheitsmatrix

    Für $n \geq 2$ ist $M_n(\mathbb{R})$ kein kommutativer Ring.
  \end{enumerate}

  \section{Proposition}

  Sei (R, +, $\cdot$) ein Ring. Dann gilt für alle $a,b \in R$:

  \begin{enumerate}[label=(\alph*)]
    \item $0 \cdot a = a \cdot 0 = 0$
    \item $(-a) \cdot b) = a \cdot (-b) = - (a \cdot b)$
    \item $(-a) \cdot (-b) = a \cdot b$
  \end{enumerate}

  \subsection{Beweis}

  \begin{enumerate}[label=(\alph*)]
    \item $0 \cdot a = (0+0) \cdot a \underbrace{=}_{Distr.} 0 \cdot a + 0 \cdot a$

    Addiere auf beiden Seiten $-(0 \cdot a)$

    $0 = 0 \cdot a +0 = 0 \cdot a$

    $a \cdot 0 = 0$ analog.

    \item $(-a) \cdot b + a \cdot b = ((-a) + a) \cdot b = 0 \cdot b \underbrace{=}_{(a)} 0$

    $\Rightarrow (-a) \cdot b = - (a \cdot b)$. Analog $a \cdot (-b) = - (a \cdot b)$.

    \item $(-a) \cdot (-b) \underbrace{=}_{(b)} - (a \cdot (-b)) \underbrace{=}_{(b)} - (- (a \cdot b)) = a \cdot b$
  \end{enumerate}

  \section{Bemerkung Ringe}

  \begin{enumerate}[label=(\alph*)]
    \item In einem Ring mit Eins sind 1 und -1 bezüglich $\cdot$ invertierbar:

    $1 \cdot 1 = 1$  ($1^{-1} = 1$) 

    $(-1) \cdot (-1) = 1$  (1.14.c)), d.h. $(-1)^{-1} = -1$.

    0 ist \textbf{nie} bezüglich Multiplikation invertierbar, denn $0 \cdot a \underbrace{=}_{1.14.a)} 0 \neq 1$.

    \item Es kann sein, dass 1 = -1 gilt.

    \textbf{Beispiel:} ($\mathbb{Z}_2, \oplus, \odot$).

    $1 \oplus 1 = 0$ $\rightarrow 1 = -1$
  \end{enumerate}

  \section{Definition Körper}

  Ein kommutativer Ring (R, +, $\cdot$) mit Eins heißt \underline{Körper}, wenn jedes Element $\neq$ 0 bezüglich der Multiplikation invertierbar ist.

  \section{Beispiel Körper}

  \begin{enumerate}[label=(\alph*)]
    \item $\mathbb{Q}, \mathbb{R}, \mathbb{C}$ sind Körper, $\mathbb{Z}$ nicht.
    \item ($\mathbb{Z}_n, \oplus, \odot)$ ist genau dann ein Körper, wenn n eine Primzahl ist.

    \textbf{Begründung:} 

    $\mathbb{Z}_n$ ist kommutativer Ring mit 1.

    1.13.c): 

    Die invertierbaren Elemente in $\mathbb{Z}_n$ sind alle $a \in \mathbb{Z}_n$ mit $ggT(a,n) = 1$.
  \end{enumerate}

  \section{Proposition (Nullteilerfreiheit) in Körpern}

  Ist K ein Körper, $a,b \in K$ mit $a \cdot b = 0$, so ist a = 0 oder b = 0.

  \subsection{Beweis}

  Sei $a \cdot b = 0$. Angenommen $a \neq 0$.

  Dann existiert $a^{-1} \in K$.

  \begin{equation}
    \begin{split}
      0 \underbrace{=}_{1.14.a)} a^{-1} \cdot 0 = a^{-1} (a \cdot b) \\
      = (a^{-1} \cdot a) \cdot b) = 1 \cdot b = b
    \end{split}
  \end{equation}

  \textbf{Beispiel:} R = ($\mathbb{Z}_6, \oplus, \odot$)

  $2 \odot 3 = 0$, aber $2 \neq 0, 3 \neq 0$
  
  \section{Definition Polynom}

  Sei K ein Körper. 

  \begin{enumerate}[label=(\alph*)]
    \item Ein (formales) \underline{Polynom} über K ist ein Ausdruck.

    \begin{equation}
      f = a_0 + a_1x+ a_2x² + ... + a_nx^n, n \in \mathbb{N}_0, a_i \in K
    \end{equation}

    (Manchmal f(x) statt f, +-Zeichen hat zunächst nichts mit Addition zu tun).

    $a_i:$ \underline{Koeffizienten} von f

    Ist $a_i$ = 0, so kann man in der Schreibweise von f 0 $\cdot x^i$ auch weglassen.

    \par \medskip

    Statt $a_0x^0$ schreibt man $a_0$, statt $a_1x^1$ schreibt man $a_1x$.

    Sind alle $a_i = 0$, so $f = 0$, \underline{Nullpolynom}.

    Ist $a_i = 1$, so schreibt man $x^i$ statt $1 \cdot x^i$.

    \item Zwei Polynome f und g sind \underline{gleich}, wenn \underline{entweder} f = 0 und g = 0 \underline{oder} f = $\sum_{i=0}^{n} a_ix^i, a_n \neq 0$, g = $\sum_{i=0}^{m} b_ix^i, b_,m \neq 0$, und n = m und $a_i = b_i$ für i=0,...,n

    \item Menge aller Polynome über K: $K[x]$

    Wir wollen $K[X]$ zu einem Ring machen. Wie?

    \textbf{Beispiel:} $f = 3x^2 + 2x + 1, g = 5x³ + x² + x \in \mathbb{Q}[x]$

    $f+g = 5x³ + 4x² + 3x + 1$

    $f \cdot g = (3x² + 2x +1) \cdot (5x³ + x² + x) = 15x^5 + 13x^4 + 10x^3 + 3x^2 + x$
 \end{enumerate}

 \section{Satz und Definition}

 K Körper. $K[x]$ wird zu einem kommutativen Ring mit Eins durch folgende Verknüpfungen:

  \begin{equation}
   f = \sum_{i=0}^{n} a_ix_i, g = \sum_{i=0}^{n} b_ix^i
  \end{equation}
  
  so:

  \begin{equation}
    f+g := \sum_{i=0}^{max(n,m)} (a_i+b_i)x^i
  \end{equation}

  \begin{equation}
    f \cdot g = \sum_{i=0}^{n+m} c_ix^i, 
  \end{equation}

  wobei:

  \begin{equation}c_i = \sum_{j=0}^{i} a_jb_{i-j} = a_0b_i + a_1b_{i-1} + ... +a_ib_0 \textnormal{ (Faltungsprodukt)}
  \end{equation}

  In beiden Fällen sind Koeffizienten $a_i$ mit $i > n$ beziehungsweise $b_i$ mit $i > m$ gleich 0 zu setzen.

  Das Einselement ist 1 (= $1x^0$).

  Das Nullelement ist das Nullpolynom.

  \begin{itemize}
    \item f = $\sum_{i=0}^{n} (-a_i)x^i$
  \end{itemize}

  ($K[x]$, +, $\cdot$) heißt \underline{Polynomring} in einer Variablen.

  \par \medskip

  \textbf{Beweis:} Nachrechnen

  \section{Bemerkung}

  \begin{enumerate}[label=(\alph*)]
    \item $f = \sum_{i=0}^{n} a_ix^i \in K[x], a \in K \subseteq K[x]$

    $a \cdot f = \sum_{i=0}^{n} (a \cdot a_i)x^i$

    $x \cdot f = \sum_{i=0}^{n} a_ix^{i+1} = a_nx^{n+1} + ... + a_0x$

    \item Das +-Zeichen in der Definition der Polynome entspricht genau der Addition der ``\underline{Monome}'' $a_ix^i$.

    $((a_0x^0) + (a^1x^1)) = a_0x^0 + a_1x^1$
  \end{enumerate}

  \section{Definition}

  Sei $0 \neq f \in K[x]$, $f = \sum_{i=0}^{n} a_ix^i, a_n \neq 0$.

  Dann heißt n der \underline{Grad} von f, Grad(f) = n

  \par \medskip

  Grad(0) := -$\infty$

  Grad(f) = 0: \underline{Konstante Polynome} $\neq$ 0.

  \section{Satz}

  Sei K ein Körper, $f,g \in K[x]$. Dann ist Grad($f \cdot g$) = Grad(f) + Grad(g)

  (Konvention : -$\infty +n = n+ (- \infty) = (-\infty) +(- \infty) = - \infty$)

  \par \medskip

  \textbf{Beweis:} Richtig, falls f = 0 oder g = 0. Sei $f \neq 0, g \neq 0$.

  \begin{equation}
    f = \sum_{i=0}^{n} a_ix^i, a_n \neq 0, \textnormal{ n = Grad(f)}
  \end{equation}

  \begin{equation}
   g = \sum_{i=0}^{m} b_ix^i, b_m \neq 0, \textnormal{ m = Grad(g)}
  \end{equation}

  Koeffizient von $x^{n+m}$ in $f \cdot g: a_n \cdot b_m \underbrace{\neq}_{1.18} 0$

  Höhere Potenzen mit Koeffizient $\neq 0$ treten in $f \cdot g$ nicht auf. Also: Grad($f \cdot g$) = n+m = Grad(f) + Grad(g)

  \section{Korollar}

  Sei K ein Körper.

  \begin{enumerate}[label=(\alph*)]
    \item Genau die konstanten Polynome $\neq$ 0 sind in $K[x]$ bezüglich $\cdot$ invertierbar. Insbesondere ist $K[x]$ \underline{kein} Körper.

    \item Sind $f,g \in K[x]$ mit $f \cdot g$ = 0, so ist f = 0 oder g = 0 (Nullteilerfreihit in $K[x]$).

    \item Sind $f,g_1,g_2 \in K[x]$ mit $f \cdot g_1 = f \cdot g_2$ und ist $f \neq 0$, so ist $g_1 = g_2$.
  \end{enumerate}

  \subsection{Beweis}
    \begin{enumerate}[label=(\alph*)]
      \item Sei $f \in K[x]$ invertierbar bezüglich $\cdot$.

      Dann ist $f \neq 0$ und es existiert $g \in K[x]$ mit $f \cdot g = 1$.

      Mit 1.23:

      0 = Grad(1) = Grad($f \cdot g$) = Grad(f) + Grad(g).

      Also: Grad(f) = 0 (= Grad(g))

      D.h. f ist konstantes Polynom.

      Ist umgekehrt $f = a \in K, a \neq 0$, so $f^{-1} = a^{-1} \in K$.

      \item Folgt aus 1.23:

      - $\infty$ = Grad(0) = Grad($f \cdot g$) = Grad(f) + Grad(g)

      $\Rightarrow$ Grad(f) = - $\infty$ oder Grad(g) = - $\infty$, d.h. f = 0 oder g = 0.

      \item $fg_1 = fg_2$ 

      $\Rightarrow 0 = fg_1 - fg_2 = f (g_1 - g_2)$

      Da $f \neq 0$, folgt mit b) $g_1-g_2 = 0$, d.h. $g_1 = g_2$.
    \end{enumerate}

    \section{Bemerkung}

    \begin{enumerate}[label=(\alph*)] 
      \item Jedem Polynom $f =  \sum_{i=0}^{n} a_ix^i \in K[x]$ kann man eine Funktion $K \rightarrow K$ zuordnen: $a \in K \mapsto f(a) = \sum_{i=0}^{n} a_ia^i \in K$ (Polynomfunktion aus Analysis für $K = \mathbb{R}$)

      Auf Grund der Definition von Addition / Multiplikation von Polynomen gilt:

      \begin{equation}
        \begin{split}
          \underbrace{(f+g)}_{K[x]} (a) = f(a) \underbrace{+}_{K} g(a) \\
          \underbrace{(f \cdot g)}_{K[x]} (a) = f(a) \underbrace{\cdot}_{K} g(a)
        \end{split}
      \end{equation}

      Es kann passieren, dass zwei verschiedene Polynome die gleiche Funktion beschreiben:

      \textbf{Zum Beispiel:} K = $\mathbb{Z}_2$ = \{0,1\}

      $f = x², g = x, f \neq g$

      f(1) = 1 = g(1)

      f(0) = 0 = g(0)

      Über unendlichen Körpern passiert das nicht (später).

      \item Schnelel Berechnung von f(a):

      $f = a_0+a_1x+...+a_nx^n$

      f(a) = $a_0 + a(a_1+a(a_2+...+a(a_{n-1}+aa---.n) ...))$

      n Multiplikation

      n Addition

      \textbf{\underline{Horner-Schema}}
    \end{enumerate}

    \section{Definition} 

    K Körper, $f,g \in K[x]$.

    f \underline{teilt} g ($f|g$), falls $q \in K[x]$ existiert mit

    \begin{equation}
      g = q \cdot f
    \end{equation}

    (Falls $g \neq 0$ und $f|g$, so ist der Grad(f) $\le$ Grad(g) nach 1.23)

    \section{Satz}

    K Körper, $0 \neq f \in K[x], g \in K[x]$. Dann existieren eindeutig bestimmte Polynome $q,r \in K[x]$ mit 

    \begin{enumerate}
      \item $g = q \cdot f +r$
      \item Grad(r) $<$ Grad(f)
    \end{enumerate}

    \underline{Division mit Rest}

     $q =: g \textnormal{ div} f$

     $r =: g \mod f$

     (Beweis: WHK, Satz 4.69)

     \section{Beispiel}

    \begin{enumerate}[label=(\alph*)]
      \item $g = x^4 + 2x^3 - x + 2$, $f = 3x² - 1$, f,g $\in \mathbb{Q}[x]$

      \underline{$x^4$}$ + 2x^3 - x + 2 : \  $\underline{$3x²$} $-1 = \underbrace{\frac{1}{3}x² + \frac{2}{3}x + \frac{1}{9}}_{q}$

      %\polyset{style=C, div=:,vars=x}
      %\polylongdiv{x⁴ + 2x³ - x + 2}{3x² -1}

      -($x^4 - \frac{1}{3}x²$)

      ------------------------

      $\ \ \ \ \ \ $\underline{$2x³$}$ + \frac{1}{3}x² - x + 2$

      $\ \ \ \ \ - (2x³ - \frac{2}{3}x)$

      $\ \ \ \ \ $ ------------------------

      $\ \ \ \ \ \ \ \ \ \ $\underline{$\frac{1}{3}x²$} $ - \frac{1}{3}x +2$

      $\ \ \ \ \ \ \ \ \ - (\frac{1}{3}x² - \frac{1}{9})$

      $\ \ \ \ \ \ \ \ \ $ -----------------------------

      $\ \ \ \ \ \ \ \ \ \ \ \ \ \ \ \underbrace{- \frac{1}{3}x + \frac{19}{9}}_{r}$

      \item $g = x⁴ - x² + 1, f = x² + x, f,g \in \mathbb{Z}_3[x]$ (-1 = 2 in $\mathbb{Z}_3$)

      $x⁴ + 2x² + 1$ : $x² + x$ = $x²+2x$

      -$(x⁴ + x³)$

      ---------------

      $\ \ \ 2x³ + 2x² + 1$

      $\ \ - (2x³ + 2x²)$

      $\ \ $---------------

      $\ \ \ \ \ \ \ \ \ \ \ \ $ 1

      g div f = $x² + 2x$

      g mod f = 1

      \par \medskip

      $x^4 + 2x² + 1 = (x² + 2x) (x² + x) + 1$

      \section{Korollar}

      K Körper, $a \in K$.

      $f \in K[x]$ ist genau dann durch (x-a) tielbar, wenn $f(a) = 0$ (d.h. a ist \underline{Nullstelle} von f).

      [f = q $\cdot$ (x-a), q $\in K[x]$]

      \subsection{Beweis}

      Falls $(x-a)|f$, so existiert q $\in K[x]$ mit f = q $\cdot (x-a)$. Dann:

      \begin{equation} 
        f(a) = q(a) \cdot (a-a) = 0
      \end{equation}

      umgekerht: Angenommen f(a) = 0. Division mit Rest von f durch x-a:

      \begin{equation}
        f = q \cdot (x-a) +r, \ \ q,r \in K[x]
      \end{equation}

      Grad(r) $<$ Grad(x-a) = 1, $r \in K$

      Zeige: r = 0.

      r = f - q $\cdot$ (x-a)

      Setze $a \in K$ ein.

      r $\underbrace{=}_{1.25a)} = f(a) - q(a) \cdot \underbrace{(a-a)}_{=0} = 0 -0 = 0$

      f = q(x-a)
    \end{enumerate}


    \section{Definition}

    K Körper. $a \in K$ heißt \underline{m-fache Nullstelle} von $f \in K[x]$, falls:

    \begin{equation}
      (x-a)^m|f \textnormal{ und } (x-a)^{m+1} \not{|} f
    \end{equation}

    D.h. $f = q \cdot (x-a)^m$ und $q(a) \neq 0$ 

    \section{Beispiel}

    $f = x⁵ + x⁴ + 1 \in \mathbb{Z}_3[x]$

    In $\mathbb{Z}_{3}$ hat f Nullstelle 1.

    1.29: x-1 (= x+2) teilt ft.

    Dividiere f durch x-1: 

    \begin{equation}
      f = (x⁴ + 2x³ +2x² + 2x +2) : (x-1)
    \end{equation}

    1 Nullstelle von $x⁴ + 2x^3 + 2x^2 + 2x + 2$.

    \begin{equation}
      (x-1)|x^4 + 2x^3 + 2x^2 + 2x +2
    \end{equation}

    \begin{equation}
      x⁴ + 2x³ + 2x² + 2x + 2 : x-1 = x³ + 2x+1
    \end{equation}

    $f = \underbrace{(x³ + 2x + 1)}_{q}  \cdot (x-1)²$

    q(1) = 1 $\neq 0$

    1 ist 2-fache Nullstelle von f

    \section{Satz}

    K Körper, $f \in K[x]$, Grad(f) = $n \geq 0 $ (d.h. f $\neq 0$). Dann hat f höchstns n Nullstellen in K (einschließlich Ivelfachheit).

    Genauer: Sind $a_1,..,a_k$ die verschiedenen Nullstellen von f, so ist $f = g \cdot (x-a_1)^{m_1} ... (x-a_k)^{m_k}$, $m_i$ Vielfachheiten der Nullstellen $a_i$, g hat keine Nullstellen in K.

    g hat keine Nullstellen in K.

    \subsection{Beweis}

    Durch Induktion nach n.

    \par \medskip

    n = 0: f = $a_0 \neq 0$, ohne Nullstellen.

    \par \medskip

    Sei n $>$ 0. Behauptung sei richtig für alle Polynome von Grad $<$ n.

    Hat f keine Nullstellen, g = f

    Hat f Nullstellen $a_1,...,a_k$, $k \geq 1$, so $f = q \cdot (x-a_1)^{m_1}$ (nach Definition).

    q($a_1$) $\neq$ 0.

    Grad(q) = n - $m_1$ $\underbrace{<}_{m_1 > 0}$ n

    Wir zeigen: 

    \par \medskip

    q hat genau die Nullstellen $a_2,...,a_k$ mit Vielfachheiten $m_2,...,m_k$. Klar: Jede Nullstelel von q ist auch Nullstelle von f.

    D.h. q hat höchstens Nullstelle $a_2, ..., a_k$. Diese Nullstellen hat q mit vielfachheit $0 \leq n_i \leq m_i$, denn $(x-a_i)^{n_i}|q$ $\rightarrow$ $(x-a_i)^{n_i}|f$

    Sei $i \in \{2,...,k\}$. Es ist: 

    \begin{equation}
     f = s \cdot (x-a_i)^{m_i}, s \in K[x], s(a_i) \neq 0
    \end{equation}

    \begin{equation}
      q = q_1 \cdot (x-a_i)^{n_i}, q_1 \in K[x], q_1(a_i) \neq0, ((x-a_i)^0 = 1)
    \end{equation}

    \begin{equation}
      f = q \cdot (x-a_i)^{m_i}
    \end{equation}

    $s \cdot (x-a_i)^{m_i - n_i} \cdot (x-a_i)^{n_i} = s \cdot (x-a_i)^{m_i} = f = q_1 (x - a_i)^{n_i} \cdot (x-a_1)^{m_1}$

    \par \medskip

    \textbf{1.24.c):}

    $s (x-a_i)^{m_i - n_i} = q_1 \cdot (x-a_1)^{m_1}$.

    Ist $m_i > n_i$, so ist $m_i - n_i > 0$.

    $0 = s(a_i)(a_i-a_i)^{m_i - n_i} = q_1(a_i)(a_i-a_1)^{m_1} \neq 0$

    D.h. $n_i = m_i$, $i=2,...,k$.

    $q = g \cdot (x-a_2)^{m_2}...(x-a_k)^{m_k}$, g ohne Nullstelle in K (nach Induktionsvoraussetzung).

    $f = g \cdot (x-a_1)^{m_1}...(x-a_k)^{m_k}$.

    \section{Korollar}

    K Körper, $f,g \in K[x]$, m = max(Grad(f), Grad(g))

    Gibt es m+1 Elemente $a_1, ..., a_{m+1} in K$, paarweise verschieden, mit $f(a_i) = g(a_i), i=1,...,m+1$, so f=g

    \par \medskip

    \textbf{Insbesondere:} Ist K unendlich, $f,g, \in K[x]$ mit f(a) = g(a) für alle $a \in K$, so ist f = g.

    \subsection{Beweis}

    $f-g \in K[x]$, Grad($f-g$) $\leq$ m.

    $f-g$ hat m+1 Nullstellen $a_1, ..., a_{m+1}$

    1.32. $f-g = 0$, $f = g$

    \section{Bemerkung}

    Über $\mathbb{Q, R}, \mathbb{Z}_p$ (p Primzahl) gibt es Polynome beliebig hohen Grades ohne Nullstellen.

    Über $\mathbb{Q, R}$: $(x² + 1)^m$ hat Grad 2m, keine Nullstelle in $\mathbb{Q, R}$

    In $\mathbb{C}$ $x² + 1 = (x+i)(x-i)$

    Über $\mathbb{Z}_p$ z.B. $(x^p - x + 1)^m$ hat Grad pm, ohne Nullstellen. (ohne Beweis) 

    \section{Fundamentalsatz der Algebra (C.F. Gauß)}

    Ist $f \in \mathbb{C}[x]$, $f \neq 0$, so ist:

    \begin{equation}
      \begin{split}
      f = a_n(x-c_1)^{m_1}...(x-c_k)^{m_k}, a_n \in \mathbb{C}, c_i,...,c_k \in \mathbb{C} \textnormal{ (Nullstellen mit Vielfachheit $m_1,...,m_k$)} \\
      m_1 +...+m_k = \textnormal{ Grad(f)}
      \end{split}
    \end{equation}

    Grad(f) = n: f hat n Nullstellen (einschließlich Vielfachheit)

    \chapter{Vektorräume}

    Verallgemeinerung von Mathe II, Kap. 10

    \section{Definition}

    Sei K ein Körper. 

    Ein \underline{K-Vektorraum} V besitzt Verknüpfung $+$, bezüglich derer er eine kommutative Gruppe ist. (Neutrales Element $\sigma$, \underline{Nullvektor}; Inverses zu $v \in V$: $-v$)

    Außerdem existiert Abbildung :

    \begin{center}
      $K \times V \rightarrow V$ 

      $(a,v) \mapsto av, a \in K, v \in V$
    \end{center}

    (``Multiplikation'' von Elementen aus $V$ (``Vektoren'') mit Körperelementen (``Skalare'')), so dass gilt:

    \par \medskip

    $(a \underbrace{+}_{in \ K} b) v = av \underbrace{+}_{in \ V} bv$ für alle $a,b \in K, v \in V$

    \par \medskip

    $a (v \underbrace{+}_{in \ V} w) = av \underbrace{+}_{in \ V} aw$ für alle $a \in K, v,w \in V$

    \par \medskip

    $(a \underbrace{\cdot}_{in \ K} b) v = a \cdot (\underbrace{bv}_{\in V})$

    \par \medskip

    $1 v = v$ für alle $v \in V$ (mit 1 neutrales Element in K bezüglich $\cdot$)

    \section{Beispiel Vektorraum}

    \begin{enumerate}[label=(\alph*)]
      \item K Körper, $n \in \mathbb{N}$

      $K^n = \{\begin{pmatrix}a_1 \\ ... \\ a_n \end{pmatrix} : a_i \in K\}$ ist K-Vektorraum begzüglich:

      \begin{equation}
        \begin{pmatrix}a_1 \\ ... \\ a_n \end{pmatrix} + \begin{pmatrix}b_1 \\ ... \\ b_n \end{pmatrix} := \begin{pmatrix}a_1 + b_1 \\ ... \\ a_n + b_n \end{pmatrix}
      \end{equation}

      \begin{equation}
        a \begin{pmatrix}a_1 \\ ... \\ a_n \end{pmatrix} := \begin{pmatrix} a \cdot a_1 \\ ... \\ a \cdot a_n \end{pmatrix}
      \end{equation}

      für alle $a \in K, \begin{pmatrix}a_1 \\ ... \\ a_n \end{pmatrix}, \begin{pmatrix}b_1 \\ ... \\ b_n \end{pmatrix} \in K^n$

      Raum der \underline{Spaltenvektoren} der Länge n über K.

      \par \medskip

      Entsprechend Raum der Zeilenvektoren.

      \begin{equation}
        \begin{pmatrix}a_1 \\ ... \\ a_n \end{pmatrix}  = \begin{pmatrix}a_1,...,a_n\end{pmatrix}^t
      \end{equation}

      Für $K =  \mathbb{R} : \mathbb{R}^n$

      n = 2,3 Elemente aus $\mathbb{R}^2, \mathbb{R}^3$ identifizierbar mit Ortsvektor der Ebene oder des 3-dimensionalen Raums.

      \item Sei K ein Körper.

      Polynomring $K[x]$ ist ein K-Vektorraum bezüglich:

      \begin{itemize}
        \item Addition von Polynomen
        \item Multiplikation von Körperelementen mit Polynomen

        \begin{center}
          $a \cdot (\sum_{i=0}^{n} a_ix^i) := \sum_{i=0}^{n} (aa_i)x^i \in K[x]$, $a \in K, a_i \in K$
        \end{center}

        (Multiplikation von Polynomen mit Polynom vom Grad $\le$ 0 (konstant))

        2.1 folgt aus den Ringeigenschaften von $K[x]$.
      \end{itemize}

      \item K Körper. V = Abb(K,K) = \{$\alpha : K \rightarrow K : \alpha$ Abb.\}

      Addition auf V: 

      $\alpha, \beta \in V$

      $(\alpha + \beta)(x) := \alpha(x) + \beta(x)$, für alle $x \in K$.

      Skalare Mutliplikation:

      $a \in K, \alpha \in V$.

      $(a \cdot \alpha)(x) = a \cdot \alpha(x)$, für alle $x \in K$

      Nachrechnen: Damit wird V ein K-Vektorraum.
    \end{enumerate}

    \section{Prop}
    $K$ Körper, $V$ K-VR.

    \begin{enumerate}[label=(\alph*)]
      \item $a \cdot \sigma$ = $\sigma$, für alle $a \in K$
      \item $0 \cdot v$ = $\sigma$, für alle $v \in V$.
      \item (-1) $\cdot v$ = $-v$ 
    \end{enumerate}

    \subsection{Beweis}

    \begin{enumerate}[label=(\alph*)]
      \item $a \cdot \sigma = a \cdot (\sigma + \sigma) \underbrace{=}_{2.1} a \cdot \sigma + a \cdot \sigma$

      $\Rightarrow a \cdot \sigma = \sigma$

      \item $0 \cdot v = (0+0) \cdot v = 0 \cdot v + 0 \cdot v$

      $\Rightarrow 0 \cdot v = \sigma$

      \item $(-1) \cdot v + v \underbrace{=}_{2.1} (-1) \cdot v + 1 \cdot v \underbrace{=}_{2.1} ((-1)+1) \cdot v \underbrace{=}_{b)} 0 \cdot v = \sigma$

      $\Rightarrow (-1) \cdot v = -v$
    \end{enumerate}

    \section{Definition Unterraum}

    K Körper, V K-VR.

    $\emptyset \neq U \subseteq V$ heißt \underline{Unterraum} (Untervektorraum oder Teilraum) von $V$, falls U begzülich Addition auf V und der skalaren Multiplikation mit Elementen aus K selbst K-Vektorraum ist.

    \section{Prop}

    U ist Unterraum von V $\Leftrightarrow$ 

    \begin{enumerate}
      \item[(1)] $u_1 + u_2 \in U$ für alle $u_1,u_2 \in U$
      \item[(2)] $au \in U$ für allle $u \in U, a \in K$.
    \end{enumerate}

    (Nullvektor in $U$ = Nullvektor in $V$)

    \subsection{Beweis}               
    $\Rightarrow$ klar

    $\Leftarrow$: Da $U \neq \emptyset$, existiert $u \in U$:

    \begin{center}
      $\sigma \underbrace{=}_{2.3.b)} 0 \cdot u \in U$
    \end{center}

    $u \in U \Rightarrow -u = (-1) \cdot u \in U$

    Mit (1): $(U,+)$ ist kommutative Gruppe.

    Restliche Axiome gelten auch für $U,K$.

    \section{Beispiel}

    \begin{enumerate}[label=(\alph*)]
      \item V K-VR, so ist V Unterraum von V und $\{\sigma\}$ ist Unterraum von V (\underline{Nullraum}).

      \item Betrachte $K[x]$ als K-VR (2.2.b)).

      Sei $n \in \mathbb{N}_0$.

      U = $\{f \in K[x] : \textnormal{ Grad(f) $\le$ n}\}$

      Unterraum von $K[x]$.
    \end{enumerate}

    \section{Prop}

    Seien $U_1, U_2$ Unterräume von K-VR V.

    \begin{enumerate}[label=(\alph*)]
      \item $U_1 \cap U_2$ ist Unterraum
      \item $U_1 + U_2 := \{u_1 +u _2: u_1 \in U_1, u_2 \in U_2\}$ ist Unterraum von V (\underline{Summe} von Unterräumen)
      \item $U_1 + U_2$ ist der kleinste Unterraum von $V$, der $U_1 \cup U_2$ enthält
      \item $U_1 \cup U_2$ ist im Allgemeinen kein Unterraum
    \end{enumerate}

    \textbf{Beweis:} Wie 10.4, Mathe II

    \section{Definition}

    V K-VR

    \begin{enumerate}[label=(\alph*)]
      \item $v_1,...,v_m \in V, a_1,...,a_m \in K$

      Dann heißt $a_1v_1 + ...a_mv_m = \sum_{i=1}^{m} a_iv_i \in V$ \underline{Linearkombination} von $v_1,...,v_m$ (mit Koeffizienten $a_1,...,a_m$).

      [Beachte: Zwei formal verschiedene Linearkombinationen derselben Vektoren können den gleichen Vektor darstellen]

      \item Ist $M \subseteq V$, so ist der von M \underline{erzeugte} oder \underline{aufgespannte Unterraum} $<M>_k$ (oder kurz $<M>$) die Menge aller (endlichen) Linearkombinationen, die man mit Vektoren aus M bilden kann:

      $<M>_k = \{\sum_{i=0}^{n} a_iv_i : n \in \mathbb{N}, a_i \in K, v_i \in M\}$

      $< \emptyset >_k := \{ \sigma\}$

      $M =  \{v_1,...v_m\}: \ <M> =: <v_1,...,v_m>$

      \item Ist $<M>_k = V$, so heißt M \underline{Erzeugendensystem} in V.
    \end{enumerate}

    \section{Satz}

    V K-VR, $M \subseteq V$.

    \begin{enumerate}[label=(\alph*)]
      \item $<M>_k$ ist Unterraum von V
      \item $<M>_k$ ist der kleinste Unterraum von V, der M enthält.

      Insbesondere: Sind $U_1, U_2$ Unterräume von V, so ist  $<U_1 \cup U_2>_k = U_1 + U_2$

      \textbf{Beweis:} Wie 10.7, Mathe II
    \end{enumerate}


    \textbf{Wiederholung}

    \par \medskip

    V K-VR

    $M \subseteq V$, $<M>_k = \{\sum_{i=1}^{n} a_iv_i : n \in \mathbb{N}, a_i \in K, v_i \in M\}$

    Falls $V = <M>_k$, so heißt M Erzeugendensystem von V.

    \section{Definition}

    V K-VR. V heißt \underline{endlich erzeugt}, falls es eine \underline{endliche} Teilmenge $M \subseteq V$ gibt mit V = $<M>_K$

    \section{Beispiel}

    \begin{enumerate}[label=(\alph*)]
      \item $K^n =  \{ \begin{pmatrix}a_1 \\ ... \\ a_n \end{pmatrix}: a_i \in K \}$

      $K^n$ ist endlich erzeugt:

      $e_1, ..., e_n$ \underline{Einheitsvektoren}

      $e_i = \begin{pmatrix}0 \\ .. \\ 1 \\.. \\ 0 \end{pmatrix}$ (1 an i-ter Stelle)

      $K^n = <e_1,...,e_n>_k$, denn $\begin{pmatrix}a_1 \\ ... \\ a_n \end{pmatrix} = a_1e_1+...a_ne_n$

      \item $K[x]$ als K-VR ist nicht endlich erzeugt:

      Angenommen es existieren $f_1,...,f_n \in K[x]$ mit $K[x] = <f_1,...,f_n>_k$.

      Sei t = max Grad($f_i$) $\in \mathbb{N}_0 \cup \{- \infty\}$

      Dann haben alle Polynome in $<f_1, .., f_n>_k$ höchstens Grad t. Also $x^{t+1} \in K[x] \setminus <f_1,...,f_n>_k$. Widerspruch !

      M = $\{1,x,x²,x³,...\} = \{ x^i : i \in \mathbb{N}_0 \}$

      $K[x] = <M>_k$

      $f = \sum_{i=0}^{t} a_ix_i$

      \item $n \in \mathbb{N}$. U = $\{ f \in K[x]: \textnormal{ Grad(f) $\le$ n}\}$

      Unterraum von $K[x]$ endlich erzeugt:

      $U = <x^0, x¹, ..., x^n>_K$
    \end{enumerate}

    \section{Definition}

    Sei V K-VR. $v_1, ..., v_m \in V$ heißen \underline{linear abhängig}, wenn es $a_1, ..., a_n \in K$, \underline{nicht alle = 0}, gibt mit

    $a_1v_1+...+a_mv_m = \sigma$

    (Beachte: Immer ist $0 \cdot v_1 +...+0 \cdot v_m = \sigma$, aber bei lin. Abhängigkeit solles noch eine andere Möglichkeit geben)

    Andernfalls nennt man $v_1, ..., v_m$ \underline{linear unabhängig}

    (D.h. aus $a_1v_1 + ... + a_mv_m = \sigma$, folgt $a_1 = ... = a_m = 0$)

    \par \medskip

    Entspr.: $\{v_1,...,v_m\}$ linear abhängig / linear unabhängig

    $\emptyset$ per Definition linear unabhängig.

    Klar. Teilmenge von linear unabhängigen Vektoren ist wieder linear unabhängig.

    \section{Beispiel}

    \begin{enumerate}[label=(\alph*)]
      \item $\sigma$ ist linear abhängig.

      $1 \cdot \sigma = \sigma$

      \item $v,w \in V, v \neq \sigma \neq w$.

      Wann sind v und w linear abhängig?

      v,w linear abhängig $\Rightarrow$ $\exists a,b \in K$, nicht beide = 0 mit $a \cdot + b \cdot w = \sigma$

      Angenommen: $a \neq 0$: $a \cdot v = -b \cdot w \ | a^{-1}$ (K körper)

      $v = 1v = (a^{-1}a)v = a^{-1}(av) = a^{-1}(-bw) = (-a^{-1}b) w \in <w>_K = \{cw : c \in K\}$

      $d \in K$

      $d v = (-d \cdot a^{-1} \cdot b) w \in <w>_k$.

      $<v>_k \subseteq <w>_k$

      Dann auch $b \neq 0$.

      Angenommen b = 0: $a \cdot v = -0 \cdot w = \sigma$.

      $v = a^{-1} \cdot \sigma = \sigma$ Widerspruch. 

      Vertausche Rollen von $v,w$: $<w>_k \subseteq <v>_k$

      v,w linear abhängig $\Leftrightarrow$ $<v>_k = <w>_k$

      $\Rightarrow$ klar

      $\Leftarrow$ $v \in <v>_k = <w>_k$ $\rightarrow v = c \cdot w$ für ein $c \in K$.                                                    

      $\rightarrow \sigma = -v + c \cdot w = (-1) v + c \cdot w$

      $\rightarrow v,w$ linear abhängig.

      \item $e_1,...,e_n \in K^n$ sind linear unabhängig.

      $\begin{pmatrix} 0 \\ .. \\ 0 \end{pmatrix} = a_1e_1 + ... + a_ne_n  = \begin{pmatrix}a_1 \\ ... \\ a_n \end{pmatrix} \rightarrow a_1 = ... = a_n = 0$

      \item $\begin{pmatrix}1 \\ 2 \\ 3 \end{pmatrix}, \begin{pmatrix}3 \\ 2 \\ 1 \end{pmatrix}, \begin{pmatrix}2 \\ 3 \\ 4 \end{pmatrix} \in \mathbb{R}^3$ lin. abhängig / lin. unabhängig?

      Für welche a,b,c, $\in \mathbb{R}$ gilt:

      $a \begin{pmatrix}1 \\ 2 \\ 3 \end{pmatrix} + b \begin{pmatrix}3 \\ 2 \\ 1 \end{pmatrix} + c \begin{pmatrix} 2 \\ 3 \\4 \end{pmatrix} = \begin{pmatrix}0 \\ 0 \\ 0 \end{pmatrix}$ ?

      Führt auf LGS für die Unbekannten: a,b,c:

      $1 \cdot a + 3 \cdot b + 2 \cdot c = 0$ 

      $2 \cdot a + 2 \cdot b + 3 \cdot c = 0$ 

      $3 \cdot a + 1 \cdot b + 4 \cdot c = 0$

      Gauß:

      $\begin{pmatrix}1 & 3 & 2 & | & 0 \\ 2 & 2 & 3 & | & 0 \\ 3 & 1 & 4 & | & 0 \end{pmatrix} \rightarrow \begin{pmatrix} 1 & 3 & 2 & | & 0 \\ 0 & -4 & -1 & | & 0 \\ 0 & -8 & -2 & | & 0 \end{pmatrix} \rightarrow \begin{pmatrix}1 & 3 & 2 & | & 0 \\ 0 & 1 & \frac{1}{4} & | & 0 \\ 0 & 0 & 0 & | & 0 \end{pmatrix} $

      c frei wählbar, $b = - \frac{1}{4} c, a = -3b - 2c = \frac{3}{4} c - 2c = - \frac{5}{4} c$

      Z.b. c = 4, b = -1, a = -5

      $(-5) \cdot \begin{pmatrix}1 \\ 2 \\ 3 \end{pmatrix} + (-1) \cdot \begin{pmatrix}3 \\ 2 \\ 1 \end{pmatrix} +  4 \cdot \begin{pmatrix}2 \\ 3 \\ 4 \end{pmatrix} = \begin{pmatrix} 0 \\ 0 \\ 0 \end{pmatrix}$

      Vektoren sind linear abhängig.
    \end{enumerate}

    \section{Bemerkung}

    Mann kann auch für unendliche Mengen $M \subseteq V$ lineare Unabhängiekt definieren:

    Jede endliche Teilmenge von M ist linear unabhängig.

    Z.B. $\{x^i : i \in \mathbb{N}_0\}$ linear  unabhängig in $K[x]$

    \section{Satz}

    V K-VR, $v_1,...,v_m \in V$.

    \begin{enumerate}[label=(\alph*)]
      \item $v_1, ..., v_m$ sind linear abhängig $\Leftrightarrow$ $\exists i : v_i = \sum_{j=1}^{n} b_jv_j$ für geeigente $b_j \in K, j \neq i$

      $\Leftrightarrow \exists i : <v_1,...,v_m>_k = <v_1,...,v_{i-1}, v_{i+1}, ..., v_m>_k$

      \item $v_1, ..., v_m$ linear unabhängig $\Leftrightarrow$ jedes $v \in <v_1,...,v_m>_k$ lässt sich auf \underline{eindeutige} Weise als Linearkombination von $v_1,...,v_m$ schreiben.

      \item Sind $v_1,...,v_m$ linear unabhängig und ist 

      \begin{center}
        $v \not \in <v_1,...,v_m>_k$
      \end{center}

      so sind $v_1,...,v_m,v$ linear unabhängig.

      \subsection{Beweis}

      Wie 10.11, Mathe II

      z.B b) $\Rightarrow$: 

      Angenommen $V \in <v_1,...,v_m>_k$

      $V = \sum_{i=1}^{m} a_iv_i = \sum_{i=1}^{m} b_i v_i, a_i,b_i \in K$

      $\sum_{i=1}^{m} (a_i - b_i) v_i = \sum_{i=1}^{m}a_iv_i - \sum_{i=1}^{m} b_i v_i = \sigma$

      $v_1,...,v_m$ linear unabhängig $\Rightarrow$ $a_i - b_i = 0$ für $i = 1,...,n \rightarrow a_i = b_i$
    \end{enumerate}

    \section{Definition}

    Sei V endlich erzeugter K-VR. Eine endliche Teilmenge $B \subseteq V$ heißt \underline{Basis} von V, falls

    \begin{enumerate}
      \item[(1)] V = $<B>_k$
      \item[(2)] B linear unabhängig
    \end{enumerate}

    (V = $\{\sigma\}$ : $\emptyset$ ist Basis von V)

    \section{Beispiel}

    \begin{enumerate}[label=(\alph*)]
      \item $e_1, ..., e_n$ Basis von $K^n$ (\underline{kanonische} Basis)

      $\begin{pmatrix}a_1 \\ ... \\ a_n \end{pmatrix}$ = $a_1e_1+...+a_ne_n$

      \item $\begin{pmatrix}1 \\ 2 \end{pmatrix}, \begin{pmatrix}3 \\ 1 \end{pmatrix}, K = \mathbb{Z}_5$

      In $\mathbb{Z}_5^2$ sind $\begin{pmatrix}1 \\ 2 \end{pmatrix}, \begin{pmatrix}3 \\ 1 \end{pmatrix}$ keine Basis, denn sie sind nicht linear unabhängig über $\mathbb{Z}_5$.

      $3 \cdot \begin{pmatrix}1 \\ 2 \end{pmatrix} = \begin{pmatrix}3 \\ 1 \end{pmatrix}$

      $\begin{pmatrix}0 \\ 0 \end{pmatrix} = 3 \cdot \begin{pmatrix}1 \\ 2 \end{pmatrix} + 4 \cdot \begin{pmatrix}3 \\ 1 \end{pmatrix}$

      \par \medskip

      $K = \mathbb{Z}_7$

      $\begin{pmatrix}1 \\ 2 \end{pmatrix}, \begin{pmatrix}3 \\ 1 \end{pmatrix}$ bilden Basis von $\mathbb{Z}_7^2$:

      Lineare Unabhängigkeit:

      $a \cdot \begin{pmatrix}1 \\ 2 \end{pmatrix} + b \cdot \begin{pmatrix}3 \\ 1 \end{pmatrix} = \begin{pmatrix}0 \\ 0 \end{pmatrix}$, $a,b \in \mathbb{Z}_7$.

      Führt auf LGS für a,b: 

      $1 \cdot a + 3 \cdot b = 0$ 

      $2 \cdot a + 1 \cdot b = 0$

      Gauß-Algorithmus (funktioniert über jedem Körper K):

      $\begin{pmatrix} 1 & 3 & 0 \\ 2 & 1 & 0 \end{pmatrix} \rightarrow \begin{pmatrix}1 & 3 & 0 \\ 0 & 2 & 0 \end{pmatrix} \underbrace{\rightarrow}_{\textnormal{multp. mit Inversem}} \begin{pmatrix}1 & 3 & 0 \\ 0 & 1 & 0 \end{pmatrix}$

      $b = 0, a +3b = 0 \rightarrow a = 0$

      $\left \langle \begin{pmatrix}1 \\ 2 \end{pmatrix}, \begin{pmatrix}3 \\ 1 \end{pmatrix} \right \rangle_{\mathbb{Z}_7} = \mathbb{Z}_7^2$

      Sei $\begin{pmatrix}c \\ d \end{pmatrix} \in \mathbb{Z}_7^2$. Gesucht sind $a,b \in \mathbb{Z}_7^2$

      $a \cdot \begin{pmatrix}1 \\ 2 \end{pmatrix} + b \cdot \begin{pmatrix}3 \\ 1 \end{pmatrix} = \begin{pmatrix}c \\ d \end{pmatrix}$

      $1 \cdot a + 3 \cdot b = c$ 

      $2 \cdot a + 1 \cdot b = d$

      Gauß:

      $\begin{pmatrix}1 & 3 & c \\ 2 & 1 & d \end{pmatrix} \rightarrow \begin{pmatrix}1 & 3 & c \\ 0 & 2 & d-2c \end{pmatrix} \underbrace{\rightarrow}_{\textnormal{2. Zeile mit $2^{-1}$ = 4}} \begin{pmatrix}1 & 3 & c \\ 0 & 1 & 4d-c \end{pmatrix}$

      $b = 4d-c = 4d + 6c$

      $a = c-3b = c-5d-4c = 4c+2d$

      $\begin{pmatrix}c \\ d\end{pmatrix} = (4c+2d) \cdot \begin{pmatrix}1 \\ 2 \end{pmatrix} + (4d +6c) \begin{pmatrix}3 \\ 1 \end{pmatrix}$
    \end{enumerate}

    \section{Satz (Existenz von Basen)}

    Sei V endlich erzeugter K-VR. Dann enhält jedes endliche Erzeugendensystem von V eine Basis v von V.

    \subsection{Beweis}

    Sei $M \subseteq V$ endlich mit $V = <M>_K$. Ist M linear unabhängig so ist M basis. 

    Ist M lin. abhängig, so existiert nach 2.15.a) $v \in M$ mit $V = <M>_k = <M \setminus{\{ v \}}>_k$.

    Da M endlich ist endet dieses Verfahren mit Basis.

    \section{Lemma}

    V endlich erzeugter K-VR V.

    $B = \{v_1,...,v_n\}$ Basis von V.

    Sei $\sigma \neq w \in V$. Dann:

    \begin{equation}
      w = \sum_{j=1}^n a_kv_j, a_j \in K
    \end{equation}

    Ist $a_i \neq 0$, so ist 

    \begin{center}
      $(B \setminus \{v_i\}) \cup \{w\}$
    \end{center}

    wieder eine Basis von V.

    \subsection{Beweis}

    $w = \sum_{j=1}^n a_jv_j \Rightarrow a_iv_i = w - \sum_{j=1, j \neq i}^{n} a_jv_j$

    $\Rightarrow v_i = a_i^{-1}(a_iv_i) = a_i^{-1}w + \sum_{j=1, j \neq i}^n (a_i^{-1} a_j) v_j$

    $v_i \in \left \langle (B \setminus \{v_i\}) \cup \{w\} \right \rangle$

    V = $\left \langle B \right \rangle_K = \left \langle B \cup \{w\} \right \rangle_K = \left \langle (B \setminus \{v_i\} \cup \{w\} \right \rangle_K$

    Zeige: $(B \setminus \{v_i\}) \cup \{w\}$ ist linear unabhängig:

    \par \medskip

    \newpage
    Angenommen:

    \begin{center}
       $\sigma = \sum_{j=1, j \neq i}^n c_jv_j + cw$ mit $c_j,c \in K$
    \end{center}

    Es folgt:

    \begin{center}
      $\sum_{j=1, j \neq i}^n c_jv_j + \sum_{j=1}^n ca_jv_j$

      $= \sum_{j=1, j \neq i}^n (c_j +ca_j)v_j + ca_iv_i$
    \end{center}

    $v_1, ..., v_n$ linear unabhängig:

    $\rightarrow$ $ca_i = 0$ und $c_j+c_aj = 0 $ für alle $j \neq i$.

    \begin{enumerate}
      \item[(1)]$ca_i = 0, \ a_i \neq 0$ $\rightarrow c = 0$
      \item[(2)]$c_j = 0$ für alle $j \neq i$
    \end{enumerate}

    \section{Satz (Austauschsatz von Steinitz)}

    V endlich erzeugter K- VR, B Basis von V, M endliche linear unabhängige Teilmenge von V. Dann existiert $C \subseteq B$ mit $|C| = |M|$, so dass                                                             

    \begin{center}
      $(B \setminus C) \cup M$
    \end{center}

    Basis von V ist.

    Insbesondere $|M| \le |B|$.

    \subsection{Beweis}

    Sei $|M| = k$. Induktion nach k.

    $k = 0$: klar

    $k > 0$: Sei $M = \tilde{M} \cup \{w\}$, $|\tilde{M}| = k-1$.

    Induktionsvorraussetzung: Existiert $\tilde{C} \subseteq B$ mit 

    \begin{center}
      $|\tilde{C}| =|\tilde{M}|$
    \end{center}

    und

    \begin{center}
      $(B \setminus \tilde{C}) \cup \tilde{M}$ 
    \end{center}

    ist Basis von V.

    w = $\sum_{u \in B \setminus \tilde{C}} a_uu+ \sum_{v \in \tilde{M}}a_v v$

    Mindestens eines der $a_k$ ist $\neq 0$, denn sonst:

    \begin{center}
      $w = \sum_{v \in \tilde{M}}a_vv$, also
    \end{center}

    $M = \tilde{M} \cup \{w\}$ linear abhängig (Widerspruch!)

    Also sei $a_n \neq 0$ für ein $u \in B \setminus \tilde{C}$

    Nach 2.19 ist 

    \begin{center}
      $(B \setminus C) \cup M$
    \end{center}

    Basis von V, wobei $C = \tilde{C} \cup \{u\}$. Fertig.

    \section{Korollar}

    V endlich erzeugter K-VR. 

    \begin{enumerate}[label=(\alph*)]
      \item Je zwei Basen von V enthalten gleich viele Vektoren
      \item Jede linear unabhängige Teilmenge von V ist endlich
      \item (Basisergänzungssatz)

      Jede linear unabhängige Menge von Vektoren lässt sich zu einer Basis ergänzen.
    \end{enumerate}

    \subsection{Beweis}

    \begin{enumerate}[label=(\alph*)]
      \item $B, \tilde{B}$ Basen von V.

      2.20 : $|B| \le |\tilde{B}|$, $|\tilde{B}| \le |B|$, also $|B| = |\tilde{B}|$.

      \item Angenommen V enthält unendlich linear unabhängige Teilmengen.

      Sei B Basis von V. Wähle $M_0 \subset M$ mit $M_0$ endlich, $|M_0| > |B|$.

      Nach Voraussetzung ist $M_0$ linear unabhängig. Widerspruch zu 2.20.

      \item Sei M linear unabhängige Teilmenge von V. Nach b) ist M endlich.

      Sei B eine Basis von V.

      2.20: $\exists C \subseteq B, |C| = |M|$, so dass $(B \setminus C) \cup M$ Basis.
    \end{enumerate}

    \section{Satz}

    V endlich erzeugter K-VR, $B \subseteq V$. Dann sind äquivalent: 

    \begin{enumerate}
      \item[(1)] B ist Basis von V
      \item[(2)] B ist maximal linear unabhängige Teilmenge von V (d.h. $B \cup \{v\}$ ist linear abhängig f.a $v \in V \setminus B$)
      \item[(3)] B ist minimales Erzeugendensystem von V (d.h. $\left \langle B \setminus \{w\} \right \rangle_K \neq V$ für alle $w \in B$)
    \end{enumerate}

    \subsection{Beweis}

    \subsubsection*{(2) $\Rightarrow$ (1)}

    Angenommen $\left \langle B \right \rangle_K \neq V$. Sei $v \in V \setminus \left \langle B \right \rangle_K$.

    2.15.c): $B \cup \{v\}$ linear unabhängig. Widerspruch !

    $\left \langle B \right \rangle_K = V.$ B ist Basis

    \subsubsection*{(1) $\Rightarrow$ (2)}

    Angenommen: $B \subseteq C$, C linear unabhängig.

    2.21: C endlich

    2.20: $|C| \le |B|$. Daher: B = C

    \subsubsection*{(3) $\Rightarrow$ (1)} 

    Angenommen B ist linear abhängig.

    2.15.a): $\exists w \in B$:

    V = $\left \langle B \right \rangle_K = \left \langle B \setminus \{w\} \right \rangle$ Widerspruch!

    B ist linear unabhängig, also Basis.

    \subsubsection*{(1) $\Rightarrow$ (3)}

    Angenommen: $\exists w \in B$ mit $\left \langle B \setminus \{w\} \right \rangle_K = V$ = $\left \langle B \right \rangle_K$

    2.15.a): B ist linear abhängig. Widerspruch!

    \section{Definition}

    V K-VR.

    \begin{enumerate}[label=(\alph*)]
      \item Ist V endlich erzeugt, B ist Basis von V, $|B| = n$, so hat V \underline{Dimension n}, $dim_K(V) = n$ (oder einfach $dim(V) = n$)

      (V heißt endlich-dimensional)

      \item Ist V nicht endlich erzeugt, so heißt V \underline{unendlich-dimensional}.
    \end{enumerate}

    (Also: endlich erzeugt = endlich-dimensional)


    \section{Korollar}

    V K-VR, $dim_K(V) = n$, $B \subseteq V$, $|B| = n$.

    \begin{enumerate}[label=(\alph*)]
      \item Ist B linear unabhängig, dann ist B Basis.

      \item Ist $\left \langle B \right \rangle_K = V$, dann ist B Basis.
    \end{enumerate}

    \subsection{Beweis}

    Folgt aus 2.22

    \section{Beispiel}

    \begin{enumerate}[label=(\alph*)]
      \item $dim_K(K^n) = n$, da $e_1,...,e_n$ Basis.

      \item $V = \mathbb{R}^4$

      U = $\left \langle \begin{pmatrix}1 \\ 2 \\ 0 \\1 \end{pmatrix}, \begin{pmatrix}0 \\ 2 \\ 1 \\ 0 \end{pmatrix} \right \rangle_{\mathbb{R}}$

      $u_1, u_2$ sind linear unabhängig.

      $a \cdot \begin{pmatrix}1 \\ 2 \\ 0 \\ 1 \end{pmatrix} + b \cdot \begin{pmatrix}0 \\ 2 \\ 1 \\ 0 \end{pmatrix} = \begin{pmatrix}0 \\ 0 \\ 0 \\ 0 \end{pmatrix}$

      $\{u_1, u_2\} $ Basis von U, $dim_{\mathbb{R}}(U) = 2$

      Ergänze $u_1, u_2$ zu Basis von $V = \mathbb{R}^4$:

      \subsubsection{1. Möglichkeit}

      $e_1, e_2, e_3, e_4$ kanonische Basis des $\mathbb{R}^4$

      $u_1 = 1 \cdot e_1 + 2 \cdot e_2 + 0 \cdot e_3 + 1 \cdot e_4$

      \par \medskip

      2.19: $u_1, e_2, e_3, e_4$ Basis von $\mathbb{R}^4$.

      \par \medskip

      $u_2 = au_1+be_2+ce_3+de_4$

      $\rightarrow \begin{pmatrix}0 \\ 2 \\1 \\ 0 \end{pmatrix} = a \cdot \begin{pmatrix}1 \\ 2 \\ 0 \\ 1 \end{pmatrix} + \begin{pmatrix} 0 \\ b \\ 0 \\ 0 \end{pmatrix} + \begin{pmatrix}0 \\ 0 \\ 0 \\ 0 \end{pmatrix} + \begin{pmatrix}0 \\ 0 \\ 0 \\ d \end{pmatrix}$ $\rightarrow c = 1$

      \par \medskip

      2.19: $u_1,u_2,e_3, e_4$ Basis von $\mathbb{R}^4$
    \end{enumerate}

    \subsection{2. Möglichkeit}

    2.15.c): 

    $v_1, ..., v_m$ linear unabhängig.

    $v \not \in \left \langle v_1, ..., v_m \right \rangle \rightarrow v_1, ..., v_m, v $ linear unabhängig.

    \par \medskip

    U = $\left \lbrace \begin{pmatrix}a \\ 2a + 2b \\ b \\ a\end{pmatrix} : a,b \in \mathbb{R} \right \rbrace$

    \par \medskip

    $e_1 \not \in U$ (1. Koord $\neq$ 4. Koordinate)

    2.15.c) $u_1, u_2, e_1 $ linear unabhängig.


    $\left \langle u_1, u_2, e_1 \right \rangle = ?$

    \par \medskip

    $u_1 := \left \lbrace \begin{pmatrix}a+c \\ 2a+2b \\ b \\ a \end{pmatrix} : a,b,c \in \mathbb{R} \right \rbrace$

    \par \medskip

    $e_2 \not \in U$

    2.15.c) $u_1, u_2, e_1, e_2$ linear unabhängig.

    2.24: $\{u_1, u_2,e_1, e_2\}$ Basis von $\mathbb{R}^4$

    \section{Satz}

    V K-VR, $dim_K(V) = n$.

    \begin{enumerate}[label=(\alph*)]
      \item Ist U Unterraum von V, so ist $dim_K(U) \le n$.

      Ist $dim_K(U) = n$, so ist U = V.

      \item (Dimensionsformel)

      U, W Unterräume von V, so gilt:

      $dim(U+W) = dim(U) + dim(W) - dim(U \cap W)$

      A, B endl. Mengen:

      $| A \cup B| = |A| + |B| - |A \cup B|$
    \end{enumerate}

    \subsection{Beweis}

    \begin{enumerate}[label=(\alph*)]
      \item ergänze Basis von U zu Basis von V. (2.21.c))

      \item Basis von $U \cap W \rightarrow$ (ergänze zu) Basis von U

      (WHK 9.23)
    \end{enumerate}

    \section{Definition}

    V K-VR, $dim_K(V) = n$.

    B = $(v_1, ..., v_n)$ \underline{geordnete} Basis von V.

    \par \medskip

    Jedes $v \in V$ hat \underline{eindeutige} darstellung $v = \sum_{i=1}^{n} a_iv_i$

    $a_i \in K$. (2.15.b))

    ($a_1, ..., a_n$) (in dieser Anordnung) heißen \underline{Koordinaten} von V bezüglich B.

    \par \medskip

    Insbesondere $v_i$ hat Koordinaten $(0,..,0,1,0,...,0)$ (1 an i-ter Stelle)

    \section{Beispiel}

    \begin{enumerate}[label=(\alph*)]
      \item V = $K^n$, ($e_1, ..., e_n$) = B kanonische Basisv 

      Koordinaten von $v = \begin{pmatrix}a_1 \\ ... \\ a_n \end{pmatrix}$ bezüglich B: $(a_1, ..., a_n)$ \underline{Kartesische Koordinaten}

      \item $V = \mathbb{Q}^3$, B = ($\begin{pmatrix}1 \\ 2 \\ 0 \end{pmatrix}, \begin{pmatrix}1 \\ 0 \\ 1 \end{pmatrix}, \begin{pmatrix}0 \\ 1 \\ 2 \end{pmatrix}$)

      B ist geordnete Basis von V (nachprüfen).

      \par \medskip

      Koordinaten von $\begin{pmatrix}1 \\ 0 \\ 0 \end{pmatrix}$ bezüglich B: 

      $\begin{pmatrix}1 \\ 0 \\ 0 \end{pmatrix} = a_1 \cdot \begin{pmatrix}1 \\ 2 \\ 0 \end{pmatrix} + a_2 \begin{pmatrix}1 \\ 0 \\ 1 \end{pmatrix} + a_3 \cdot \begin{pmatrix}0 \\ 1 \\ 2 \end{pmatrix}$

      Gauß-Algorithmus:

      $\begin{pmatrix} 1 & 1 & 0 & 1 \\ 2 & 0 & 1 & 0 \\ 0 & 1 & 2 & 0 \end{pmatrix} \rightarrow \begin{pmatrix}1 & 1 & 0 & 1 \\ 0 & -2 & 1 & -2 \\ 0 & 1 & 2 & 0 \end{pmatrix} \rightarrow \begin{pmatrix}1 & 1 & 0 & 1 \\ 0 & 1 & - \frac{1}{2} & 1 \\ 0 & 1 & 2 & 0 \end{pmatrix} \rightarrow \begin{pmatrix}1 & 1 & 0 & 1 \\ 0 & 1 & -\frac{1}{2} & 1 \\ 0 & 0 & 1 & - \frac{2}{5} \end{pmatrix}$

      Damit folgt:

      $a_3 = - \frac{2}{5}$

      $a_2 = 1 + \frac{1}{2} a_3 = 1 - \frac{1}{5} = \frac{4}{5}$

      $a_1 = 1 - a_2 = \frac{1}{5}$

      Koordinaten von $\begin{pmatrix}1 \\ 0 \\ 0 \end{pmatrix}$ bezüglich B: ($\frac{1}{5}, \frac{4}{5} , - \frac{2}{5}$)
    \end{enumerate}

    $\mathbb{R}^2:$ 1-dim. Unterräume (Geraden durch $\sigma$), 0-dim: $\{\sigma\}$

    0-dim. affiner Unterräume $\{v\}, v \in V$

    \section{Definition}

    V K-VR, U Unterraum von V, $w \in V$

    Dann heißst $w + U := \{w +u: u\in U\}$ \underline{affiner Unterraum} von V.

    (w + U ist im Allgemienen kein Untervektorraum)

    \par \medskip

    $dim(w+U) := dim(U)$

    \section{Satz}

    V K-VR, U,W Unterräume von V, $w \in V, v_1, v_2 \in V$

    \begin{enumerate}[label=(\alph*)]
      \item w+U ist Unterraum (1) $\Leftrightarrow$ $w \in U  (2) \Leftrightarrow$ w + U = U (3)

      \item Ist $v \in w+U$, so ist $v + U = w + U$

      \item Sind $v_1 + U, v_2 + W$ affine Unterräume, so ist entweder ($v_1 + U) \cap (v_2 + W) = \emptyset$ oder es existiert $v \in V$ mit $(v_1+U) \cap (v_2 + W) = v + (U \cap W)$ (affiner Unterraum)
    \end{enumerate}

    \subsection{Beweis}

    \begin{enumerate}[label=(\alph*)]
      \item  \textbf{(1) $\rightarrow$ (2)}

      w+U Unterraum $\Rightarrow \sigma \in w + U$

      $\Rightarrow \exists u \in U$ mit $w+u = \sigma$ 

      $\Rightarrow w = -u \in U$

      \textbf{(2) $\rightarrow$ (3)}

      $w \in U, w+U \subseteq U$ (da U Unterraum)

      Sei $u \in U$. Dann $u-w \in U$, $u = w + (u-w) \in w + U$

      w + u = U

      \textbf{(3) $\rightarrow$ (1)}

      $\checkmark$

      \item $v \in w+U, v = w+u$ für ein $u \in U$.

      v + U = w + $\underbrace{u + U}_{=U}$ = w+U

      \item Angenommen: $(v_1 + U) \cap (v_2 + W) \neq \emptyset$

      Sei $v \in (v_1+U) \cap (v_2+U)$

      \par \medskip

      Nach b): 

      \begin{itemize}
        \item $v + U = v_1 + U$         
        \item $v + W = v_2 + W$
      \end{itemize}

      $(v_1 + U) \cap (v_2 + W) = (v+ U) \cap (v + W) = v+ (U \cap W)$ 
    \end{enumerate}

    \section{Bemerkung}
    affine Unterräume: 

    Spezielle Rolle von $\sigma$ ist aufgehoben.

    \par \medskip

    Zur Beschreibung eines $x \in K^n$ kann man jeden Punkt p als ``Nullpunkt'' wählen und dann die Koordinaten von x bezüglich einer nach p ``verschobenen'' Basis berechnen.

    \par \medskip

    p hat Koordinaten ($p_1, ..., p_n$) bezüglich Basis $v_1,...,v_n$.

    (1) Ursprüngliches Koordinatensystem: $\sigma$, $v_1,...,v_n$.

    (2) Neues Koordinatensystem: p, $v_1+p,...,v_n+p$.

    \par \medskip

    x hat Koordinaten ($a_1, ..., a_n$) bezüglich (1) $\rightarrow$ Koordinaten von x bezüglich (2) = ($a_1+p, ..., a_n+p$)

    \par \medskip
    x hat Koordinaten ($a'_1,...,a'_n$) bezüglich (2) $\rightarrow$ x hat Koordinaten ($a'_1+p,...,a'_n+p$) bezüglich (1)

    (Robotik)

    \section{Bemerkung}

    \begin{enumerate}[label=(\alph*)]
      \item In Mathe II:

      $n \times m$-Matrizen über $\mathbb{Q,R,C}$

      Das geht auch über beliebigen Körpern K.

      Addition, Multiplikation mit Skalaren, Matrizenmultiplikation werden analog definiert. Es gelten die gleichen Rechenregeln wie in Mathe II, 9.5

      \item In Mathe II wurden Matrizen verwendet zur Beschreibung von LGS.

      Analog: LGS über beliebigen Körpern K. Gauß-Algorithmus funktioniert analog.

      $(a_1,...,a_m), a_1 \neq 0$

      $\rightarrow$ ($1, a_2 \cdot a_1^{-1}, ...$) (Mutliplikation der Zeile mit $a_1^{-1}$) (K Körper!)
    \end{enumerate}

    \section{Satz}

    \begin{enumerate}[label=(\alph*)]
      \item Die Menge der Lösungen eines \underline{homogenen} LGS

      \begin{center}
        $A \cdot x = 0$
      \end{center}

      $(A \in M_{n,m}(K)$, $x \in K^m$, 0 ist Nullvektor in $K^n$)

      bildet Untervektorraum  von $K^m$.

      \item Ist das \underline{inhomogene} LGS

      \begin{center}
        $A \cdot x = b$
      \end{center}

      ($A,x$ wie oben, $b \in K^n$)

      lösbar und ist $x_0 \in K^m$ eine spezielle Lösung(d.h. $A \cdot x_0 = b$), so erhält man alle Lösungen von $A \cdot x = b$ durch

      \begin{center}
        $\{x_0 + y : Ay = 0 \}$
      \end{center}

      Ist U der Lösungsraum von $A \cdot x = 0$, so ist die Lösungsmenge von $A \cdot x = b$ gerade der affine Unterraum $x_0 + U$ von $K^m$
    \end{enumerate}

    \subsection{Beweis}

    \begin{enumerate}[label=(\alph*)] 
      \item Folgt aus Rechenregeln für Matrizen

      $x_1, x_2 \in K^m$ Lösungen von 

      \begin{center}
        $A \cdot x = 0$
      \end{center}

      $A \cdot (x_1+x_2) = Ax_1 + Ax_2 = 0 + 0 = 0$ 

      $\rightarrow x_1 + x_2$ auch Lösung.

      $a \in K$.

      $A \cdot (a \cdot x_1) = a \cdot (Ax_1) = a \cdot 0 = 0 $

      $\rightarrow a \cdot x_1$ Lösung.

      Null-Lösung existiert immer.

      \item $A \cdot x_0 = b$. Sei $y \in K^m$ mit $Ay = 0$.

      $A \cdot (x_0 + y) = Ax_0 + Ay = b + 0 = b$

      $\rightarrow x_0 + y$ ist Lösung von $Ax = b$.

      \par \medskip

      Zeige: Jede Lösung von Ax = b ist von der Form $x_0 + y$ für ein $y$ mit $Ay = 0$.

      Sei x Lösung von $Ax = b$.

      $x = x_0 + (x-x_0)$

      $A \cdot (x - x_0) = Ax - Ax_0 = b - b = 0 \ \checkmark$
    \end{enumerate}

    \section{Beispiel}

    Gegeben LGS:

    \begin{center}

      $x_1 + x_2 + x_3 - x_4 = 0$

      $x_1 - 2x_2 + 0 x_3 - x_4 = 1$ 

    \end{center}

    über $\mathbb{Q}$.

    $\begin{pmatrix}1 & 1 & 1 & -1 \\ 1 & -2 & 0 & 1 \end{pmatrix} \cdot \begin{pmatrix}x_1 \\ ... \\ x_4 \end{pmatrix} = \begin{pmatrix}0 \\ 1 \end{pmatrix}$

    Gauß:


    \par \medskip

    $\begin{pmatrix}1 & 1 & 1 & -1 & 0 \\ 1 & -2 & 0 & 1 & 1 \end{pmatrix} \rightarrow \begin{pmatrix}1 & 1 & 1 & -1 & 0 \\ 0 & -3 & -1 & 2 & 1 \end{pmatrix} \rightarrow \begin{pmatrix}1 & 1 & 1 & -1 & 0 \\ 0 & 1 & \frac{1}{3} & - \frac{2}{3} & - \frac{1}{3} \end{pmatrix}$

    \par \medskip

    $x_3, x_4$ frei wählbar.

    $x_2 = - \frac{1}{3} - \frac{1}{3} x_3 + \frac{2}{3} x_4$, $x_1 = ...$

    \par \medskip

    \textbf{Zugehöriges homogenes System:}

    \par \medskip

    $\begin{pmatrix}1 & 1 & 1 & -1 & 0 \\ 1 & -2 & 0 & 1 & 0 \end{pmatrix} \rightarrow \begin{pmatrix}1 & 1 & 1 & -1 & 0 \\ 0 & 1 & \frac{1}{3} & - \frac{2}{3} & 0 \end{pmatrix}$

    \par \medskip

    Lösungsmenge = Unterraum.

    Basis des Lösungsraums:

    Setze die frei wählbaren $x_4, x_3:$

    \begin{itemize}
      \item $x_4 = 1, x_3 = 0$ $\rightarrow$ Lösung $\begin{pmatrix}\frac{1}{3} \\ \frac{2}{3} \\ 0 \\ 1 \end{pmatrix}$
      \item $x_4 = 0, x_3 = 1$ $\rightarrow$ Lösung $\begin{pmatrix}- \frac{2}{3} \\ - \frac{1}{3} \\ 1 \\ 0 \end{pmatrix}$
    \end{itemize}

    Jede Lösung $d \cdot \begin{pmatrix} \frac{1}{3} \\ \frac{2}{3} \\ 0 \\ 1 \end{pmatrix} + c \cdot \begin{pmatrix}- \frac{2}{3} \\ - \frac{1}{3} \\ 1 \\ 0 \end{pmatrix} = \begin{pmatrix}* \\ * \\ c \\ d \end{pmatrix}$

    \par \medskip

    Lösungsraum von zugehörigen homogenen LGS: 

    $\left \langle \begin{pmatrix} \frac{1}{3} \\ \frac{2}{3} \\ 0 \\ 1 \end{pmatrix}, \begin{pmatrix}- \frac{2}{3} \\ - \frac{1}{3} \\ 1 \\ 0 \end{pmatrix} \right \rangle = \left \langle \begin{pmatrix}1 \\ 2 \\ 0 \\ 3 \end{pmatrix}, \begin{pmatrix}2 \\ 1 \\ -3 \\ 0 \end{pmatrix} \right \rangle$

    \par \medskip

    affiner Lösungsraum des inhomogenen LGS:

    Spezielle Lösung $x_4 = x_4 = 0$, $x_2 = - \frac{1}{3}$, $x_1 = \frac{1}{3}$.

    $\begin{pmatrix}\frac{1}{3} \\ - \frac{1}{3} \\ 0 \\ 0 \end{pmatrix} +  \left \langle \begin{pmatrix}1 \\ 2 \\ 0 \\ 3 \end{pmatrix}, \begin{pmatrix}2 \\ 1 \\ -3 \\ 0 \end{pmatrix} \right \rangle$


    \chapter{Lineare Abbildungen}

    \section{Definition}

    V, W K-VR

    \begin{enumerate}[label=(\alph*)]
      \item $\alpha: V \rightarrow W$ heißt \underline{(K-)lineare Abbildung} (oder \underline{Vektorraum-Homomorphismus}), falls:

        \begin{enumerate}
          \item[(1)]$\alpha$(u+v) = $\alpha(u)$ + $\alpha(v)$, für alle $u,v \in V$. (Additivität)
          \item[(2)]$\alpha(kv)$ = $k \cdot \alpha(v)$, für alle $k \in K, v \in V$. (Homogenität)
        \end{enumerate}
    \end{enumerate}

    \section{Bemerkung}

    $\alpha: V \rightarrow W$ lineare Abbildung.

    \begin{enumerate}[label=(\alph*)]
      \item $\alpha(\sigma) = \sigma$
      \item $\alpha(\sum_{i=1}^n k_i v_i)$ = $\sum_{i=1}^n k_i \alpha(v_i)$
    \end{enumerate}

    \subsection{Beweis}

    \begin{enumerate}[label=(\alph*)]
      \item $\alpha(\sigma) = \alpha(\sigma + \sigma) = \alpha(\sigma) + \alpha(\sigma) \rightarrow \alpha(\sigma) = 0$
      \item Definition + Induktion nach n.
    \end{enumerate}

    \section{Beispiel}
    \begin{enumerate}[label=(\alph*)]
      \item Nullabbildung $\alpha: V \rightarrow W$

      $\alpha(v) = \sigma$ für alle $v \in V$

      \item $c \in K$.

      $\alpha: V \rightarrow V, \alpha(v) = c \cdot v$

      lineare Abbildung.

      $c = 1:  \alpha = id_v$

      \item $\varphi : \begin{cases}\mathbb{R}^3 \rightarrow \mathbb{R}^3 \\ \begin{pmatrix}x_1 \\ x_2 \\ x_3 \end{pmatrix} \mapsto \begin{pmatrix}x_1 \\ x_2 \\ - x_3 \end{pmatrix} \end{cases}$ linear.

      Spiegelung an der $\{x_1, x_2\}$-Ebene in $\mathbb{R}^3$

      \item $\alpha: \begin{cases}\mathbb{R}^2 \rightarrow \mathbb{R}^1 \\ \begin{pmatrix}x_1 \\ x_2 \end{pmatrix} \mapsto x_1^2 \end{cases}$ nicht linear.

      $\alpha(\begin{pmatrix}x_1 \\ x_2 \end{pmatrix} + \begin{pmatrix}y_1 + y_2 \end{pmatrix}) = \alpha(\begin{pmatrix}x_1 + y_1 \\ x_2 + y_2 \end{pmatrix}) = (x_1+y_1)^2 = x_1² + 2x_1y_1 + y_1² \neq x_1² + y_1^2$
      \end{enumerate}

      \section{Satz}

      Sei $A \in M_{m,n}(k)$

      Definiere $\alpha: K^n \rightarrow K^m$ (Spaltenvektoren) durch $\alpha(x) = \underbrace{A}_{m \times n} \cdot \underbrace{x}_{n \times 1} \in K^m$. Dann ist $\alpha$ lineare Abbildung.

      \subsection{Beweis}

      Folgt aus Rechenregeln für Matrizenmultiplikation: 

      \begin{itemize}
        \item $\alpha(x+y) = A \cdot (x+y) = Ax + Ay = \alpha(x) + \alpha(y)$ $\checkmark$
        \item $\alpha(k \cdot x) = A (k \cdot x) = k \cdot (Ax) = k \alpha(x)$ $\checkmark$
      \end{itemize}

      \underline{Beispiel aus 3.3 a) - c)}

      \begin{itemize} 
        \item $V = K^n$, Nullabbildung $K^n \rightarrow K^m$

        Von der Form in 3.4 mit $A = \begin{pmatrix}0 & ... & 0 \\ ... & ... &... \\ 0 & ... & 0 \end{pmatrix}$ Nullmatrix

        \item $\alpha: \begin{cases}K^n \rightarrow K^n \\ x \mapsto c \cdot x \end{cases}, (c \in K)$

        3.4 mit $A = \begin{pmatrix}c & .. & 0 \\ ... & ... & .. \\ 0 & ... & c \end{pmatrix}$

        \item Spiegelung aus 3.3.c)

        3.4 mit $A = \begin{pmatrix}1 & & 0 \\ 0 & 1 & 0 \\ 0 & 0 & -1 \end{pmatrix}$
      \end{itemize}

      \underline{Später:} Alle linearen Abbildungen $K^n \rightarrow K^m$ sind von der Form 3.4

      \section{Satz}

      U, V, W K-VR:

      \begin{enumerate}[label=(\alph*)]
        \item $\alpha, \beta: V \rightarrow W$ linear, so \underline{auch $\alpha + \beta$} (definiert durch $(\alpha + \beta) (v) := \alpha(v) + \beta(v)$ f.a. $v \in V$), und \underline{$k \cdot \alpha$} (definiert durch ($k \cdot \alpha) (v) := k \cdot \alpha(v)$ f.a. $v \in V$) linear von V nach W.

        \item $\alpha: V \rightarrow W, \gamma : W \rightarrow U$ linear, so auch $\gamma \circ \alpha : V \rightarrow U$ lineare Abbildung.
      \end{enumerate}

      \subsection{Beweis}

      Blatt 5.

      \section{Satz}

      $\alpha: V \rightarrow W$ lineare Abbildung.

      \begin{enumerate}[label=(\alph*)]
        \item Ist U Unterraum von V, so ist $\alpha(U) := \{ \alpha(u) : u \in U\}$ Unterraum von W.

        Insbesondere ist $\alpha(V)$, \underline{Bild} von $\alpha$, Unterraum von W.

        \item Ist U endlich-dimensional, so auch $\alpha(U)$ und es gilt $dim(\alpha(U)) \le dim(U)$ 
      \end{enumerate}

      \subsection{Beweis}

      \begin{enumerate}[label=(\alph*)]
        \item $\alpha(u_1), \alpha(u_2) \in \alpha(U)$, d.h. $u_1, u_2 \in U$, so $\alpha(u_1) + \alpha(u_2) = \alpha(\underbrace{u_1+u_2}_{\in U}) \in \alpha(U)$

        $k \in K$

        $k \cdot \alpha(u_1) = \alpha(k \cdot u_1) \in \alpha(U)$

        \item Sei $u_1, ..., u_k$ Basis von U.

        $u \in U, u = \sum_{i=1}^{k} c_i u_i, c_i \in K$

        $\alpha(u) = \sum_{i=1}^{k} c_i \alpha(u_i)$

        Also $\alpha(u) = \left \langle \alpha(u_1), ..., \alpha(u_k) \right \rangle_K$

        Nach 2.13: 

        $\{\alpha(u_1), ..., \alpha(u_k)\}$ enthält Basis von $\alpha(U)$.

        $dim(\alpha(U)) \le k = dim(U)$
      \end{enumerate}

      \section{Definition}

      V,W K-VR, V endlich dimensional, $\alpha : V \rightarrow W$ lineare Abbildung. 

      \par \medskip

      Dann $dim(\alpha(V)) =: rg(\alpha)$ (Rang von $\alpha$)

      \section{Satz}

      V, W K-VR, $\alpha: V \rightarrow W$ lineare Abbildung.

      \begin{enumerate}[label=(\alph*)]
        \item $ker(\alpha) := \{v \in V : \alpha(v) = \sigma\}$, \underline{Kern von $\alpha$}, ist Unterraum von V.

        \item $\alpha$ injektiv $\Leftrightarrow$ $ker(\alpha) = \{\sigma\}$

        \item Ist $\alpha$ bijektiv, so ist die Umkehrabbildung $\alpha^{-1} : W \rightarrow V$ bijektiv \underline{und linear}.
      \end{enumerate}

      \subsection{Beweis}

      \begin{enumerate}[label=(\alph*)]
        \item $v_1, v_2 \in ker(\alpha)$

        $\alpha(v_1 + v_2) = \alpha(v_1) + \alpha(v_2) = \sigma + \sigma = \sigma$

        \par \medskip

        Also: $v_1 + v_2 \in ker(\alpha)$

        \par \medskip

        $\alpha(k \cdot v_1) = k \cdot \alpha(v_1) = k \cdot \sigma = \sigma$

        \par \medskip

        Also: $k \cdot v_1 \in ker(\alpha)$

        \item $\Rightarrow : \checkmark$, denn falls $\sigma \neq v \in ker(\alpha)$, so $\alpha(v) = \sigma = \alpha(\sigma)$ $\rightarrow \alpha$ nicht injektiv. Widerspruch !

        $\Leftarrow$: Angenommen $v_1, v_2 \in V$ mit $\alpha(v_1) = \alpha(v_2)$.

        Zu zeigen $v_1 = v_2$.

        $\sigma = \alpha(v_1) = \alpha(v_2) = \alpha(v_1 - v_2))$

        $\rightarrow v_1 - v_2 = \sigma, v_1 = v_2$

        \item Zu zeigen: $\alpha^{-1}$ ist linear

        Seien $w_1, w_2 \in W$.

        Zeige: $\alpha^{-1}(w_1+w_2) = \alpha^{-1}(w_1) + \alpha^{-1}(w_2)$

        $\alpha$ bijektiv $\rightarrow$ ex. $v_1. v_2 \in V$ mit $\alpha(v_1) = w_1, \alpha(v_2) = w_2$.

        $v_1 = \alpha^{-1}(w_1) , v_2 = \alpha^{-1}(w_2)$

        \par \medskip

        $\alpha^{-1} (w_1+ w_2) = \alpha^{-1}(\alpha(v_1) + \alpha(v_2)) = \alpha^{-1}(\alpha(v_1 + v_2)) = v_1 + v_2 = \alpha^{-1}(w_1) + \alpha^{-1}(w_2) \checkmark$

        \par \medskip

        Homogenität analog.
      \end{enumerate}

      \section{Beispiel}

      $\alpha : \begin{cases}\mathbb{R}^3 \rightarrow \mathbb{R}^3 \\ \begin{pmatrix}x_1 \\ x_2 \\ x_3 \end{pmatrix} \mapsto \begin{pmatrix}x_1 \\ 2x_1 \\ x_1+x_2+2x_3 \end{pmatrix} \end{cases}$ 

      \par \medskip

      ist lineare Abbildung, da:

      \par \medskip

      $\alpha(\begin{pmatrix}x_1 \\ x_2 \\ x_3\end{pmatrix}) = \begin{pmatrix}1 & 0 & 0 \\ 2 & 0 & 0 \\ 1 & 1 & 2 \end{pmatrix} \cdot \begin{pmatrix}x_1 \\ x_2 \\ x_3 \end{pmatrix}$ (3.4)

      $\alpha(e_1) = \begin{pmatrix}1 \\ 2 \\ 1 \end{pmatrix}, \alpha(e_2) = \begin{pmatrix}0 \\ 0 \\ 1 \end{pmatrix}, \alpha(e_3) \begin{pmatrix}0 \\0 \\ 2 \end{pmatrix}$

      Bild von $\alpha$ wird erzeugt $\alpha(e_1), \alpha(e_2), \alpha(e_3)$ lineare abhängig.

      \par \medskip

      $\alpha(\mathbb{R}^3) = \left \langle \underbrace{\alpha(e_1), \alpha(e_2)}_{\textnormal{lin. unabhäng}} \right \rangle$

      $rg(\alpha) = 2$

      U = $\left \langle e_2, e_3 \right \rangle$ 2-dim. Unterraum von $\mathbb{R}^3$ 2-dim. Unterraum von $\mathbb{R}^3$.

      $\alpha(U) = \left \langle \alpha(e_2) \right \rangle = \left \langle e_3 \right \rangle$ 1-dim. 

      $ker(\alpha) = ?$

      Suche alle $\begin{pmatrix}x_1 \\ x_2 \\ x_3 \end{pmatrix}$ mit $\begin{pmatrix}x_1 \\ 2x_1 \\ x_1 + x_2 +2x_3 \end{pmatrix} = \begin{pmatrix}0 \\ 0 \\ 0 \end{pmatrix}$

      \par \medskip

      \par \medskip

      LGS: $x_1 = 0 , 2x_1 = 0, x_1 + x_2 + 2x_3 = 0$

      \par \medskip

      $ker(\alpha) =  \left \lbrace \begin{pmatrix}0 \\ -2c \\ c \end{pmatrix} : c \in \mathbb{R} \right \rbrace$ = $\left \langle \begin{pmatrix}0 \\ -2 \\ 1 \end{pmatrix} \right \rangle$ 1-dim.

      \section{Satz}

      V, W, K-VR, $dim(V) = n$.

      $\{v_1, ..., v_n \}$ sei Basis von V, $w_1, ..., w_n \in W$ beliebig (nicht notwendigerweise verschieden).

      Dann existiert genau eine lineare Abbildung $\alpha : V \rightarrow W$ mit $\alpha(v_i) = w_i$, $i=1,...,n$, nämlich:

      \begin{center}
        $\alpha(\sum_{i=1}^n c_i v_i) := \sum_{i=1}^n c_i w_i \ \ (*)$
      \end{center}

      Also: Kennt man die Bilder einer Basis, so kennt man die lineare Abbildung vollständig.

      \subsection{Beweis}
      Die in $(*)$ definierte Abbildung $\alpha$ ist linear und es gilt $\alpha(v_i) = w_i$ für $i=1, ..., n$ (Nachrechnen)

      $\alpha$ eindeutig.

      Angenommen : $\beta : V \rightarrow W$ linear mit $\beta(v_i) = w_i$, so gilt:

      \begin{center}
        $\beta(\sum_{i=1}^n c_i v_i) = \sum_{i=1}^n c_i \beta(v_i) = \sum_{i=1}^n c_i w_i = \alpha(\sum_{i=1}^n c_i v_i)$

        $\rightarrow \alpha = \beta$
      \end{center}

      \textbf{Beispiel}

      V = W = $\mathbb{R}^3$

      $\alpha(e_1) = \begin{pmatrix} \\ -12 \\ 3 \end{pmatrix}, \alpha(e_2) = \begin{pmatrix}1 \\ 1 \\ 1 \end{pmatrix}, \alpha(e_3) = \begin{pmatrix}0 \\ -5 \\ 0 \end{pmatrix}$

      \par \medskip

      $\alpha(\begin{pmatrix}2 \\ 3 \\4 \end{pmatrix}) = ?$

      $\alpha(\begin{pmatrix}2 \\3 \\¼ \end{pmatrix}) = 2 \cdot \begin{pmatrix}-2 \\ -12\\ 3 \end{pmatrix} + 3 \cdot \begin{pmatrix}1 \\ 1 \\ 1 \end{pmatrix} + 4 \cdot \begin{pmatrix}0 \\ -5 \\ 0 \end{pmatrix} = \begin{pmatrix}7 \\ -51 \\9 \end{pmatrix} \checkmark$

      \section{Beispiel}

      V = $\mathbb{R}^2$, $\alpha: V \rightarrow V$. Drehung um Winkel $\varphi$, $0 \le \varphi < 2 \pi$, um Nullpunkt (entgegen Uhrzeigersinn). $\alpha$ ist lineare Abbildung. 

      \par \medskip

      $\alpha(e_1) = \alpha(\begin{pmatrix}1 \\ 0 \end{pmatrix}) = \begin{pmatrix}cos(\varphi) \\ sin(\varphi) \end{pmatrix}$

      \par \medskip

      $\alpha(e_2) = \alpha(\begin{pmatrix}0 \\ 1 \end{pmatrix} = \begin{pmatrix}-sin(\varphi) \\ cos(\varphi)\end{pmatrix}$

      3.10

      $x = \begin{pmatrix}x_1 \\ x_2 \end{pmatrix}$

      $\alpha(x) = x_1 \cdot \alpha(e_1) + x_2 \cdot \alpha(e_2) = \begin{pmatrix}x_1 \cdot cos(\varphi) - x_2 \cdot sin(\varphi) \\ x_1 \cdot sin(\varphi) - x_2 \cdot cos(\varphi) \end{pmatrix} = \underbrace{\begin{pmatrix}cos(\varphi) & - sin(\varphi) \\ sin(\varphi) & cos(\varphi) \end{pmatrix}}_{\textnormal{Drehmatrix}} \cdot \begin{pmatrix}x_1 \\ x_2 \end{pmatrix}$

      \section{Satz}

      $\alpha: V \rightarrow W$ lin. Abbildung

      $dim(V) = n, \{v_1, ..., v_n$\} Basis von V.

      \begin{enumerate}[label=(\alph*)]
        \item $\alpha$ injektiv $\Leftrightarrow$ \{$\alpha(v_1), ..., \alpha(v_n)$\} ist linear unabhängig. 
        \item $\alpha$ surjektiv $\Leftrightarrow$ $\alpha$ bijektiv $\Leftrightarrow$ $\{\alpha(v_1), ..., \alpha(v_n)\}$ Basis von W.
      \end{enumerate}

      \subsection*{Beweis}

      \begin{enumerate}[label=(\alph*)]
        \item $\Rightarrow$:

        Zeige: $\sum_{i=1}^n c_i \alpha(v_i) = \sigma$ $\rightarrow c_1 = ... = c_n = 0$.

        $\sigma = \sum_{i=1}^n c_i \alpha(v_i) = \alpha(\sum_{i=1}^n c_i v_i)$

        \par \medskip

        $\sum_{i=1}^n c_i v_i \in ker(\alpha) \underbrace{=}_{\textnormal{ $\alpha$ injektiv}} \{\sigma\}$

        $\rightarrow \sum_{i=1}^n c_i v_i = \sigma \Rightarrow c_1 = ... = c_n = 0$ ($v_1, ..., v_n$ linear unabhängig)

        $\Leftrightarrow$:

        Zeige: $ker(\alpha) = \{\sigma\}$

        Angenommen : $\sigma_{i=1}^n c_i v_i \in ker(\alpha)$

        $\sigma = \alpha(\sum_{i=1}^n c_i v_i) \underbrace{=}_{\textnormal{$\alpha$ lin.}} = \sum_{i=1}^n c_i \alpha(v_i) \Rightarrow$ ($\alpha(v_1), ..., \alpha(v_n)$ linear unabhängig.)

        $\Rightarrow c_1 = ... = c_n = 0$

        $\Rightarrow \sum_{i=1}^n c_i v_i = \sigma$ $\checkmark$

        \item $\alpha(V) = \left \langle \alpha(v_1), ..., \alpha(v_n) \right \rangle$

        Behauptung folgt.

        \item Folgt aus (a) und (b)

      \end{enumerate}

        \section{Korollar}

        Seien V,W K-VR, $dim(V) = dim(W)$. Dann sind V und W isomorph.

        \subsection{Beweis}

        Sei $v_1, ..., v_n$ Basis von V, $w_1, ..., w_n$ Basis von W.

        Nach 3.10 existiert genau eine lin. Abbildung $\alpha : V \rightarrow W$ mit $\alpha(v_i) = w_i$.

        Nach 3.12c) ist $\alpha$ bijektiv,

        \begin{center}
          $V \cong W$
        \end{center}



        \section{Korollar}

        V n-dim. VR über K, $\mathcal{B} = (v_1, ..., v_n)$ geordnete basis von V. Dann ist die Abb. 

        \begin{equation}
          K_{\mathcal{B}} : \begin{cases}V \rightarrow K^n \textnormal{ (Zeilenvektoren)}\\ \sum_{i=1}^n c_i v_i \mapsto (c_1, ..., c_n) \end{cases}
        \end{equation}

        (Koordinatenabbildung bezüglich $\mathcal{B}$) ein Isomorphismus. Das heißt $V \cong K^n$.

        \subsection{Beweis}

        $K_{\mathcal{B}}(v_i) = (0,..,0,1,0,...,0)$

        $v_i$ werden auf die kanonische Basis des $K^n$ abgebildet.

        $K_{\mathcal{B}}$ ist Isomorph.

        \section{Satz (Dimensionsformel)}

        V endlich dim. K-VR, $\alpha: V \rightarrow W$ lin. Abbildung.

        Dann:

        $dim(V) = rg(\alpha) + dim(ker(\alpha)) = dim(\alpha(v)) + dim(ker(\alpha))$

        \subsection{Beweis}

        Sei $u_1, ..., u_k$ Basis von $ker(\alpha)$.

        Basisergänzungssatz (2.21.c). Ergänze zu Basis $u_1, ..., u_k, u_{k+1}, ..., u_n$ von V.

        Sei U = $\left \langle u_{k+1}, ..., u_n \right\rangle_K$ Unterraum von V, $ker(\alpha) \cap U = \{\sigma\}$:

        Angenommen: $v \in ker(\alpha) \cap U$

        v = $\sum_{i=1}^k c_i u_i = \sum_{i=k+1}^n c_i u_i$

        $\Rightarrow \sum_{i=1}^k c_i u_i + \sum_{i=k+1}^n (-c_i)u_i = \sigma$

        $\Rightarrow$ $c_1 = ... = c_n = 0$, $v = \sigma$

        $ker(\alpha) \cap U = \{\sigma\}$, also $\alpha|_{U}$ ist injektiv, d.h. $dim(U) = dim(\alpha(U))$

        $\alpha(V) = \alpha(U)$

        $v \in V, v = \sum_{i=1}^k c_i u_i + \sum_{i=k+1}^n c_i u_i$

        $\alpha(v) = \underbrace{\sum_{i=1}^k c_i \alpha(u_i)}_{0} + \sum_{i=k+1}^n c_i \alpha(u_i) \in \alpha(U)$

        $V = ker(\alpha) + U$

        $dim(V) = dim(ker(\alpha)) + dim(U) - \underbrace{dim(ker(\alpha) \cap U)}_{=0}$

        $= dim(ker(\alpha)) + dim(\alpha(U))$

        $= dim(ker(\alpha)) + dim(\alpha(v))$

        $= dim(ker(\alpha)) + rg(\alpha)$

        \section{Korollar}

        V, W endlich-dimensional K-VR mit \underline{$dim(V) = dim(W)$}, $\alpha: V \rightarrow W$ linear. 

        Dann gilt:

        $\alpha$ ist injektiv $\Leftrightarrow$ $\alpha$ ist surjektiv $\Leftrightarrow$ $\alpha$ ist bijektiv

        \subsection{Beweis}

        $\alpha$ ist surjektiv $\Leftrightarrow$ $\alpha(v) = w$ $\Leftrightarrow$ dim($\alpha(V)) = dim(W) = dim(V) \Leftrightarrow$ dim(ker($\alpha$)) = 0 $\Leftrightarrow$ ker($\alpha$) = $\{\sigma\} \Leftrightarrow$ $\alpha$ ist injektiv.

        \chapter{Der Rang einer Matrix und lineare Gleichungssysteme}

        \section{Definition}

        Der \underline{Zeilenrang} einer Matrix A über Körper K ist die Maximalzahl linear unabhängiger Zeilen in A. Das heißt sind $z_1, ..., z_m$ die Zeilen von A, so ist Zeilenrang von A = dim($\left \langle z_1, ..., z_m \right \rangle$)

        \par \medskip

        Analog: Spaltenrang

        \par \medskip

        $A = \begin{pmatrix}1 & -2 & 2 \\ 1 & -2 & 1 \\ 1 & -2 & 0 \end{pmatrix}$ Spaltenrang von A = 2, Zeilenrang $z_1 + 2z_3 - 2z_2 = 0$, Zeilenrang von A = 2.

        \section{Satz}

        Bei elementaren Zeilenumformungen ändert sich der Zeilenrang einer Matrix nicht. (Analog: Spaltenumf. / Spaltenrang)

        \subsection{Beweis}

        $\left \langle z_1, ..., z_m \right \rangle$

        = $\left \langle z_1, ..., az_2, ..., z_m \right \rangle$, $a \neq 0$

        \par \medskip

        $\left \langle z_1, ..., z_m \right \rangle$ = $\left \langle z_1, ..., z_i + az_j, ..., z_m \right \rangle$, $i \neq j$

        \section{Bemerkung}

        Zeilenrangbest. von A: 

        Bringe A mit Gauß auf Zeilenstufenform (ändert Zeilenrang nicht)

        Zeilenrang = Anzahl der von Nullzeile verschiedenen Zeilen.

        \section{Korollar}

        Sei $A \cdot x = b$ ein LGS über K, $A \in \mathcal{M}_{m,n}(k), x \in K^n, b \in K^m$ (m Gleichungen, n Unbekannte)

        \begin{enumerate}[label=(\alph*)]
          \item $A \cdot x = b$ ist genau dann lösbar, wenn Zeilenrang von A = Zeilenrang von ($A|b$)

          \item $A \cdot x = b$ ist genau dann eindeutig lösbar, wenn:

          Zeilenrang A = Zeilenrang von ($A|b$) = n (= Anzahl der Unbekannten)

          \item Dimension des Lösungsraums von $A \cdot x = \sigma$ = n - Zeilenrang von A
        \end{enumerate}

        \section{Satz}

        Sei $A \in \mathcal{M}_{m,n} (K)$, $\alpha: \begin{cases}K^n \rightarrow K^m \\ x \mapsto Ax \end{cases}$

        $\alpha$ ist lineare Abbildung und es gibt:

        \begin{center} 
          $rg(\alpha) = $ Spaltenrang von A
        \end{center}

        \subsection{Beweis}

        $\alpha(K^n) = \left \langle \alpha(e_1), ..., \alpha(e_n) \right \rangle$, $e_1, ..., e_n$ kan. Basis von $K^n$

        \begin{center}
          $\alpha(e_i) = A \cdot \begin{pmatrix}0 \\ ... \\ 1 \\ 0 \\ 0 \end{pmatrix} = \begin{pmatrix}a_{1i} \\ a_{2i} \\ ... \\ a_{mi}\end{pmatrix}$ = i-te Spalte von A =: $s_i$.
        \end{center}

        $rg(\alpha) = dim(\alpha(K^n)) = dim(\left \langle \alpha(e_1), ..., \alpha(e_n) \right \rangle) = dim(\left \langle s_1, ..., s_n \right \rangle) = $ Spaltenrang von A.

        \section{Satz und Definition}

        Sei $A \in \mathcal{M}_{m,n} (K)$. Dann ist Zeilenrang von A = Spaltenrang von A. Diese gemeinsame Zahl heißt \underline{Rang von A}, $rg(A)$.

        (Also: $\alpha : \begin{cases}K^n \rightarrow K^m \\ x \mapsto Ax \end{cases}$, so $rg(\alpha) = rg(A)$)

        \subsection{Beweis}

        Betrachte homogenes LGS

        \begin{center}
          $Ax = 0$, $(*)$
        \end{center}

        Dimension des Lösungsraums von $(*)$ = Dimension von $ker(\alpha)$, $\alpha$ in 4.5.

        3.15: $dim(ker(\alpha)) = n - rg(\alpha)$ $\underset{4.5}{=}$ $n$ - Spaltenrang von A.

        4.4: $dim$ Lösungsraum von $Ax = 0$ = $n$ - Zeilenrenrang von A

        \par \medskip

        Damit folgt die Behauptung.

        \section{Korollar}

        $A \in \mathcal{M}_{m,n} (K)$. $rg(A) = rg(A^t)$

        \par \medskip

        \underline{Beweis:} $rg(A)$ = Zeilenrang von A = Spaltenrang von $A^t$ = $rg(A^t$)

        \section{Satz}

        Sei $V$ endlich dimensionaler K-VR, $\mathcal{B}$ geordnete Basis von V, $u_1, ..., u_m \in V$ beliebig.

        Seien $K_{\mathcal{B}}(u_i)$ die Koordinatenvektoren von $u_i$ bezüglich $\mathcal{B}$ (Zeilenvektoren).

        \par \medskip

        Dann gilt: $dim(\left \langle u_1, ..., u_m \right \rangle) = rg\begin{pmatrix}K_{\mathcal{B}}(u_1) \\ ... \\ K_{\mathcal{B}}(u_m) \end{pmatrix}$ ($m \times n$ - Matrix, $n = dim(V))$

        \par \medskip
        
        Lässt sich durch Gauß-Algorithmus bestimmen.

        \subsection{Beweis}

        Sei $U = \left \langle u_1, ..., u_m \right \rangle$. $K_{\mathcal{B}} : V \rightarrow K^n$ wie in 3.14. $K_{\mathcal{B}}$ Isomorphismus.

        \par \medskip

        $dim(U) = dim(K_{\mathcal{B}}(U)) = dim(\left \langle K_{\mathcal{B}}(u_1), ..., K_{\mathcal{B}}(u_m) \right \rangle)$ = Zeilenrang von $\begin{pmatrix}K_{\mathcal{B}}(u_1) \\ ... \\ K_{\mathcal{B}}(u_m) \end{pmatrix}$

        \section{Beispiel}

        V R-VR aller Polynome vom Grad $\le$ 3. $dim(V) = 4$, Basis $\mathcal{B} = (1,x,x^2, x^3)$ 

        \begin{center}
          U = $\left \langle 1 + 6x^2 + x^3, 2x-2x^2 + 3x^3, 3x+x², 2x+15x^2-x^3 \right \rangle_{\mathbb{R}}$
        \end{center}

        $dim(U) =$ ?

        \par \medskip

        \par \medskip

        Bilde gemäß 4.8 die Matrix der Koordinatenvektoren der $u_i$ bezüglich $\mathcal{B}$.

        \par \medskip

        $\begin{pmatrix}1 & 0 & 6 & 1 \\ 0 & 2 & -2 & 3 \\ 0 & 3 & 1 & 0 \\ 2 & 1 & 15 & -1 \end{pmatrix} \rightarrow \begin{pmatrix}1 & 0 & 6 & 1 \\ 0 & 2 & -2 & 3 \\ 0 &  & 1 & 0 \\ 0 & 1 & 3 & -3 \end{pmatrix} \rightarrow \begin{pmatrix}1 & 0 & 6 & 1 \\ 0 & 1 & -1 & \frac{3}{2} \\ 0 & 0 & 4 & - \frac{9}{2} \\ 0 & 0 & 4 & - \frac{9}{2} \end{pmatrix} \rightarrow \begin{pmatrix}1 & 0 & 1 & 1 \\ 0 & 1 & -1 & \frac{3}{2} \\ 0 & 0 & 4 & - \frac{9}{2} \\ 0 & 0 & 0 & 0 \end{pmatrix}$

        \par \medskip
        
        Rang der Matrix = 3

        $dim(U) = 3$

        \chapter{Matrizen und lineare Abbildungen}

        \section{Definition}

        Seien V, W K-VR, $\mathcal{B} = (v_1, ..., v_n)$ und $\mathcal{C} = (w_1, ..., w_m)$ geordnete Basen von V bezw. W. Sei $\alpha : V \rightarrow W$ lineare Abbildung.

        \par \medskip

        Nach 3.10 ist $\alpha$ eindeutig bestimmt durch $\alpha(v_1), ..., \alpha(v_n)$. $(v = \sum_{i=1}^n b_i v_i \rightarrow \alpha(v) = \sum_{i=1}^n b_i \alpha(vi))$

        \par \medskip

        \par \medskip

        Stelle $\alpha(v_1), ..., \alpha(v_n)$ jeweils als Linearkombination von $w_1, ..., w_n$ dar:

        \begin{center}
          $\alpha(v_1) = a_{11}w_{1} + ... + a_{m1}w_m$

          $\alpha(v_2) = a_{21}w_{1} + ... + a_{m2} w_m$

          ...

          $\alpha(v_n) = a_{1m}w_{1} + ... + a_{mm} w_m$
        \end{center}

        (Ordnung der Indizes beachten!)

        \par \medskip

        Dann heißt die $m \times n$ - Matrix:

        \begin{equation}
          A_{\alpha}^{\mathcal{B, C}} := \begin{pmatrix}a_{11} & a_{12} & ... & a_{1m} \\ ... & ... & ... \\ a_{m1} & a_{m2} & ... & a_{mn} \end{pmatrix}
        \end{equation}

        die \underline{Darstellungsmatrix} von $\alpha$ bezüglich $\mathcal{B}$ und $\mathcal{C}$. (In den Spalten stehen die Koordinaten von $\alpha(v_i)$ bezüglich $\mathcal{C}$)

        (Abkürzende Schreibweise: $A_{\alpha}$, falls $\mathcal{B}$ und $\mathcal{C}$ aus Kontext klar)

        \par \medskip

        \par \medskip

        Falls V = W und $\mathcal{B} = \mathcal{C}$, so

        \begin{center}
          $A_{\alpha}^{\mathcal{B}} := A_{\alpha}^{\mathcal{B}, \mathcal{B}}$
        \end{center}

        \section{Bemerkung}

        \begin{enumerate}[label=(\alph*)]
          \item Bei Kenntnis von $\mathcal{B}$ und $\mathcal{C}$ ist $\alpha$ durch $A_{\alpha}^{\mathcal{B,C}}$ eindeutig bestimmt:

          Sei $v \in V$. $v = \sum_{i=1}^n b_i v_i$.

          \begin{align*}
            \alpha(v) & = \sum_{i=1}^n b_i \alpha(v_i) \\
                      & = \sum_{i=1}^n b_i (a_{1i} w_1 + ... + a_{mi} w_m) \\
                      & = \sum_{i=1}^n b_i (\sum_{j=1}^m a_{ji} w_j) \\
                      & = \sum_{j=1}^m \cdot \underbrace{(\sum_{i=1}^n a_{ji} b_i)}_{\textnormal{Koord. von $\alpha(v)$ bezgl. $\mathcal{C}$}} \cdot  w_j
          \end{align*}

          Und jede $m \times n$ - Matrix A bestimmt lin. Abbildung $\alpha : V \rightarrow W$ mit $A = A_{\alpha}^{\mathcal{B, C}}$

          \item Beachte: Dieselbe lin. Abb. hat im Allgemeinen bezüglich anderer Wahl der Basen eine andere Darstellungsmatrix
        \end{enumerate}

        \section{Beispiel}

        \begin{enumerate}[label=(\alph*)]
          \item V = W = $\mathbb{R}^2$, $\alpha$ Drehung um 0 mit Winkel $\varphi$ (entgegen Uhrzeigersinn).

          Nach 3.11:

          $\mathcal{B} = \mathcal{C} = (e_1, e_2)$

          $\alpha(e_1) = cos(\varphi)e_1 + sin(\varphi) e_2$

          $\alpha(e_2) = -sin(\varphi)e_1 + cos(\varphi)e_2$

          \par \medskip

          \begin{equation}
            A_{\alpha}^{\mathcal{B}} = \begin{pmatrix} cos(\varphi) & -sin(\varphi) \\ sin(\varphi) & cos(\varphi) \end{pmatrix}
          \end{equation}

          \item Nullabbildung

          $\beta : \begin{cases}V \rightarrow W \\ v \mapsto \sigma \textnormal{ (Nullvektor)} \end{cases}$

          hat bezüglich allen Basen $\mathcal{B}$ und $\mathcal{C}$ Nullmatrix als Darstellungsmatrix

          \item V, $\mathcal{B}$, $id_v$

          $A_{id_v}^{\mathcal{B}} = E_n = \begin{pmatrix}1 & ... & 0 \\ 0 & ... & 1 \end{pmatrix}$

          \item $V = \mathbb{R}^2, \mathcal{B} = (e_1, e_2), \mathcal{C} = (e_2, e_1)$

          $A_{id_v}^{\mathcal{B, C}} = \begin{pmatrix}0 & 1 \\  1 & 0 \end{pmatrix}$

          \item $V = \mathbb{R}^2, \mathcal{B} = (e_1, e_2)$, $\sigma$ Spiegelung an $\left \langle e_1 \right \rangle$, d.h. $\sigma(\begin{pmatrix}x_1 \\ x_2 \end{pmatrix}) = \begin{pmatrix}x_1 \\ - x_2 \end{pmatrix}$

          $A_{\sigma}^{\mathcal{B}} = \begin{pmatrix}1 & 0 \\ 0 & -1 \end{pmatrix}$

          \par \medskip

          \par \medskip

          $\mathcal{B'} = (e_1 + e_2, e_1 - e_2)$ Basis.

          $\sigma(e_1 + e_2) = e_1-e_2$

          $\sigma(e_1 - e_2) = e_1 + e_2$

          $A_{\sigma}^{\mathcal{B'}} = \begin{pmatrix}0 & 1 \\ 1 & 0 \end{pmatrix}$

          \par \medskip

          \par \medskip


          $\sigma(e_1) = e_1 = a(e_1 + e_2) + b(e_1 - e_2) = \frac{1}{2} (e_1 + e_2) + \frac{1}{2}(e_1 - e_2)$

          $- e_2 = \sigma(e_2) = c (e_1 + e_2) + d (e_1 - e_2) = - \frac{1}{2}(e_1 + e_2) + \frac{1}{2} (e_1 - e_2)$

          $A_{\sigma}^{\mathcal{B, B'}} = \begin{pmatrix}\frac{1}{2} & - \frac{1}{2} \\ \frac{1}{2} & \frac{1}{2} \end{pmatrix}$
        \end{enumerate}

        \section{Satz}

        V, W, $\mathcal{B}, \mathcal{C}, \alpha : V \rightarrow W$ linear.

        \begin{equation}
          k_{\mathcal{C}}(\alpha(V))^t = A_{\mathcal{\alpha}}^{\mathcal{B,C}} \cdot k_{\mathcal{B}}(V)^t
        \end{equation}

        \subsection{Beweis}

        Folgt aus 5.2.a)

        \begin{align*}
          \textnormal{Basis } \mathcal{B} & & \textnormal{Basis } \mathcal{C} \\
          V & \overset{\alpha}{\longrightarrow} & W \\
          \downarrow & & \downarrow \\
          K^n & \longrightarrow & K^m
        \end{align*}

        \section{Beispiel}

        V, W $\mathbb{R}$ - VR, $dim(V) = 4$, $dim(W) = 3$, $\mathcal{B} = (v_1, ..., v_4)$, $\mathcal{C} = (w_1, w_2, w_3), \alpha : V \rightarrow W$.

        \begin{center}
          $A_{\alpha}^{\mathcal{B, C}} = \begin{pmatrix}1 & 1 & 2 & 3 \\ 2 & 0 & -1 & 1 \\ 3 & 2 & 0 & 2 \end{pmatrix}$
        \end{center}

        $V = 5_1 - 6v_2 + 7v_3 - 2v_4$

        $\alpha(V) =$ ?

        \par \medskip

        $\begin{pmatrix}1 & 1 & 2 & 3 \\ 2 & 0 & -1 & 1 \\ 3& 2 & 0 & 2 \end{pmatrix} \cdot \begin{pmatrix}5 \\ -6 \\ 7 \\ -2 \end{pmatrix} = \begin{pmatrix}7 \\ 1 \\ -1 \end{pmatrix}$

        \par \medskip

        5.4: $\alpha(V) = 7w_1 + w_2 - w_3$

        \section{Korollar}

        Jede lineare Abbildung $K^n \rightarrow K^m$ ist von der Form $\alpha(x) = A \cdot x$ für ein $A \in \mathcal{M}_{m,n} (K)$.

        Es ist $A = A_{\alpha}^{\mathcal{B, C}}$, wobei $\mathcal{B, C}$ die kanonischen Basen von $K^n$ bzw. $K^m$ sind.

        \subsection{Beweis}

        $x \in K^m$, $k_{\mathcal{B}}(x)^t = x$, $k_{\mathcal{C}}(\alpha(x))^t = \alpha(x)$

        Behauptung folgt aus 5.4

        \section{Satz}

        $\alpha, \beta$ lineare Abbildung $U \rightarrow V$, $\gamma$ lin. Abbildung $V \rightarrow W$. $\mathcal{B,C,D}$ geordnete Basen von $U,V,W$.

        \begin{enumerate}[label=(\alph*)]
          \item $A_{\alpha + \beta}^{\mathcal{B,C}} = A_{\alpha}^{\mathcal{B, C}} + A_{\beta}^{\mathcal{B,C}}$

          $A_{k \cdot \alpha}^{\mathcal{B,C}} = k \cdot A_{\alpha}^{\mathcal{B,C}}$ ($k \in K$)

          \item $A_{\gamma \circ \alpha}^{\mathcal{B,D}} = A_{\gamma}^{\mathcal{C,D}} \cdot A_{\alpha}^{\mathcal{B,C}}$ (Matrixmultiplikation) (Reihenfolge beachten!)
        \end{enumerate}

        \subsection{Beweis}

        \begin{enumerate}[label=(\alph*)]
          \item Nachrechnen
          \item $\mathcal{B} = (u_1, ..., u_l)$

          $\mathcal{C} = (v_1, ..., v_m)$

          $\mathcal{D} = (w_1, ..., w_n)$

          \par \medskip

          \begin{center}
            $A_{\alpha}^{\mathcal{B,C}} = (a_{ij})$ $m \times l$ - Matrix 

            $A_{\gamma}^{\mathcal{C,D}} = (b_{ij})$ $n \times m$ - Matrix
          \end{center}

          \begin{align*}
            (\gamma \circ \alpha) (u_i) & = \gamma(\alpha(u_i)) \\
                                        & = \gamma(\sum_{j=1}^m a_{ji} \cdot v_j) \\
                                        & = \sum_{j=1}^m a_{ji} \cdot \gamma(v_j) \\
                                        & = \sum_{j=1}^m a_{ji} (\sum_{k=1}^m b_{k_j} w_k) \\
                                        & = \sum_{k=1}^m (\underbrace{\sum_{j=1}^m b_{k_j} \cdot a_{ji}}_{\textnormal{Koeff. (k,i)}}) w_k
          \end{align*}
        \end{enumerate}

        \section{Beispiel}

        $U = V = W = \mathbb{R}^2$

        $\mathcal{B = C = D} = (e_1, e_2)$

        \par \medskip

        $\alpha$ Drehung um $\varphi$, $\beta$ Drehung um $\psi$ (jeweils um 0)

        \begin{equation}
          A_{\alpha}^{\mathcal{B}} = \begin{pmatrix}cos(\varphi) & - sin(\varphi) \\ sin(\varphi) & cos(\varphi)\end{pmatrix}
        \end{equation}

        \begin{equation}
          A_{\beta}^{\mathcal{B}} = \begin{pmatrix}cos(\psi) & - sin(\psi) \\ sin(\psi) & cos(\psi)\end{pmatrix}
        \end{equation}

        $\beta \circ \alpha$ Drehumg um $\varphi$ + $\psi$

        \begin{equation}
          A_{\beta \circ \alpha}^{\mathcal{B}} = \begin{pmatrix}cos(\varphi + \psi) & - sin(\varphi + \psi) \\ sin(\varphi + \psi) & cos(\varphi + \psi) \end{pmatrix}
        \end{equation}

        Nach 5.7

        \par \medskip

        $A_{\beta \circ \alpha}^{\mathcal{B}} = A_{\beta}^{\mathcal{B}} \cdot A_{\alpha}^{\mathcal{B}} = \begin{pmatrix}cos(\varphi) cos(\psi) - sin(\varphi) sin(\psi) & -sin(\varphi)cos(\psi) - cos(\varphi) sin(\psi) \\ cos(\varphi) sin(\psi) + sin(\varphi) cos(\psi) & -sin(\varphi) sin(\psi) + cos(\varphi) cos(\psi) \end{pmatrix}$ 

        \par \medskip

        $\Rightarrow$ $cos(\varphi + \psi) = cos(\varphi)cos(\psi) - sin(\varphi) sin(\psi)$, usw.

        (Additionstheoreme der Tirgonometrie)

        \section{Definition}

        Sei $A \in \mathcal{M}_{n} (K)$ ($n \times n$ - Matrix).

        A heißt \underline{invertierbar}, falls $A^{-1} \in \mathcal{M}_{n} (K)$ existiert (\underline{Inverse}, \underline{inverse Matrix} zu A) mit 

        \begin{center}
          $A \cdot A^{-1} = A^{-1} \cdot A = E_n =$ Einheitsmatrix $(*)$
        \end{center}

        (($\mathcal{M}_n (K), \cdot$) ist Monoid, neutrales Element $E_n$)

        \subsection{Bemerkung}

        Gilt $A \cdot A^{-1} = E_n$ so auch $A^{-1} \cdot A = E_n$ (und umgekehrt). (Folgt aus 5.10 und 3.16)

        \section{Korollar}

        $dim_K(V) = n$, $\mathcal{B}$ geord. Basis von V, $\alpha : V \rightarrow V$ linear. Dann gilt:

        \begin{center}
          $\alpha$ invertierbar (d.h. bijektiv) $\Leftrightarrow$ $A_{\alpha}^{\mathcal{B}}$ invertierbar
        \end{center}

        Dann: $A_{\alpha^{-1}}^{\mathcal{B}} = (A_{\alpha}^{\mathcal{B}})^{-1}$

        \subsection{Beweis}

        \subsubsection*{$\Rightarrow$}

        $A_{\alpha}^{\mathcal{B}} \cdot A_{\alpha^{-1}}^{\mathcal{B}}$ $\underbrace{=}_{5.7} A_{\alpha \circ \alpha^{-1}}^{\mathcal{B}} = A_{id_y}^{\mathcal{B}} = E_n$

        Gegenrichtung analog.

        \subsubsection*{$\Leftarrow$}

        Es existiert inverse Matrix $B$ zu $A_{\alpha}^{\mathcal{B}}$, d.h. $A_{\alpha}^{\mathcal{B}} \cdot B = B \cdot A_{\alpha}^{\mathcal{B}} = E_n$

        Dann B = $A_{\beta}^{\mathcal{B}}$ für eine eindeutig bestimmte lineare Abbildung $\beta : V \rightarrow V$ (5.2)

        \par \medskip

        $A_{\alpha}^{\mathcal{B}} \cdot A_{\beta}^{\mathcal{B}} = A_{\beta}^{\mathcal{B}} \cdot A_{\alpha}^{\mathcal{B}} = E_n$


        Damit folgt $\alpha \circ \beta = id_v$ Analog: $\beta \circ \alpha = id_v$ $\beta = \alpha^{-1}$

        \section{Satz}

        $A \in \mathcal{M}_n (K)$. A invertierbar $\Leftrightarrow$ rg(A) = n (D.h. Zeilen $|$ Spalten von A sind lin. unabhängig)

        \subsection{Beweis}

        Def. $\alpha : K^n \rightarrow K^n$ durch $\alpha(x) = A \cdot x$.

        $A = A_{\alpha}^{\mathcal{B}}$ bezüglich der kanonischen Basis B von $K^n$

        A invertierbar $\underbrace{\Leftrightarrow}_{5.10} \alpha$ invertierbar $\underbrace{\Leftrightarrow}_{3.16}$ $\alpha$ surjektiv / injektiv $\Leftrightarrow$ $rg(\alpha) = n \Leftrightarrow rg(A) = n$

        \section{Lemma}

        $A \in \mathcal{B}_{m,n} (K), X \in \mathcal{M}_{n,l} (K)$, $C = AX \in \mathcal{M}_{m,l} (K)$

        Wendet man dieselben elementaren Zeilenumformungen auf A und C an (beachte: A und C haben beide m Zeilen) so gilt für die entstehenden Matrizen A', C'

        \begin{center}
          C' = A' X
        \end{center}

        \section{Bestimmung der Inversen einer invertierbaren Matrix (Gauß-Jordan-Verfahren)}

        A invertierbare $n \times n$ - Matrix. Gesucht $A^{-1}$ mit:
        
        \begin{center}
          $A  \cdot A^{-1} = E_n$
        \end{center}

        Mann kann A durch elementare Zeilenumformungen auf die Form $E_n$ bringen. Analog zu Gauß-Algorithmus:

        \begin{center}
          $A \longrightarrow \begin{pmatrix}1 & * & * & ... \\ 0 & * & * & ... \\ ... & ... & ... & ... \\ 0 & * & * & ... \end{pmatrix}$
        \end{center}

        $rg(A) = n$: In der zweiten Spalte findet man Eintrag $\neq 0$ unterhalb (einschließlich) der Diagonalen.

        Erzeuge wie bei Gauß 1 in der Diagonale, unterhalb der Diagnoale erzeuge Nullen \underline{und} auch oberhalb.

        So fortfahren (keine Vertauschung von Zeilen oberhalb der Diagonalen !).

        \begin{center}
          $A \cdot A^{-1} = E_n$
        \end{center}

        Durch elementare Zeilenumformung entsteht aus A die Einheitsmatrix $E_n$. Dieselben zeilenumformungen angwandt auf $E_n$ liefert Matrix $A'$.

        \par \medskip

        5.12: $E_n \cdot A^{-1} = A'$

        \begin{center}
          $(A | E_n) \longrightarrow (E_n|A^{-1})$
        \end{center}

        (Verf. zeigt gleichzeitig, ob A invertierbar ist)

        \section{Beispiel}

        $A = \begin{pmatrix}1 & 0 & 2 \\ 2 & 2 & 1 \\ 0 & 1 & 0 \end{pmatrix} \in \mathcal{M}_3(\mathbb{Q})$

        \par \medskip

        $\begin{pmatrix}1 & 0 & 2 & | & 1 & 0 & 0 \\ 2 & 2 & 1 & | & 0 & 1 & 0 \\ 0 & 1 & 0 & | & 0 & 0 & 1 \end{pmatrix} \rightarrow \begin{pmatrix}1 & 0 & 2 & | & 1 & 0 & 0 \\ 0 & 2 & -3 & | & -2 & 1 &  0 \\ 0 & 1 & 0 & | & 0 & 0 & 1 \end{pmatrix} \rightarrow \begin{pmatrix}1 & 0 & 2 & | & 1 & 0 & 0 \\ 0 & 1 & \frac{-3}{2} & | & -1 & \frac{1}{2} & 0 \\ 0 & 1 & 0  & | & 0 & 0 & 1 \end{pmatrix}$

        \par \medskip
        \par \medskip

        $\rightarrow \begin{pmatrix}1 & 0 & 2 & | & 1 & 0 & 0 \\ 0 & 1 & - \frac{3}{2} & | & -1 & \frac{1}{2} & 0 \\ 0 & 0 & \frac{3}{2} & | & 1 & - \frac{1}{2} & 1 \end{pmatrix} \rightarrow \begin{pmatrix}1 & 0 & 2 & | & 1 & 0 & 0 \\ 0 & 1 & - \frac{3}{2} & | & -1 & \frac{1}{2} & 0 \\ 0 & 0 & 1 & | & \frac{2}{3} & - \frac{1}{3} & \frac{2}{3} \end{pmatrix}$

        \par \medskip
        \par \medskip

        $\rightarrow \begin{pmatrix}1 & 0 & 0 & | & - \frac{1}{3} & \frac{2}{3} & - \frac{4}{3} \\ 0 & 1 & 0 & | & 0 & 0 & 1 \\ 0 & 0 & 1 & | & \frac{2}{3} & - \frac{1}{3} & \frac{2}{3} \end{pmatrix}$

        \section{Bemerkung}

        Sei $A x = b$ LGS mit $n$ Gleichungen und $n$ Unbekannten (d.h. A $n \times n$ - Matrix).

        4.4.b: $A x = b$ hat eindeutige Lösung, wenn $rg(A) = n$. Dann ex. $A^{-1}$ und es gilt:

        \begin{center}
          $x = A^{-1} \cdot b$
        \end{center}

        \section{Definition}

        V K-VR mit geordneten Basen $\mathcal{B} = (v_1, ..., v_n)$, $\mathcal{B'} = (v'_1, ..., v'_n)$

        \par \medskip

        \begin{equation}
          v'_j = \sum_{i=1}^n s_{ij}v_i, j = 1, ..., n
        \end{equation}

        (Reihenfolge der Indizes beachten!)

        $S_{\mathcal{B, B'}} = (s_{ij})_{i,j=1,...,n}$ heißt \underline{Basiswechselmatrix}

        \par \medskip

        Spalten: Koordinaten der Basisvektoren aus $\mathcal{B}'$ bzgl. $\mathcal{B}$.

        Analaog.  $v_k = \sum_{j=1}^n t_{jk} v'_j$

        $S_{\mathcal{B', B}} = (t_{jk})_{j,k=1,...,n}$

        \section{Satz}

        Bezeichnung wie in 5.16

        $S_{\mathcal{B, B'}}$ ist invertierbar und $S_{\mathcal{B, B'}}^{-1} = S_{\mathcal{B', B}}$, d.h. $S_{\mathcal{B, B'}} \cdot S_{\mathcal{B', B}} = S_{\mathcal{B', B}} \cdot S_{\mathcal{B, B'}} = E_n$

        \subsection{Beweis}

        \begin{align*}
          V_k & = \sum_{j=1}^n t_{jk} v'_j \\
              & = \sum_{j=1}^n t_{jk} (\sum_{i=1}^n s_{ij} v_i) \\
              & = \sum_{i=1}^n (\sum_{j=1}^n s_{ij} t_{jk})v_i 
        \end{align*}

        $\sum_{j=1}^n s_{ij} \cdot t_{jk} = \begin{cases}0 \textnormal{ für $i \neq k$} \\ 1 \textnormal{ für $i = k$} \end{cases}$

        \begin{center}
          $S_{\mathcal{B, B'}} \cdot S_{\mathcal{B', B}} = E_n$
        \end{center}

        \section{Satz}

        $V, \mathcal{B, B'}$ wie oben, $v \in V$. 

        $K_{\mathcal{B'}}(v)^t$ = $S_{\mathcal{B', B}} \cdot K_{\mathcal{B}}(v)^t$

        \subsection{Beweis}

        Analog zu 5.4 (5.2.a)

        \section{Beispiel}

        V = $\mathbb{R}^2$, $\mathcal{B} = (e_1, e_2)$, $\mathcal{C} = (e_1 + e_2, e_1-2e_2) = ((1,1)^t, (1,-2)^t)$

        \par \medskip
        \par \medskip

        $S_{\mathcal{B, B'}} = \begin{pmatrix}1 & 1 \\ 1 & -2 \end{pmatrix}$

        \par \medskip
        \par \medskip

        $S_{\mathcal{B, B'}} = S_{\mathcal{B, B'}}^{-1}$

        5.14:

        $\begin{pmatrix}1 & 1 & | & 1 & 0 \\ 1 & -2 & | & 0 & 1 \end{pmatrix} \rightarrow \begin{pmatrix}1 & 1 & | & 1 & 0 \\ 0 & -3 & | & -1 & 1 \end{pmatrix} \rightarrow \begin{pmatrix}1 & 1 & | & 1 & 0 \\ 0 & 1 & | & \frac{1}{3} & - \frac{1}{3} \end{pmatrix} \rightarrow \begin{pmatrix}1 & 0 & | & \frac{2}{3} & \frac{1}{3} \\ 0 & 1 & | & \frac{1}{3} & - \frac{1}{3} \end{pmatrix}$

        $S_{\mathcal{B', B}} = \begin{pmatrix}\frac{2}{3} & \frac{1}{3} \\ \frac{1}{3} & - \frac{1}{3} \end{pmatrix}$

        V = $\begin{pmatrix}5 \\ 3 \end{pmatrix} \in K_{\mathcal{B}} (v)^t$

        $K_{\mathcal{B}}(v) = ?$

        $K_{\mathcal{B'}}(v)^t = \begin{pmatrix}\frac{2}{3} & \frac{1}{3} \\ \frac{1}{3} & - \frac{1}{3} \end{pmatrix} \cdot \begin{pmatrix}5 \\ 3 \end{pmatrix} = \begin{pmatrix}\frac{13}{3} \\ \frac{2}{3} \end{pmatrix}$

        \section{Satz}

        $\alpha : V \rightarrow W$ linear, $\mathcal{B, B'}$ geordnete Basen von V $\mathcal{C, C'}$ geordnete Basen W.

        Dann:

        \begin{center}
          $A_{\alpha}^{\mathcal{B', C'}} = S_{\mathcal{C', C}} \cdot A_{\alpha}^{\mathcal{B,C}} \cdot S_{\mathcal{B, B'}}$
        \end{center}

        \subsection{Beweis:}

        Sei $v \in V$.

        \begin{align*}
        A_{\alpha}^{\mathcal{B', C'}} \cdot K_{\mathcal{B'}}(v) & = K_{\mathcal{C}}(\alpha(v))^t  \\
                                                                & = S_{\mathcal{C', C}} \cdot K_{\mathcal{C}}(\alpha(v))^t \\
                                                                & = S_{\mathcal{C', C}} \cdot A_{\alpha}^{\mathcal{B, C}} \cdot K_{\mathcal{B}}(v)^t \\
                                                                & = S_{\mathcal{C', C}} \cdot A_{\alpha}^{\mathcal{B, C}} \cdot S_{\mathcal{B, B'}} \cdot K_{\mathcal{B}}(v)^t
        \end{align*}

        Wenn v alle Vektoren aus V durchläuft, durchläuft $K_{\mathcal{B}}(v)^t$ alle Vektoren aus $K^n$ (n = dim(V)). Daraus folgt Behauptung.

        \section{Korollar}

        $\alpha : V \rightarrow V, \mathcal{B}, \mathcal{B'}$ geordnete Basis von v.

        $S = S_{\mathcal{B, B'}}$. Dann:

        \begin{center}
          $A_{\alpha}^{\mathcal{B}}$ = $S^{-1} \cdot A_{\alpha}^{\mathcal{B}} \cdot S$
        \end{center}

        \subsection{Beweis}

        Folgt aus 5.20 und 5.17

        \par \medskip
        (Bemerkung: Zwei $n \times n$ - Matrizen A, B heißen ähnlich, wenn es eine invertierbare MatrixS gibt mit B = $S^{-1} A S$)

        \section{Beispiel}

        V = $\mathbb{R}^2$, $\mathcal{B} = (e_1, e_2)$

        $\mathcal{B'} = (e_1 + e_2, e_1 - 2_e2)$

        \par \medskip
        \par \medskip

        $S_{\mathcal{B, B'}} = \begin{pmatrix}1 & 1 \\ 1 & -2 \end{pmatrix}$,

        $S_{\mathcal{B', B}} = \begin{pmatrix}\frac{2}{3} & \frac{1}{3} \\ \frac{1}{3} & - \frac{1}{3} \end{pmatrix}$ (5.19)

        Sei $A_{\alpha}^{\mathcal{B}} = \begin{pmatrix}1 & 0 \\ 0 & -1 \end{pmatrix}$

        \par \medskip

        $\alpha$ ist Spiegelung an $e_1$ - Achse


        \par \medskip

        \par \medskip

        $A_{\alpha}^{\mathcal{B'}} = \begin{pmatrix}\frac{2}{3} & \frac{1}{3} \\ \frac{1}{3} & - \frac{1}{3} \end{pmatrix} \cdot \begin{pmatrix}1 & 0 \\ 0 & -1 \end{pmatrix} \cdot \begin{pmatrix}1 & 1 \\ 1 & -2 \end{pmatrix} = \begin{pmatrix}\frac{1}{3} & \frac{4}{3} \\ \frac{2}{3} & - \frac{1}{3}\end{pmatrix}$

        $\alpha(e_1+e_2) = \frac{1}{3} \cdot (e_1 + e_2) + \frac{2}{3} \cdot (e_1- 2e_2)$

        $\alpha(e_1 - 2e_2) = \frac{4}{3} \cdot (e_1 + e_2) - \frac{1}{3} \cdot (e_1 - 2e_2)$

        \chapter{Determinanten}

        $\mathcal{M}_{n}(K) \longrightarrow K$

        \section{Definition}
        
        $A \in \mathcal{M}_n(K), i,j, \in \{1,...,n\}$.

        $A_{ij} \in \mathcal{M}_{n-1}(K)$ ist die Matrix, die aus A entsteht, wenn man in A die i-te Zeile und j-te Spalte streicht.

        \subsection{Beispiel}

        \begin{center}
          $A = \begin{pmatrix}3 & 4 & 5 \\ 6 & 7 & 8 \\ 9 & 10 & 11 \end{pmatrix} \in \mathcal{M}_3(\mathbb{R}) \longrightarrow A_{11} = \begin{pmatrix}7 & 8 \\ 10 & 11 \end{pmatrix}, A_{23} = \begin{pmatrix}3 & 4 \\ 9 & 10 \end{pmatrix}$
        \end{center}

        Definiere Determinante einer \underline{quadratischen} Matrix rekursiv.

        \section{Laplacescher Entwicklungssatz}

        det: $\mathcal{M}_n(K) \rightarrow K$ ist eine Abbildung, die \underline{Determinante}, die folgendermaßen berechnet wird:

        \begin{itemize}
          \item[(1)] $det((a)) := a$
          \item[(2)] $A \in \mathcal{M}_n(K)$. Wähle irgendein $i \in \{1,...,n\}$.

          $det(A) = \sum_{j=1}^n (-1)^{i+j} \cdot a_{ij} \cdot det(A_{ij})$

          (Entwicklung nach der i-ten Zeile)

          (Schachbrettmuster der Vorzeichen)

          \item[(3)] Alternativ: 

          Wähle $j \in \{1,...,n\}$

          $det(A) = \sum_{i=1}^n (-1)^{i+j} \cdot a_{ij} \cdot det(A_{ij})$

          (Entwicklung nach der j-ten Spalte)
        \end{itemize}

        \subsection{Bemerkung}

        \textbf{Wichtig:} Egal nach welcher Zeile oder Spalte man entwickelt, es kommt immer dasselbe raus!

        \par \medskip

        (Schwierigster Beweis in der elementaren Determinantentheorie (WHK 10.4))


        \section{Beispiel}

        \begin{enumerate}[label=(\alph*)]
          \item $det\begin{pmatrix}a_{11} & a_{12} \\ a_{21} & a_{22} \end{pmatrix} = a_{11} \cdot a_{22} - a_{12} \cdot a_{21}$ (Entwicklung nach 1. Zeile)

          Entwicklung nach 2. Spalte: $-a_{12} \cdot a_{21} + a_{22} \cdot a_{11}$

          \item $A = \begin{pmatrix}2 & 0 & 3 \\ -1 & 0 & 4 \\ 2 & 3 & -1 \end{pmatrix} \in \mathcal{M}_3 (\mathbb{Q})$

          \par \medskip

          Entwicklung nach der 1. Zeile:

          \begin{align*}
            det(A) & = 2 \cdot det \begin{pmatrix}0 & 4 \\ 3 & -1 \end{pmatrix} - 0 \cdot det \begin{pmatrix}-1 & 4  \\ 2 & -1 \end{pmatrix} + 3 \cdot det \begin{pmatrix}-1 & 0 \\ 2 & 3 \end{pmatrix} \\
                   & = -24 - 0 -9 \\
                   & = -33
          \end{align*}

          Entwicklung nach der 2. Spalte:

          \begin{align*}
            det(A)& = -3 \cdot det \begin{pmatrix}2 & 3 \\ -1 & 4 \end{pmatrix} \\
                  & = -33
          \end{align*} 

          \underline{Allgemeine Strategie:} Verwende zur Determinantenberechnung eine Zeile oder Spalte mit möglichst vielen Nullen!

          \item $det \begin{pmatrix}a_{11} & 0 & 0 \\ a_{21} & a_{22} & 0 \\ ... & ... & 0 \\ a_{n1} & ... & a_{nn} \end{pmatrix} = a_{11} \cdot a_{22} \cdot ... \cdot a_nn$ (untere Dreiecksmatrix)

          Induktion nach n:

          n = 1 $\checkmark$

          n-1 $\rightarrow$ n:

          Entwicklung nach 1. Zeile

          \par \medskip

          \par \medskip

          Insbesondere: $det(E_n) = 1$

          \end{enumerate}

          \section{Korollar}

          $det(A) = det(A^t)$

          \section{Rechenregeln für Determinanten}

          Sei $A \in \mathcal{M}_n(K)$.

          \begin{enumerate}[label=(\alph*)]
            \item Zeilen bzw. Spaltenvertauschung ändern das Vorzeichen der Determinante.

            \item Addiert man das Vielfache einer Zeile / Spalte zu einer anderen Zeile / Spalte, so ändert sich die Determinante überhaupt nicht.

            \item Multipliziert man eine Zeile / Spalte von A mit $a \in K$ so ändert sich det(A) um Faktor a.

            \underline{Insbesondere:} 

            $A \in \mathcal{M}_n(K)$

            $det(a \cdot A) = a^n \cdot det(A)$

            \item $A, B \in \mathcal{M}_n(K).$

            $det(A \cdot B) = det(A) \cdot det(B).$

            (Determinantenmultiplikationssatz)

            (Aber: Im Allgemeinen $det(A+B) \neq det(A) + det(B)$)
          \end{enumerate}
        

        \section{Bemerkung}

        Strategie zur Det.berechnung:

        Wende auf A elementare Zeilen / Spaltenumformungen an, um Dreiecksgestalt zu erhalten. Dann 6.3.c

        \par \medskip

        (Buchführen über Vorzeichen!)

        \section{Beispiel}

        $A = \begin{pmatrix}0 & 1 & 2 & 3 \\ -1 & 0 & 0 & 4 \\  1 & 3 & 4 & 2 \\ 0 & 3 & 0 & 1 \end{pmatrix}, K = \mathbb{Q}$

        \par \medskip

        \begin{align*}
          det(A)  & = - det \begin{pmatrix}-1 & 0 & 0 & 4 \\ 0 & 1 & 2 & 3 \\ 1 & 3 & 4 & 2 \\ 0 & 3 & 0 & 1 \end{pmatrix}  \\
                  & = - det \begin{pmatrix}-1 & 0 & 0 & 4 \\ 0 & 1 & 2 & 3 \\ 0 & 3 & 4 & 6 \\ 0 & 3 & 0 & 1 \end{pmatrix}  \\
                  & = - det \begin{pmatrix}1 & 0 & 0 & 4 \\ 0 & 1 & 2 & 3 \\ 0 & 0 & -2 & -3 \\ 0 & 0 & -6 & -8 \end{pmatrix} \\
                  & = - det \begin{pmatrix}-1 & 0 & 0 & 4 \\ 0 & 1 & 2 & 3 \\ 0 & 0 &-2 & -3 \\ 0 & 0 & 0 & 1 \end{pmatrix} \\
                  & \underbrace{=}_{6.3.c)} -2 
        \end{align*}

        \section{Satz}

        $A \in \mathcal{N}(K)$. Dann gilt:

        \begin{center}
          A invertierbar $\Leftrightarrow$ $rg(A) = n$ $\Leftrightarrow det(A) \neq 0$
        \end{center}

        In diesem Fall gilt:

        \begin{center}
          $det(A^{-1}) = det(A)^{-1}$
        \end{center}

        [$\Rightarrow: \ A \cdot A^{-1} = E_n, 1 = det(E_n) = det(A \cdot A^{-1}) = det(A) \cdot det(A^{-1})$]

        \par \medskip

        Andere Berechnungsmethode von $A^{-1}$ mit Hilfe der Determinante.

        \section{Definition}

        $A \in \mathcal{M}_n(K)$. Die \underline{Adjunkte} $A^{ad}$ zu A ist $n \times n$ - Matrix über K:

        \begin{center}
          $A^{ad} := (b_{ij})_{i,j=1,...,n}$
        \end{center}

        wobei $b_{ij} = (-1)^{i+j} \cdot det(A_{ji})$ (Indizes beachten).

        \section{Satz}

        $A \in \mathcal{M}_n(K)$

        \begin{enumerate}[label=(\alph*)]
          \item $A^{ad} \cdot A = A \cdot A^{ad} = det(A) \cdot E_n$

          \item Ist $det(A) \neq 0$, so ist

          \begin{center}
            $A^{-1} = \frac{1}{det(A)} \cdot A^{ad}$
          \end{center}
        \end{enumerate}

        \section{Beispiel}

        $A = \begin{pmatrix}a_{11} & a_{12} \\ a_{21} & a_{22} \end{pmatrix}$.

        \par \medskip

        Angenommen:

        $det(A) = a_{11} \cdot a_{22} - a_{12} \cdot a_{21} \neq 0$.

        \begin{center}
          $A^{-1} = \frac{1}{det(A)} \cdot \begin{pmatrix}a_{22} & - a_{12} \\ - a_{21} & a_{11} \end{pmatrix}$
        \end{center}

        \section{Demerkung}

        $\alpha : V \rightarrow V $ lin. Abbildung. V endl. dimensional. $\mathcal{B, B'}$ Basen von V.

        \par \medskip

        \par \medskip

        \begin{center}
          $A_{\alpha}^{mathcal{B'}} = S^{-1} \cdot A_{\alpha}^{\mathcal{B}} \cdot S$
        \end{center}

        wobei $S = S_{\mathcal{B, B'}}$ (5.21)

        \par \medskip

        \par \medskip

        \begin{align*}
          det(A_{\alpha}^{\mathcal{B'}}) & = det(S^{-1} \cdot A_{\alpha}^{\mathcal{B}} \cdot S)  \\
                                         & = det(S^{-1}) \cdot det(A_{\alpha}^{\mathcal{B}}) \cdot det(S) \\
                                         & = det(A_{\alpha}^{\mathcal{B}}) \cdot \underbrace{det(S)^{-1} \cdot det(S)}_{1} \\
                                         & = det(A_{\alpha}^{\mathcal{B}})
        \end{align*}

        Daher definiert man:

        \begin{center}
          $det(\alpha) := det(A_{\alpha}^{\mathcal{B}})$
        \end{center}

        (unabhängig von der Wahl von $\mathcal{B}$) 

        \par \medskip

        [Im Allgemienen ist $det(A_{\alpha}^{\mathcal{B, C}}) \neq det(A_{\alpha}^{\mathcal{B', C'}})$]

        \chapter{Eigenwerte}

        \textbf{\underline{Problem:}} $\alpha: V \rightarrow V$ linear. Suche Basis $\mathcal{B}$ von V bezüglich der $A_{\alpha}^{\mathcal{B}}$ besonders einfache Gestalt hat. 

        Am besten wäre Dreiecksmatrix (Untere und Obere. Somit nur Diagonale $\neq 0$).

        Dh. $\mathcal{B} = (v_1, ..., v_n)$, so $\alpha(v_i) = a_iv_i$, $i=1, ...,n$ 

        Das geht allerdings im Allgemeinen nicht.

        \section{Beispiel:}

        \begin{enumerate}[label=(\alph*)]

          \item $\sigma : \mathbb{R}^2 \rightarrow \mathbb{R}^2$ Spiegelung an der $e_1-$Achse. $\mathcal{B} = (e_1, e_2)$ - kanonische Basis.

          \begin{center}
            $A_{\sigma}^{\mathcal{B}} = \begin{pmatrix}1 & 0 \\ 0 & -1 \end{pmatrix}$
          \end{center}

          (Diagonalmatrix)

          \item Drehung $\rho$ um 0 mit Winkel  $k \cdot \pi$.

          \par \medskip

          Kein Vektor $\neq \sigma$ wird auf ein Vielfaches von sich abgebildet.
        \end{enumerate}

        Für \underline{keine} Basis $\mathcal{B}$ ist $A_{\rho}^{\mathcal{B}}$ Diagonalmatrix.

        \section{Definition}

        $\alpha : V \rightarrow V$ lineare Abbildung. $c \in K$ heißt \underline{Eigenwert} von $\alpha$, falls $v \in V, v \neq \sigma$, existiert mit 

        \begin{center}
          $\alpha(v) = c \cdot v$ 
        \end{center}

        Jeder solcher Vektor $v \neq \sigma$ heißt \underline{Eigenvektor} von $\alpha$ zu dem Eigenwert $c$.

        \par \medskip

        Die Menge aller Eigenvektoren zu $c$, zusammen mit dem Nullvektor, heißt \underline{Eigenraum} von $\alpha$ zum Eigenwert $c$.

        \section{Bemerkung}

        $\alpha: V \rightarrow V$ linear, $c$ sei ein Eigenwert von $\alpha$. 

        \par \medskip

        Eigenraum von $\alpha$ zu $c$ = $ker(c \cdot id_v - \alpha)$, also Unterraum von V.

        \par \medskip

        Insbesondere: 0 ist Eigenwert von $\alpha$ $\Leftrightarrow ker(\alpha) \neq \{\sigma\}$

        \subsection{Beweis}

        $\alpha(v) = c \cdot v \Leftrightarrow c \cdot v - \alpha(v) = \sigma \Leftrightarrow (c \cdot id_v - \alpha) (v) = 0 \Leftrightarrow v \in ker(c \cdot id_v - \alpha)$

        \section{Beispiel}

        \begin{enumerate}[label=(\alph*)]
          \item $id_v$ hat nur Eigenwert 1, Eigenraum zu 1 ist V.

          \item Spiegelung aus 7.1.a):

          1 ist Eigenwert

          -1 ist Eigenwert

          Eigenraum zu 1: $\left \langle e_1 \right \rangle$

          Eigenraum zu -1: $\left \langle e_2 \right \rangle$

          \item Drehung um $\rho \neq k \cdot \pi$ hat keine Eigenwerte.
        \end{enumerate}

        \section{Definition}

        A $n \times n$ - Matrix über K. 

        \underline{Eigenwerte von A} := Eigenwerte von $\alpha_{A} : \begin{cases}K^n \rightarrow K^n \\ x \mapsto A \cdot x \end{cases}$

        (d.h. $c \in K$ ist Eigenwert von A $\Leftrightarrow \exists x \neq 0 \in K^n : A \cdot x = c \cdot x$)

        \section{Satz}

        $\alpha : V \rightarrow V$ lin. Abbildung. Dann haben $\alpha$ und $A_{\alpha}^{\mathcal{B}}$ die gleichen Eigenwerte für jede Basis $\mathcal{B}$ von V.

        \subsection{Beweis}

        Sei $c$ Eigenwert von $\alpha$, $v \neq \sigma$ mit $\alpha(v) = c \cdot v$.

        \par \medskip

        \par \medskip

        \begin{align*}
        A_{\alpha}^{\mathcal{B}} \cdot \mathcal{K}_{\mathcal{B}}(v)^t & = \mathcal{K}_{\mathcal{B}}(\alpha(v))^t \\
                                                                      & = \mathcal{K}_{\mathcal{B}}(c \cdot v⁾^t \\
                                                                      & = c \cdot \mathcal{K}_{\mathcal{B}}(v)^t
        \end{align*}

        Da $v \neq \sigma$ ist $\mathcal{K}_{\mathcal{B}}(v) \neq 0$.

        Also ist $c$ Eigenwert von $A_{\alpha}^{\mathcal{B}}$.

        Umgekehrt: Sei $0 \neq x = \begin{pmatrix}x_1 \\ ... \\ x_n \end{pmatrix} \in K^n$ mit $A \cdot x = c \cdot x$. ($c$ ist Eigenwert von A).

        \par \medskip

        \par \medskip

        Sei $v = \sum_{i=1}^n x_i v_i$, $\mathcal{B} = (v_1, ..., v_n)$.

        $\mathcal{K}_{\mathcal{B}}(v)^t = x$.

        Es folgt $\mathcal{K}_{\mathcal{B}}(\alpha(v)) = \mathcal{K}_{\mathcal{B}}(c \cdot v)$. 

        \par \medskip

        Dann $\alpha(v) = c \cdot v$

        $c$ ist Eigenwert von $\alpha$.

        \section{Satz}

        V n-dim. K-VR, $\mathcal{B}$ Basis von V, $\alpha: V \rightarrow V$ linear, $A := A_{\alpha}^{\mathcal{B}}$, $c \in K$.

        Dann sind äquivalent:

        \begin{itemize}
          \item[(1)] $c$ ist Eigenwert von $\alpha$
          \item[(2)] $ker(c \cdot id_v - \alpha) \neq \{\sigma\}$
          \item[(3)] $det(c \cdot E_n -A) = 0$
        \end{itemize}

        \subsection{Beweis}

        \subsubsection*{$(1) \Leftrightarrow (2)$}

        7.3

        \subsubsection*{$(2) \Leftrightarrow (3)$}

        $A_{c \cdot id_v -\alpha}^{\mathcal{B}} = c \cdot E_n - A$

        $\alpha(v) = c \cdot v$ 

        \par \medskip

        Es gilt:

        \begin{align*}
          det(c \cdot E_n -A) = 0 \\ 
          \Leftrightarrow c \cdot E_n -A \textnormal{ nicht invertierbar.} \\
          \Leftrightarrow c \cdot id_v - \alpha \textnormal{ nicht invertierbar.} \\
          \Leftrightarrow c \cdot id_v - \alpha \textnormal{ ist nicht injektiv} \\
          \Leftrightarrow ker(c \cdot id_v -\alpha) \neq \{\sigma\}
        \end{align*}

        Wie berechnet man Eigenwerte einer lin. Abbildung und wie viele gibt es ?

        \par \medskip

        Nach 7.7 muss man alle $c \in K$ bestimmen mit $det(c \cdot E_n -A) = 0$. Betrachte Funktion:

        \begin{center}
          $f_A : \begin{cases}K \rightarrow K \\ t \mapsto det(t \cdot E_n -A) \end{cases}$
        \end{center}

        \section{Satz}

        Die Funktion $f_A$ ist Polynomfunktion vom Grad n, d.h.

        \begin{align*}
          f_A(t) & = det(t \cdot E_n -A) \\
                 & = t^n + a_{n-1}t^{n-1} + ... + a_1t + a_0
        \end{align*}

        wobei $a_i \in K$ (unabhängig von t)

        \subsection{Beweis}

        Mit Entwicklungsformel. Machen wir hier nicht.

        \section{Definition}

        \begin{enumerate}[label=(\alph*)]
          \item Das Polynom $f_A(t) = det(t \cdot E_n -A) \in K[t]$ heißt \underline{charakteristisches Polynom} von $A \in \mathcal{M}_n(K)$.

          \item $\alpha : V \rightarrow V$ linear, $\mathcal{B}$ Basis von V, so 

          \begin{align*}
            det(t \cdot id_v - \alpha) & = det(A_{t \cdot id_v -\alpha}^{\mathcal{B}}) \\
                                       & = det(t \cdot E_n - A_{\alpha}^{\mathcal{B}})
          \end{align*}

          heißt charakteristisches Polynom von $\alpha$ (nach 6.12 unabhängig von $\mathcal{B}$).
        \end{enumerate} 

        \section{Korollar und Definition}

        $\alpha : V \rightarrow V$ linear, $dim(V) = n$.

        \begin{enumerate}[label=(\alph*)]
          \item c ist Eigenwert von $\alpha$ $\Leftrightarrow c$ ist Nullstelle des char. Polynoms von $\alpha$

          \par \medskip

          \underline{Vielfachheit} des Eigenwerts c := \underline{Vielfachheit} von $c$ als Nullstelle des char. Polynoms. 

          \item $\alpha$ hat höchstens n Eigenwerte (einschließlich Vielfachheit).
        \end{enumerate}

        \section{Beispiel}

        \begin{enumerate}[label=(\alph*)]
          \item $\rho : \mathbb{R}^2 \rightarrow \mathbb{R}^2$ Spiegelung an $\left \langle e_1 \right \rangle$ - Achse

          $\mathcal{B} = (e_1, e_2)$ kan. Basis.

          \par \medskip

          $A := A_{\rho}^{\mathcal{B}} = \begin{pmatrix}1 & 0 \\ 0 & -1 \end{pmatrix}$

          char. Polynom : $det(t \cdot E_2 - A) = det(\begin{pmatrix}t & 0 \\ 0 & t \end{pmatrix} - \begin{pmatrix}1 & 0 \\ 0 & -1 \end{pmatrix}) = det \begin{pmatrix}t-1  & 0 \\ 0 & t+1 \end{pmatrix} = (t-1) \cdot (t+1)$

          \par \medskip

          Nullstellen 1, -1 alle Eigenwerte von $\rho$

          \item $\alpha : \mathbb{R}^2 \rightarrow \mathbb{R}^2$, 

          $A = A_{\alpha}^{\mathcal{B}} = \begin{pmatrix}-1 & 2 \\ 4 & -3 \end{pmatrix}$

          $\mathcal{B}$ kanonische Basis.

          char. Polynom von $\alpha$:

          \begin{center}
            $det(t \cdot E_n - A_{\alpha}^{\mathcal{B}}) = det \begin{pmatrix}t+1 & -2 \\ -4 & t+3 \end{pmatrix} = (t+1) \cdot (t+3) - 8 = t^2 + 4t -5$
          \end{center}

          $t_{1,2} = -2 \pm \sqrt{4+5}$

          Eigenwerte von $\alpha$: 1, -5

          \par \medskip

          \par \medskip

          Eigenvektor zu 1: $\begin{pmatrix}x \\ y \end{pmatrix}$

          $1 \cdot \begin{pmatrix}x \\ y \end{pmatrix} = \alpha \cdot \begin{pmatrix}x \\ y \end{pmatrix} = \begin{pmatrix}-x + 2y \\ 4x -3y \end{pmatrix}$

          $\rightarrow$ x = y

          \par \medskip

          \par \medskip

          Eigenraum zu Eigenwert 1: $\left \langle \begin{pmatrix}1 \\ 1 \end{pmatrix} \right \rangle$

          \par \medskip

          Eigenwert zu 5:

          $-5 \cdot \begin{pmatrix}x \\ y \end{pmatrix} = \begin{pmatrix}-x + 2y \\ 4x -3y \end{pmatrix}$

          $\rightarrow y = -2x$

          Eigenraum zu EW-5: 

          $\left \langle \begin{pmatrix}1 \\ -2 \end{pmatrix} \right \rangle$

          \par \medskip

          $\mathcal{B'} = ((1,1)^t, (1,-2)^t)$

          $A_{\alpha}^{\mathcal{B'}} = \begin{pmatrix}1 & 0 \\  0 & -5 \end{pmatrix}$

          (Diagonalmatrix)

          \item $\rho : \mathbb{R}^2 \rightarrow \mathbb{R}^2$ Drehung um $\frac{\pi}{2}$ um 0. 

          $\mathcal{B}$ kan. Basis.

          $A = A_{\rho}^{\mathcal{B}} = \begin{pmatrix}cos(\frac{\pi}{2}) & - sin(\frac{\pi}{2}) \\ sin(\frac{\pi}{2}) & cos(\frac{\pi}{2}) \end{pmatrix} = \begin{pmatrix}0 & -1 \\ 1 & 0 \end{pmatrix}$

          char. Polynom:

          $f_A(t) = det(t \cdot E_2 - A) = det \begin{pmatrix}t & 1 \\ -1 & t \end{pmatrix} = t^2 + 1$

          Keine Nullstellen in $\mathbb{R}$, also hat $\rho$ keine EW in $\mathbb{R}$

          \par \medskip

          \par \medskip

          Fasst man $\rho$ als Abbildung $\mathbb{C}^2 \rightarrow \mathbb{C}^2$ auf, so gibt es EW i, -i.

          \par \medskip

          Die zugehörigen Eigenräume sind:

          $\left \langle \begin{pmatrix}1 \\ i \end{pmatrix} \right \rangle_{\mathbb{C}}, \left \langle \begin{pmatrix}1 \\ -i \end{pmatrix} \right \rangle_{\mathbb{C}}$
        \end{enumerate}

        \section{Korollar}

        $\alpha : V \rightarrow V$ linear. Falls Basis $\mathcal{B}$ von V existiert mit $A_{\alpha}^{\mathcal{B}}$ in Dreiecksgestallt, so sind die Diagonalelemente $a_{11}, a_{22}, ..., a_{nn}$ sämtliche Eigenwerte von $\alpha$ (mit Vielfachheit).

        \subsection{Beweis}

        $det(t \cdot E_n -A) = det \begin{pmatrix}t - a_{11} & ...  \\ 0 & t-a_{nn} \end{pmatrix}$ (Dreiecksmatrix) = $(t-a_{11}) \cdot ... \cdot (t-a_{nn})$ 

        \section{Bemerkung}

        Über $\mathcal{C}$ lässt sich für jede lineare Abbildung $\alpha : V \rightarrow V$ Basis $\mathcal{B}$ finden, so dass $A_{\alpha}^{\mathcal{B}}$ Dreiecksmatrix ist.


        \section{Satz}

        Seien $c_1, ..., c_r$ die paarweise verschiedenen Eigenwerte der lin Abb. $\alpha : V \rightarrow V$. Seien $v_1, ..., v_r$ zugehörige Eigenvektoren. Dann sind $v_1, ..., v_r$ linear unabhägnig. 

        \subsection{Beweis}

        Induktion nach r.

        $r = 1: \ v_1 \neq 0$ lin unabhängig $\checkmark$

        Behauptung sei richtig für $i-1$.

        Zu zeigen: Richtig für $i \le r$. 

        $v_1, ..., v_{i-1}$ lin unabhängig.

        Angenommen: $v_1, ..., v_{i-1}, v_i$ lin. abhängig.

        Dann: $v_i = \sum_{j=1}^{i-1} a_j v_j, a_j \in K (*)$

        Mult. mit $c_i$: 

        $c_i v_i = \sum_{j=1}^{i-1} c_i a_j v_j$ (1)

        \par \medskip

        Andererseits: Wende $\alpha$ auf $(*)$ an.

        \par \medskip

        $c_i v_i = \alpha(v_i) = \sum_{j=1}^{i-1} a_j \alpha(v_j) = \sum_{j=1}^{i-1} a_jc_jv_j$ (2)

        \par \medskip
        
        Subtr. (1) von (2):

        \begin{align*}
          0 & = \sum_{j=1}^{i-1} (a_jc_j - c_i a_j) v_j  \\
            & = \sum_{j=1}^{i-1} a_j (c_j- c_i)v_j 
        \end{align*}

        $\Rightarrow$ $a_j(c_j-c_i) = 0$ für $j=1, ..., i-1$

        \par \medskip

        \par \medskip

        Nach Voraus. ist $c_j - c_i \neq 0$ für alle $j=1, ..., i-1$

        $\Rightarrow a_j = 0$ für $j=1, ..., i-1$

        $\Rightarrow v_i = \sigma$ Widerspruch!

        \section{Definition}

        $\alpha : V \rightarrow V$ linear.

        $\alpha$ heißt \underline{diagonalisierbar}, falls $V$ eine Basis $\mathcal{B}$ aus Eigenvektoren von $\alpha$ besitzt, d.h.:

        \begin{center}
          $A_{\alpha}^{\mathcal{B}}$ ist Diagonalmatrix
        \end{center}

        \section{Satz}

        $dim_K(V) = n, \alpha : V \rightarrow V$ linear.

        Hat $\alpha$ n \underline{verschiedene} Eigenwerte, so ist $\alpha$ diagonalisierbar (Hinreichend, nicht notwendig, z.B. $\alpha = id_v$ EW 1 mit Vielfachheit n, diagonalisierbar).

        \section{Beispiel}

        $A_{\alpha}^{\mathcal{B}} = \begin{pmatrix}1 & 1 \\ 0 & 1 \end{pmatrix}$

        $\alpha$ hat EW 1 mit Vielfachheit 2

        \par \medskip

        $\alpha$ ist nicht diagonalisierbar, denn sonst ex. Basis $\mathcal{B'}$ mit $A_{\alpha}^{\mathcal{B}} = \begin{pmatrix}1 & 0 \\ 0 & 1 \end{pmatrix} \rightarrow \alpha = id_v$ Widerspruch !

        \par \medskip

        \par \medskip

        Also:

        Zur Diagonalisierbarkeit reicht es nicht, dass $\alpha$ n Eigenwete (mit Vielfachheit) besitzt.

        \section{Bemerkung}

        Sei $\alpha : V \rightarrow V$ linear, $dim_K(V) = n$.

        Besitze $\alpha$ n Eigenwerte (mit Vielfachheit), d.h. $det(t \cdot E_n -A) = (t-c_1)^{m_1} \cdot ... \cdot (t-c_r)^{m_r}$

        \par \medskip

        Ist $V_i$ Eigenraum von $\alpha$ zu $c_i$, so kann man zeigen: 

        \begin{center}
          $dim(V_i) \le m_i$
        \end{center}

        Es gilt: $\alpha$ ist diagonalisierbar $\Leftrightarrow dim(V_i) = m_i, i =1, ..., r$

        \section{Definition}

        $A \in \mathcal{M}_n(K)$ heißt \underline{diagonalisierbar}, falls

        \begin{center}
          $\alpha_A : \begin{cases}K^n \rightarrow K^n \\ x \mapsto A \cdot x \end{cases}$  
        \end{center}

        diagonalisierbar ist.

        \section{Satz}

        \begin{enumerate}[label=(\alph*)]
          \item $A \in \mathcal{M}_n(K)$ ist diagonalisierbar $\Leftrightarrow$ es ex. invertierbare Matrix $S \in \mathcal{M}_n(K)$ mit $S^{-1} \cdot A \cdot S$ Diagonalmatrix.

          \item Hat A $n$ verschiedene Eigenwerte, so ist A diagonalisierbar.
        \end{enumerate}

        \subsection{Beweis}

        \begin{enumerate}[label=(\alph*)]
          \item $\mathcal{B}$ kanonische Basis von $K^n$.

          $A = A_{\alpha_A}^{\mathcal{B}} \cdot A$ diagonalisierbar, so existiert Basis $\mathcal{B'}$ von $K^n$ mit $A_{\alpha_A}^{\mathcal{B'}}$ Diagonalgestalt hat.

          Setze S = $S_{\mathcal{B, B'}}$, dann nach 5.21:

          \begin{center}
            $A_{\alpha_A}^{\mathcal{B'}} = S^{-1} \cdot A_{\alpha_A}^{\mathcal{B}} \cdot S$
          \end{center}

          Umgekehrt analog, da jede inverse Matrix S Wechsel von B zu anderer Basis beschreibt.

          \item Folgt aus 7.16
        \end{enumerate}

        \chapter{Vektorräume mit Skalarprodukt}

        Jetzt: \underline{$K = \mathbb{R}$}.

        $\mathbb{R}^2$: Länge von $v \in \mathbb{R}^2$, $v \leftrightarrow \begin{pmatrix}x \\ y \end{pmatrix}$

        \par \medskip

        $||v|| = + \sqrt{x^2 + y^2}$ (länge)

        \par \medskip

        \underline{\textbf{Abstand}} zwischen

        \begin{center}
          $v \leftrightarrow \begin{pmatrix}x_1 \\ y_1 \end{pmatrix}$

          \par \medskip

          $w \leftrightarrow \begin{pmatrix}x_2 \\ y_2 \end{pmatrix}$
        \end{center}

        entspricht: $d(v,w) := || v - w || = + \sqrt{(x_1 - x_2)^2 + (y_1 - y_2)^2}$

        \par \medskip

        \underline{\textbf{Winkel:}}

        Pythagoras: 

        \begin{align*}
          ||v-w||^2 & = ||v||^2 \cdot sin^2(\varphi) + (||w|| - ||v|| \cdot cos(\varphi))1^2 \\
                    & = ||v||^2 + ||w||^2 - 2 ||v|| \cdot ||w|| \cdot cos(\varphi) \text{ (Kosinussatz)} \\
          ||v-w||^2 & = (x_1-x_2)^2 + (y_1-y_2)^2 \\
                    & = x_1^2 + x_2^2 - 2x_1x_2 + y_1^2 + y_2^2 - 2y_1y_2 \\
                    & = ||v||^2 + ||w||^2 - 2(x_1x_2 + y_1y_2)
        \end{align*}

        Damit folgt:

        \begin{equation}
          \underbrace{x_1x_2 + y_1y_2}_{Skalarprodukt} = ||v|| \cdot ||w|| \cdot cos(\varphi)
        \end{equation}

        \section{Definition Skalarprodukt}

        Seien $v = \begin{pmatrix}u_1 \\ ... \\ u_n \end{pmatrix}, w = \begin{pmatrix}z_1 \\ ... \\ z_n \end{pmatrix} \in \mathbb{R}^n$.

        \par \medskip

        Das \underline{(Standard-)Skalarprodukt} von $v$ und $w$:

        \begin{center}
          $(v|w) :=u_1z_1 + u_2z_2 + ... + u_nz_n \in \mathbb{R}$
        \end{center}

        (Skalarprodukt zweier Vektoren ist reelle Zahl!)

        \par \medskip

        Es gilt:

        \begin{itemize}
          \item[(1)] $(v|v) \ge 0$

          $(v|v) = 0 \Leftrightarrow v = \sigma$

          (positiv definiert)

          \item[(2)] $(v|w) = (w|v)$

          (Symmetrie)

          \item[(3)] $(v|w_1 + w_2) = (v | w_1) + (v | w_2)$

          $(v|a \cdot w) = a \cdot (v|w)$

          (Linearität im 2. Argument)

          Analog: Linearität im 1. Argument.

        \end{itemize}

        $e_1, ..., e_n$ kanonische Basis.

        \begin{equation}
         (e_i | e_j) = \begin{cases}0, \text{ für $i \neq j$} \\ 1, \text{ für i = j} \end{cases}
        \end{equation} 

        \section{Definition}

        V $\mathbb{R}$-Vektorraum

        Abbildung 

        \begin{equation*}
          (.|.) : \begin{cases} V \times V \rightarrow \mathbb{R} \\ (v,w) \mapsto (v | w) \in \mathbb{R} \end{cases}
        \end{equation*}

        heißt \underline{Skalarprodukt} auf V, falls sie die Eigenschaften (1)-(3) aus 8.1 erfüllt (mit V statt $\mathbb{R}^n$).

        \par \medskip

        V heißt dann \underline{Euklidischer Vektorraum} (oder \underline{Skalarproduktraum}).

        Dann gilt auch:

        \begin{itemize}
          \item[(4)] $(v_1+v_2|w) = (v_1|w) + (v_2|w)$

          $(av | w) = a (v|w)$

          (Linearität im 1. Argument)

          (Folgt aus (2) und (3))

        \end{itemize}

        Es folgt auch:

        \begin{center}
          $(\sigma | w) = 0 = (v | \sigma)$
        \end{center}

        Weil $(\sigma | w ) = (0 \cdot \sigma | w) \underbrace{=}_{(4)} 0 \cdot (\sigma | w) = 0$

        \section{Beispiel}

        \begin{enumerate}[label=(\alph*)]
          \item Standard-Skalarprodukt auf $\mathbb{R}^n$ ist Skalarprodukt im Sinne von 8.2

          \item V n-dim. $\mathbb{R}$-Vektorraum.

          $v_1, ..., v_n$ Basis von V.

          $v = \sum_{i=1}^n a_iv_i$, w = $\sum_{i=1}^n b_iv_i$

          Def. $(v|w) = \sum_{i=1}^n a_i \cdot b_i$ ist Skalarprodukt.

          Das Standard-Skalarprodukt auf $\mathbb{R}^n$ entsteht auf diese Weise, wenn man für $v_1, ..., v_n$ die kan. Basis nimmt.

          \item V $\mathbb{R}$-Vektorraum.

          $C[a,b]$ der stetigen Funktionen auf $[a,b]$ (mit Werten in $\mathbb{R}$).

          \par \medskip

          $f,g \in V$

          Def. $(f|g) := \int_{a}^b f(g) \cdot g(x) \ dx \in \mathbb{R}$

          Skalarprodukt.
        \end{enumerate}

        \section{Satz (Cauchy-Schwarz'sche Ungleichung)}

        V Euklidischer Vektorraum. Dann:

        \begin{center}
          $(v|w)^2 \le (v|v) \cdot (w|w)$ für alle $v,w \in V$
        \end{center}

        Gleichheit gilt genau dann, wenn $v$ und $w$ linear abhängig sind.

        \subsection{Beweis}

        Ist $w = \sigma$, so auf beiden Seiten 0 (und $v, w = \sigma$ sind lin. abhängig).

        \par \medskip

        Sei $w \neq \sigma$.

        Setze $a := \frac{(v|w)}{\underbrace{(w|w)}_{> 0}} \in \mathbb{R}$

        Bilde:

        \begin{align*}
          0 \le (v-a \cdot w | v - a \cdot w) & = (v-a \cdot w | v) - a \cdot (v- a \cdot w) |w) \\
                                              & = (v|v) - a \cdot (w|v) - a(v|w) + a^2 \cdot (w|w)  \\
                                              & = (v|v) - 2 \cdot \frac{(v|w)^2}{(w|w)} + \frac{(v|w)^2}{(w|w)} \\
                                              & = (v|v) - \frac{(v|w)^2}{(w|w)}
        \end{align*}
        
        Daraus folgt:

        \begin{align*}
          \frac{(v|w)^2}{(w|w)} \le (v|v) \\
          (v|w)^2 \le (v|v) \cdot (w|w)
        \end{align*}

        Gleichheit $\Leftrightarrow$ $(v-a \cdot w) | v - a \cdot w) = 0 \Leftrightarrow v = a \cdot w$

        \section{Definition}

        V Euklidischer Vektorraum.

        \begin{enumerate}[label=(\alph*)]
          \item Für $v \in V$ ist $||v|| := + \underbrace{\sqrt{(v|v)}}_{\ge 0}$ \underline{(Euklidische) Norm} von $v.$ ('Länge' von v)

          \item $v,w \in V$

          $d(v,w) := ||v-w||$, \underline{(Euklidischer) Abstand} von $v$ und $w$.

          (8.4 bedeutet dann: $|(v|w)| \le ||v|| \cdot ||w||$)
        \end{enumerate}

        \section{Beispiel}

        \begin{enumerate}[label=(\alph*)]
          \item Standard-Skalarprodukt auf $\mathbb{R}^n$:

          \begin{align*}
            v = \begin{pmatrix} x_1 \\ ... \\ x_n \end{pmatrix}, w = \begin{pmatrix}y_1 \\ ... \\ y_n \end{pmatrix} \\
            ||v|| = + \sqrt{\sum_{i=1}^n x_i^2} \\
            d(v,w) = + \sqrt{\sum_{i=1}^n (x_i - y_i)^2}
          \end{align*}

          \item $V = C[a,b], a,b \in \mathbb{R}, a < b$

          $(f|g) = \int_a^b f(x) \cdot g(x) \ dx$

          $||f|| = \sqrt{\int_a^b f^2(x) \ dx}$
        \end{enumerate}

        \section{Satz (Eigenschaften der Norm)}

        V Euklidischer VR, Norm $||.||$. Dann:

        \begin{enumerate}[label=(\alph*)]
          \item $||v|| \ge 0$, $||v|| = 0 \Leftrightarrow v = \sigma$ (Definiertheit)

          \item $||a \cdot v|| = |a| \cdot ||v||$ (absolute Homogenität)

          \item $||v + w|| \le ||v|| + ||w||$ (Dreiecksungleichung)

          \item $||v+w||^2 = ||v||^2 + ||w||^2 + 2 (v|w)$

          \item $||v+w||^2 + ||v-w||^2 = 2 (||v||^2 + ||w||^2)$ (Parallelogrammgleichung)
        \end{enumerate}

        \subsection{Beweis}

        (a) - (b) $\checkmark$

        \par \medskip

        (c)-(d):

        \begin{align*}
        ||v+w||^2 &= (v+w|v+w) = (v|v) + (w|w) + 2(v|w) \text{ (also (d))} \\
                  & \le (v|v) + (w|w) + 2 \cdot \sqrt{(v|v) \cdot (w|w)} = (\sqrt{(v|v)} + \sqrt{(w|w)})^2 = (||v|| + ||w||)^2 \rightarrow (c)
        \end{align*}

        $\checkmark$

        (e) folgt aus (d)

        \section{Bemerkung}

        Jede Abb. $\mathbb{R}$-VR in $\mathbb{R}$

        \begin{center}
          $||.|| : \begin{cases} V \rightarrow \mathbb{R}, \text{ die (a)-(c) erfüllt} \\ v \mapsto ||v|| \end{cases}$
        \end{center}

        heißt \underline{Norm} auf V. 

        \par \medskip

        Es gibt Normen, die nicht von Skalarprodukt herkommen, zb. in $\mathbb{R}^n$:

        \begin{center}
          $||\begin{pmatrix}x_1 \\ ... \\ x_n \end{pmatrix} ||_{max} := max \{|x_i| : i=1,...,n \}$
        \end{center}

        \section{Definition}

        V Euklidischer VR.

        \begin{enumerate}[label=(\alph*)]
          \item $v, w \in V$, $v \neq \sigma, w \neq \sigma$

          Nach 8.4 gilt:

          \begin{equation}
           -1 \le \frac{|(v|w)|}{||v|| \cdot ||w||} \le 1
          \end{equation}

          Dann ex. genau ein $\varphi \in [0, \pi]$ mit:

          \begin{equation}
            \frac{(v|w)}{||v|| \cdot ||w||} = cos(\varphi)
          \end{equation}

          Das heißt:

          \begin{equation} 
            (v|w) = ||v|| \cdot ||w|| \cdot cos(\varphi)
          \end{equation}

          $\varphi$ heißt Winkel zwischen v,w. ($v \neq \sigma, w \neq \sigma$)

          (kein orientierter Winkel, kleinerer der beiden möglich)

          \item v,w heißen \underline{orthogonal} (senkrecht), falls $(v|w) = 0$.

          Falls $v \neq \sigma$ und $w \neq \sigma$, so  heißt das: 

          \begin{equation}
            cos(\varphi) = 0 \Leftrightarrow \varphi = \frac{\pi}{2}
          \end{equation}

          ($\sigma$ ist orthogonal zu allen Vektoren)

          \item $M \subseteq V$

          $M^{\perp} := \{w \in V : (v|w) = 0$ für alle $v \in M \}$

          \underline{Orthogonalraum} zu M. Unterraum von V (selbst wenn M kein Unterraum ist)

          \par \medskip

          $\{e_1, e_2\}^{\perp} = \left \langle e_3 \right \rangle$

          $\{\sigma \}^{\perp} = V$

          $V^{\perp} = \{\sigma\}$

          ($v \in V^{\perp} \Rightarrow (v|v) = 0 \Rightarrow v = \sigma$)
        \end{enumerate}

        \section{Bemerkung}

        Sind $v,w$ orthogonal, so ist 

        \begin{equation}
          ||v + w||^2 = ||v||^2 + ||w||^2
        \end{equation}

        (8.7.d) 

        \section{Beispiel}

        \begin{enumerate}[label=(\alph*)]
          \item $\mathbb{R}^n$, Standard-Skalarprodukt.

          $(e_i | e_j) = 0$ für $i \neq j$

          $||e_i|| = 1$

          \item $\mathbb{R}^3$, Standard-Skalarprodukt.

          $v = \begin{pmatrix}-1 \\ 2 \\ 3 \end{pmatrix}, w = \begin{pmatrix}2 \\ 2 \\ 4 \end{pmatrix}$

          \par \medskip

          $||v|| = \sqrt{6}, \ ||w|| = \sqrt{24}$

          Für den Winkel folgt:

          \[ cos(\varphi) = \frac{(v|w)}{||v|| \cdot ||w||} = \frac{6}{\sqrt{6} \cdot \sqrt{24}} = \frac{1}{2} \Rightarrow \varphi = \frac{\pi}{3} \]

          \item $\mathbb{R}^2$, Standard-Skalarprod.

          $v = \begin{pmatrix}x_1 \\ x_2 \end{pmatrix} \neq \sigma$

          $\{v\}^{\perp} = \left \langle \begin{pmatrix}x_2 \\ x_1 \end{pmatrix} \right \rangle$
        \end{enumerate} 

        \section{Definition}

        V Euklidischer VR. $M \subseteq V$.

        \begin{enumerate}[label=(\alph*)]
          \item M heißt \underline{Orthonormalsystem}, falls \[ ||v|| = 1 \] für alle $v \in M$ und \[ (v|w) = 0 \] für alle $v,w \in M, v \neq w$

          \item Ist V endl. dim., so heißt M \underline{Orthonomalbasis} (ONB) von V, falls M Orthonomalsystem und Basis von V.
        \end{enumerate}

        Beachte: $v \neq \sigma$ \[ v' = \frac{1}{||v||}  v \in V \]

        Damit ergibt sich:

        \[ ||v‘|| = || \frac{1}{||v||} \cdot ||v|| = \frac{1}{||v||} \cdot ||v|| = 1 \]

        \underline{Normierung}

        \section{Bemerkung}

        Ist $(v_1, ..., v_n)$ ONB, $v \in V, v = \sum_{i=1}^n c_iv_i, c_i \in \mathbb{R}$

        \begin{align*}
          (v|v) & = (\sum_{i=1}^n c_ic_j | \sum_{j=1}^n c_jv_j) \\
                & = \sum_{i,j=1}^n c_ic_j \cdot (v_i|v_j)  \\
                & = \sum_{i=1}^n c_i^2 \cdot (v_i|v_i) \\
                & = \sum_{i=1}^n c_i^2 \\
          ||v|| & = \sqrt{\sum_{i=1}^n c_i^2}
        \end{align*}

        \section{Satz}

        \begin{enumerate}[label=(\alph*)]
          \item Ein Orthonormalsystem ist linear unabhängig.

          \item Ist M = $\{v_1, ..., v_n \}$ ein Orthonormalsystem, $v \in V$, so ist:

          \[ v - \sum_{i=1}^n (v|v_i) \cdot v_i \in M^{\perp} \]
        \end{enumerate}

        \subsection{Beweis}

        \begin{enumerate}[label=(\alph*)]
          \item Sei $\{v_1, ..., v_m\}$ endliche Teilmenge von M.

          Zu zeigen: $\{v_1, ..., v_m \}$ linear unabhängig. 

          Ist $\sum_{i=1}^m c_i v_i = \sigma$, so:

          \[ 0 = (\sum_{i=1}^m c_i v_i | v_j) = \sum_{i=1}^m c_i (v_i | v_j) = c_j (v_j | v_j) = c_j \]

          $c_j = 0$ für j = 1, ..., m 


          \item $(v_j | v - \sum_{i=1}^n (v | v_i) \cdot v_i) = (v | v_j) - \sum_{i=1}^m (v | v_i) \cdot \underbrace{(v_j | v_i)}_{= 0 \text{ für i $\neq j$}}$

          $ = (v | v_j) - (v | v_j) \cdot \underbrace{(v_j | v_j)}_{\le 1} = 0$
        \end{enumerate}

        \section{Satz (Gram-Schmidt'sches Orthonormalisierungsverfahren)}

        Sei $M = \{w_1, ..., w_n \}$ lin. unabhängige Menge im Eukl. VR V. Dann gibt es Orthonormalsystem $\{v_1, ..., v_m \}$ mit $\left \langle w_1, ..., w_i \right \rangle = \left \langle v_1, ..., v_i \right \rangle$ für alle $i=1,...,m$.

        Insbesonderes enthält V eine ONB.

        \subsection{Beweis}

        $w_1 \neq \sigma$. Setze \[v_1 = \frac{1}{||w_1||} \cdot w_1\].

        $||v_1|| = 1, \left \langle w_1 \right \rangle = \left \langle v_1 \right \rangle$

        \par \medskip

        Sei schon Orthonormalsystem $\{v_1, ..., v_i \}$ konstruiert mit $\left \langle w_1, ..., w_j \right \rangle = \left \langle v_1, ..., v_j \right \rangle$ für alle $j = 1, ..., i$ (i $<$ m)

        \par \medskip

        Setze $v_{i+1}' = w_{i+1} - \sum_{j=1}^i (v_j | w_{i+1}) \cdot v_j$ $\rightarrow$ 8.14.b): $(v_{i+1}' | v_j) = 0$ für $j = 1, ..., i$.

        \par \medskip

        Da $w_{i+1} \neq \left \langle w_1, ..., w_i \right \rangle$ = $\left \langle v_1, ..., v_i \right \rangle$, ist $v_{i+1}' \neq \sigma$.


        \par \medskip

        Setze $v_{i+1} = \frac{1}{||v_{i+1}'||} \cdot v_{i+1}'$, $||v_{i+1}|| = 1, (v_j | v_{i+1}) = 0, \ j=1,...i$

        \par \medskip

        Es gilt:

        \[ \left \langle v_1, ..., v_i, v_{i+1} \right \rangle = \left \langle v_1, ..., v_i, v_{i+1} \right \rangle = \left \langle v_1, ..., v_i, w_{i+1} \right \rangle = \left \langle w_1, ..., w_i, w_{i+1} \right \rangle \]

        \section{Beispiel}

        \begin{enumerate}[label=(\alph*)]
          \item $e_1, ..., e_n$ ist ONB des $\mathbb{R}^n$ bezgl. Standard-Skalarprodukt.

          \item V = $\mathbb{R}^3$ mit Standard-Skalarprodukt.

          \[ w_1 = \begin{pmatrix}1 \\ 1 \\ 1 \end{pmatrix}, w_2 = \begin{pmatrix}1 \\ 1 \\ 2 \end{pmatrix}, w_3 = \begin{pmatrix}1 \\ 0 \\ 0 \end{pmatrix} \]

          linear unabhängig.

          Gram-Schmidt: ONB $\{v_1, v_2, v_3 \}$

          $\left \langle v_1 \right \rangle = \left \langle w_1 \right \rangle, \left \langle v_1, v_2 \right \rangle = \left \langle w_1, w_2 \right \rangle, \left \langle v_1, v_2, v_3 \right \rangle = \mathbb{R}^3$

          \par \medskip

          $v_1 = \frac{1}{\sqrt{3}} \cdot w_1 = \frac{1}{\sqrt{3}} \cdot \begin{pmatrix}1 \\ 1 \\ 1 \end{pmatrix}$

          \par \medskip

          $v'_2 = w_2 - (v_1 | w_2) \cdot v_1 = \begin{pmatrix}1 \\ 1 \\ 2 \end{pmatrix} - \frac{1}{3} \cdot (\begin{pmatrix}1 \\ 1 \\ 1 \end{pmatrix} | \begin{pmatrix}1 \\ 1 \\ 2 \end{pmatrix}) \cdot \begin{pmatrix}1 \\ 1 \\ 1 \end{pmatrix} = \begin{pmatrix}- \frac{1}{3} \\ - \frac{1}{3} \\ \frac{2}{3}\end{pmatrix}$

          \par \medskip

          $||v'_2|| = \frac{\sqrt{6}}{3}, \ v_2 = \frac{1}{\sqrt{6}} \cdot \begin{pmatrix}-1 \\ -1 \\ 2 \end{pmatrix}$

          \par \medskip

          $v'_3 = w_3 - (v_1|w_3) \cdot v_1 - (v_2 | w_3) \cdot v_2 = ... = \begin{pmatrix}\frac{1}{2} \\ - \frac{1}{2} \\ 0 \end{pmatrix}$, $||v'_3 = \frac{\sqrt{2}}{2}$, $v_3 = \frac{1}{\sqrt{2}} \cdot \begin{pmatrix}1 \\ -1 \\ 0 \end{pmatrix}$
        \end{enumerate}

        \section{Satz}

        V endl. dim. Eukld. VR, U unterraum von V.

        \begin{enumerate}[label=(\alph*)]
          \item $V = U \oplus U^{\perp}$ (d.h. $U \cap U^{\perp} = \{\emptyset\}$, $U + U^{\perp} = V$)

          Insb: $dim(U) + dim(U^{\perp}) = dim(V)$

          \item $(U^{\perp})^{\perp} = U$
        \end{enumerate}

        \subsection{Beweis}

        \begin{enumerate}[label=(\alph*)]
          \item Basis Ergänzung + Gram-Schmidt

          \item folgta us a)
        \end{enumerate}

        \section{Definition}

        $V = \begin{pmatrix}x_1 \\ x_2 \\ x_3 \end{pmatrix}, W = \begin{pmatrix}y_1 \\ y_2 \\ y_3 \end{pmatrix} \in \mathbb{R}^3$

        \par \medskip

        \underline{Vektorprodukt} (Kreuzprodukt) von v und w:

        \[ V \times W := \begin{pmatrix}x_2y_3 - x_3y_2 \\ x_3y_1 - x_1 y_3 \\ x_1y_2 - x_2y_1 \end{pmatrix} \]

        \section{Satz}

        \begin{enumerate}[label=(\alph*)]
          \item $(v \times w | v) = (v \times w | w) = 0$, d.h. $v \times w$ ist orthogonal zu $v$ und $w$.

          \item $v \times w = -w \times v$

          \item $u \times (v+w) = (u \times v) + (u \times w)$

          $u \times (a \cdot v) = a \cdot (u \times v)$, $u,v,w \in \mathbb{R}^3, a \in \mathbb{R}$

          \par \medskip

          Ebenso in der ersten Komponente.

          \item $v,w$ linear unabhängig $\Leftrightarrow$ $v \times w = \sigma$.

          \item $v,w \neq \sigma$. $\varphi \in [0, \pi]$ Winkel zwischen $v$ und $w$.

          \par \medskip

          \begin{align*}
            ||v \times w|| & = ||v|| \cdot ||w|| \cdot sin(\varphi) \\
                           & = \text{ Flächeninhalt des von v und w aufgespannten Parallelogramms}
          \end{align*}
        \end{enumerate}

        \subsection{Beweis}

        Nachrechnen. 

        \section{Bemerkung}

        $v, w, v \times w$ bilden sogenanntes \underline{Rechtssystem}.

        Faust der rechten Hand. Fingerspitzen von v nach w (kleinerer Winkel).

        Daumen zeigt in Richtung $v \times w$.

        \section{Beispiel}

        $v = \begin{pmatrix}1 \\ 2 \\ 0 \end{pmatrix}, w = \begin{pmatrix}1 \\ 1 \\ 1 \end{pmatrix} \in \mathbb{R}^3$.

        \par \medskip

        Best. $\left \langle v, w \right \rangle^{\perp}$.

        $v, w$ lin. unabhängig, d.h. $\left \langle v,w \right \rangle = 2$

        $\Rightarrow dim \left \langle v,w \right \rangle^{\perp} = 3-2 = 1$.

        \par \medskip

        $\left \langle v \times w \right \rangle = \left \langle v,w \right \rangle^{\perp}$

        Damit folgt:

        \[ v \times w = \begin{pmatrix}2 \\ -1 \\ -1 \end{pmatrix} \]

        \[ || v \times w || = \sqrt{4 + 1+1} = \sqrt{6} \]

        \chapter{Orthogonale Abb., symmetrische Abb., Kongruenzabbildungen}

        \section{Definition Orthogonale Abbildungen}

        V Eukl. VR mit Skalarprodukt $(.|.)$.

        $\alpha: V \rightarrow V$ lineare Abb.

        \par \medskip

        $\alpha$ heißt \underline{orthogonale Abbildung} $\Leftrightarrow (\alpha(v) | \alpha(w)) = (v | w)$ für alle $v,w \in V$

        \section{Folgerungen}

        \begin{enumerate}[label=(\alph*)]
          \item Orthogonale Abb. sind \underline{längentreu}, d.h. \[ ||\alpha(v)|| = ||v|| \] für alle $v \in V$.

          (da $||v|| = \sqrt{(v|v)}$ )

          \item Ort. Abb. sind winkeltreu, da \[ cos(\varphi) = \frac{(v|w)}{||v|| \cdot ||w||}, v,w \neq \sigma \]

          \item Orth. Abb. auf endlich dim. Eukl. Räumen sind bijektiv, da nach a) $ker(\alpha) = \{\sigma\}$, also $\alpha$ injektiv, also bijektiv, da $dim(V) < \infty$.

          \item V endl. dim

          $\alpha$ orth. $\Rightarrow \alpha^{-1}$ orthogonal.

          ($u,v \in V$. Es ex. $x,y \in V$ mit $\alpha(x) = u$, $\alpha(y)$ = w, d.h. $\alpha^{-1}(u) = x, \alpha^{-1}(v) = y$.

          $(u|v) = (\alpha(x) | \alpha(y)) = (x|y) = (\alpha^{-1}(u)|\alpha^{-1}(w)$)

          \item $\alpha, \beta$ orthogonal, so auch $\alpha \circ \beta$.

          \item (d) + (e) besagen, dass die Menge der orth. Abb. auf V bzgl. $\circ$ eine Gruppe ist. (V endl. dim.)
        \end{enumerate}

        \section{Beispiel}

        \begin{enumerate}[label=(\alph*)]
          \item Drehungen um $\sigma$ im $\mathbb{R}^2$ sind orth. Abb. (bzgl. des Standard-Skalarprodukts)

          \item Spiegelungen $\varrho$ in $\mathbb{R}^2$ an Achse durch $\sigma$ sind orth.

          $v_1$ Richtungsvektor der Achse, $||v_1|| = 1$, $\mathcal{B} = (v_1, v_2)$ ONB (Gram-Schmidt)

          \[ A_{\varrho}^{\mathcal{B}} = \begin{pmatrix}1 & 0 \\ 0 & -1 \end{pmatrix} \]
        \end{enumerate}

        \section{Satz (Charakterisierung orth. Abb.)}

        V endl. dim. Eukl. VR., $\mathcal{B} = (v_1, ..., v_n)$ ONB, $\alpha: V \rightarrow V$ linear, $A = A_{\alpha}^{\mathcal{B}}$. 

        Dann sind äquivalent:

        \begin{enumerate} 
          \item $\alpha$ ist orthogonale Abb.

          \item $A \cdot A^t = A^t \cdot A = E_n$

          (d.h. $A^t = A^{-1}$)

          \item $(\alpha(v_1), ..., \alpha(v_n))$ ist ONB

          \item $||\alpha(v)|| = ||v||$ für alle $v \in V$.

        \end{enumerate}

        \subsection{Beweis}

        \subsubsection*{1. $\Rightarrow$ 2.}

        $A = (a_{ij})_{i,j =1,...,n}$

        Es gilt:

        \begin{align*}
          \delta_{ij} & =  (v_i|v_j)  \\ & = (\alpha(v_i)|\alpha(v_j)) \\
                                  & = (\sum_{k=1}^n a_{ki} \cdot v_k | \sum_{l=1}^n a_{lj} \cdot v_l) \\
                                  & = \sum_{k,l=1}^n a_{ki} \cdot a_{lj} \cdot (v_k | v_l) \\
                                  & = \sum_{k=1}^n a_{ki} \cdot a_{kj} \leftarrow \text{ Eintrag $(i,j)$ von $A \cdot A^t$}
        \end{align*}

        $\Rightarrow A \cdot A^t = E_n$. (*)

        $\alpha$ orth. $\Rightarrow \alpha$ bijektiv $\rightarrow A$ invetierbar, d.h. $A^{-1}$ ex.

        \par \medskip

        Multipliziere (*) von links mit $A^{-1}$ von rechts mit A:

        \[ A^{-1} \cdot A \cdot A^t \cdot A = A^{-1} \cdot A = E_n \]

        \subsection*{2. $\Rightarrow$ 3.}

        $A \cdot A^t = E_n$.

        Dann wie in 1. $\Rightarrow$ 2.:

        \[ (\alpha(v_i) | \alpha(v_j)) = \delta{ij} \]

        $(\alpha(v_1), ..., \alpha(v_n))$ ONB.

        \subsection*{3. $\Rightarrow$ 4.}

        \[ v = \sum_{i=1}^n c_i v_i \]

        \[ \alpha(v) = \sum_{i=1}^n c_i \cdot \alpha(v_i) \]

        \[ ||v||^2 = (v|v) = \sum_{i=1}^n c_i^2 = || \alpha(v) ||^2\]

        \subsection*{4. $\Rightarrow$ 1.}

        8.7.d) $(v|w) = \frac{1}{2} (||v + w||^2 - ||v||^2 - ||w||^2)$

        \par \medskip

        Behauptung folgt. 

        \section{Definition}

        Eine $n \times n$ - Matrix A über $\mathbb{R}$ heißt \underline{orthogonal}, falls $A \cdot A^t = A^t \cdot A = E_n$.

        \par \medskip

        D.h. $z_1, ..., z_n$ von A:

        \begin{align*} 
            (z_i^t | z_j^t) & = z_i \cdot z_j^t \\
                            & = z_i \cdot \underbrace{s_j}_{\text{j-te Spalte von $A^t$}} \\
                            & = \delta_{ij}
        \end{align*}

        Die Spalten von A bilden Orthonormalbasis von $\mathbb{R}^n$.

        Analog für die Zeilen.

        \section{Korollar}

        $\alpha$ orth. Abb. auf endl. dim.  Eukld. VR. V, $\mathcal{B}$ ONB von V, $A = A_{\alpha}^{\mathcal{B}}$

        \begin{enumerate}[label=(\alph*)]
          \item $det(\alpha) = det(A) = \pm 1$

          \item $\alpha$ hat höchstens die Eigenwerte 1 oder -1 (in $\mathbb{R}$)

        \end{enumerate}

        \subsection{Beweis}

        \begin{enumerate}[label=(\alph*)]
          \item 1 = $det(E_n) = det(A \cdot A^t) = det(A) \cdot det(A^t) = detA)^2$

          \item $c$ Eigenwert von $\alpha, v$ zugehöriger Eigenvektor. $||v|| = ||\alpha(v) = || c \cdot v|| = |c| \cdot ||v|| \Rightarrow |c| = 1$

        \end{enumerate}

        \section{Satz}

        Sei $\alpha$ orthogonale Abb. auf $\mathbb{R}^n$, $n \le 3$.

        \begin{enumerate}[label=(\alph*)] 
          \item $n = 1: \alpha = id$ oder $\alpha = - id$

          \item $n = 2$: Sei $\mathcal{B}$ ONB von $\mathbb{R}^2$.

          \begin{enumerate}[label=(\roman*)]
            \item Ist $det(\alpha) = 1$, so ist \[ A_{\alpha}^{\mathcal{B}} = \begin{pmatrix}cos(\varphi) & - sin(\varphi) \\ sin(\varphi) & cos(\varphi) \end{pmatrix}\] für ein $\varphi \in [0, 2 \pi]$.

            $\alpha$ ist Drehung um Winkel $\varphi$.

            \item Ist $det(\alpha) = -1$, so ist \[ A_{\alpha}^{\mathcal{B}} = \begin{pmatrix}cos(\varphi) & sin(\varphi) \\ sin(\varphi) & - cos(\varphi) \end{pmatrix} \]

            für ein $\varphi \in [0, 2 \pi]$.

            Es gibt ONB $\mathcal{C} = (w_1, w_2)$ von V mit $A_{\alpha}^{\mathcal{C}} = \begin{pmatrix}1 & 0 \\ 0 & -1 \end{pmatrix}$

            \par \medskip

            Spiegelung an der Achse $\left \langle w_1 \right \rangle$.

            ($\mathcal{B} = (v_1, v_2)$, so ist der Winkel zwischen $v_1$ und $w_1$ $\frac{\varphi}{2}$)

          \end{enumerate}

          \item $n = 3$: 

          \begin{enumerate}[label=(\roman*)]
            \item Ist $det(\alpha) = 1$, so existiert ONB $\mathcal{C} = (v_1, v_2, v_3)$ mit \[ A_{\alpha}^{\mathcal{C}} = \begin{pmatrix}cos(\varphi) & - sin(\varphi) & 0 \\ sin(\varphi) & cos(\varphi) & 0 \\ 0 & 0 & 1 \end{pmatrix} \]

            $\alpha$ ist Drehung um $\varphi \in [0,2 \pi[$ parallel zur $\left \langle v_1, v_2 \right \rangle$-Ebene mit Drehachse $\left \langle v_3 \right \rangle$

            \item Ist $det(\alpha) = -1$, so existiert ONB $\mathcal{C} = (v_1, v_2,v_3)$ mit \[ A_{\alpha}^{\mathcal{C}} = \begin{pmatrix}cos(\varphi) & - sin(\varphi) & 0 \\ sin(\varphi) & cos(\varphi) & 0 \\ 0 & 0 & -1 \end{pmatrix} \]

            $\alpha$ ist Drehspiegelung.

            Drehung um Achse $\left \langle v_3 \right \rangle$, dann Spiegelung an Ebene $\left \langle v_1, v_2 \right \rangle$.
          \end{enumerate}

        \end{enumerate}

        \textbf{Bemerkung:} n = 2:

        Ist $\mathcal{B} = (v_1, v_2)$ positiv orientiert (d.h. bewegt man sich von $v_1$ nach $v_2$ entgegen Uhrzeigersinn, so ist Winkel $\frac{\pi}{2}$), so ist $\alpha$ Drehung gegen Uhrzeigersinn um $\varphi$. 

        \par \medskip

        Ist $\mathcal{B}$ negativ orientiert, so ist $\alpha$ Drehung im Uhrzeigersinn.

        \par \medskip

        (Spezialfälle in (c):

        (i) $\varphi = \pi$: $A_{\alpha}^\mathcal{C} = \begin{pmatrix}-1 & 0 & 0 \\ 0 & -1 & 0 \\ 0 & 0 & 1 \end{pmatrix}$ Achsenspiegelung an $\left \langle v_3 \right \rangle$

        \par \medskip

        (ii) $\varphi = 0 : A_{\alpha}^{\mathcal{C}} = \begin{pmatrix}1 & 0 & 0 \\ 0 & 1 & 0 \\ 0 & 0 & -1 \end{pmatrix}$ (Ebenen-)Spiegelung an $\left \langle v_1, v_2 \right \rangle$

        $\varphi = \pi: A_{\alpha}^{\mathcal{C}} = \begin{pmatrix}-1 & 0 & 0 \\ 0 & -1 & 0 \\ 0 & 0 & -1 \end{pmatrix}$ Punktspiegelung in 0.)


        \subsection{Beweis}

        \begin{enumerate}[label=(\alph*)]
          \item trivial $\checkmark$

          \item Untersuchen der $2 \times 2$ - Mattrizen A. $A \cdot A^t = A^t \cdot A = E_2$ (WHK 10.67)

          \item Für bel. n: Fischer, Lineare Algebra

        \end{enumerate}


        \section{Definition}

        \begin{enumerate}[label=(\alph*)]
          \item V K-VR, $b \in V$.

          Die Abb. $t_b \begin{cases}V \rightarrow V \\ v \mapsto v+b \end{cases}$ heißt \underline{Translation} um Vektor b

          (Nicht linear, falls $b \neq 0$)

          \item Komposition von lin. Abb. mit Translationen heißen \underline{affine Abbildungen}

          \par \medskip

          $\beta(v) = \alpha(v) + b,  \alpha$ lin. Abb. $V \rightarrow V, b \in V$ = ($t_b \circ \alpha) (v)$

          \item Ist V Eukl. VR, so ist \underline{Kongruenzabbildung} auf V eine affine Abb. der Form $t_b \circ \alpha$, wobei $\alpha$ orthogonale Abb. ist. (lassen den Abstand zwischen Vektoren (und Winkel) fest).

        \end{enumerate}

        \section{Bemerkung}

        \begin{enumerate}[label=(\alph*)]
          \item Affine Abb. bilden affine Unterräume wieder auf affine Unterräume ab:

          \[ \beta = t_b \circ \alpha, \ u+ \underset{\text{Unterraum von V}}{W} \]

          \begin{align*}
            \beta(u+W) & = t_b (\alpha(u+W)) \\
                       & = t_b (\alpha(u) + \alpha(W)) \\
                       & = b + \alpha(u) + \alpha(W) 
          \end{align*}

          \item Hinterinanderausführung von affinen (Kongruenz-)Abb. ist wieder affine  (Kongruenz-) Abb.:

          \begin{align*}
            ((t_b' \circ \alpha') \circ (t_b \circ \alpha))(v) & = (t_b' \circ \alpha') (\alpha(v) + b) \\
                                                               & = \underbrace{(\alpha' \circ \alpha)}_{\text{lin. Abb.}}(v) + \underbrace{\alpha'(b) +b'}_{\text{Translationsvektor}}
          \end{align*}

          \textbf{Beispiel:} $\beta$ Drehung um $c \in \mathbb{R}^2$ mit Winkel $\varphi$ (in $\mathbb{R}^2$)

        \begin{align*}
          \beta(v) & = t_c \circ \underset{\text{Dehung um 0 mit $\varphi$}}{\alpha} \circ t_{-c} \\
                   & = \alpha(v) + \alpha(-c) + c \\
                   & = \underset{lin.}{\alpha}(v) \underbrace{- \alpha(c) + c}_{\text{Translationsv.}} \\
                   & = (t_{-\alpha(c) +c} \circ \alpha)(v) \\
                   & = \alpha(v-c) + c
        \end{align*}

        \item Affine Abb. auf $n-dim$. Räumen lassen sich nicht durch $n \times n$- Matrizen beschreiben.

        Es gibt Beschreibung mit $(n+1) \times (n+1)$ - Matrizen:

        \par \medskip

        $\mathcal{B}$ Basis von V. $\mathcal{B} = (v_1, ..., v_n)$

        $\beta = t_b \circ \alpha$.

        \par \medskip

        $b = b_!v_1 + ... + b_nv_n$, $A = A_{\alpha}^{\mathcal{B}}$

        \[ \beta \rightarrow \begin{pmatrix}A & b_1 \\ ... & ... \\ ... & b_n \\ 0 & 1 \end{pmatrix} \]

        $v \in V, v = x_1v_1 + .... + x_n v_n$

        \[ v \rightarrow \begin{pmatrix}x_1 \\ ... \\ x_n \\ 1 \end{pmatrix} \]

        [BILDER VON JOHANNES.]

        \par \medskip


        Nachrechnen: Hintereinandersauführung von aff. Abb. entspricht Multiplikation der entsprechenden $(n+1) \times (n+1)$- Matrizen.

        \end{enumerate}

        \section{Lemma}

        $A \in \mathcal{M}_n(\mathbb{R}), x,y \in \mathbb{R}^n$ so gilt für das Standard-Skalarprodukt:

        \[ (Ax|y) = (x|A^t y) \]

        \subsection{Beweis}

        \begin{align*}
          (Ax |y )  & = (Ax)^t \cdot y \\
                    & = (x^t \cdot A^t) \cdot y \\
                    & = x^t \cdot (A^t \cdot y) \\
                    & = (x|A^t y)
        \end{align*}

        \section{Definition}

        V endl. dim. Eukl. VR. Lineare Abb. $\alpha : V \rightarrow V$ heißt \underline{symmetrisch} (selbstadjungiert), falls \[ (\alpha(v) | w) = (v | \alpha(w)) \] für alle $v,w \in V$.

        (z.B. alle $\alpha \cdot id_v$ sind symm.)

        \section{Bemerkung und Definition}

        V endl. dim. Eukl. VR., $\mathcal{B}$ ONB von V. $\alpha: V \rightarrow V$ linear, $A = A_{\alpha}^{\mathcal{B}}$. Dann: \[ \alpha \text{ symmetrisch} \Leftrightarrow A = A^t \]

        $n \times n$ - Matrizen A mit $A = A^t$ heißen \underline{symmetrisch}.

        \par \medskip

        Matrix ändert sich nicht bei Spiegelung (wenn sie symmetrisch ist) bei Spiegelung an Diagonalen.

        \subsection{Beweis}

        Folgt aus 9.10.

        \section{Satz über die Hauptachsentransformation}


        \begin{enumerate}[label=(\alph*)]
          \item Ist $\alpha: V \rightarrow V$ symm. Abb., so existiert \underline{ONB $\mathcal{B}$} von V aus Eigenvektoren zu $\alpha$, insbesondere ist $\alpha$ diagonalisierbar.

          \item Ist $A$ $n \times n$ - Matrix, $A$ symmetrisch, so existiert orthogonale Matrix $B$ mit $B^{-1} A B = B^t A B$ Diagonalmatrix.
        \end{enumerate}

        \subsection{Beweis}

        \begin{enumerate}[label=(\alph*)]
          \item WHK 10.75

          \item $\alpha(x) = A \cdot x$

          $A = A_{\alpha}^{\mathcal{C}}$ bzgl. kan. Basis $\mathcal{C}$. $\alpha$ symmetrisch nach 9.12.

          (a): $\exists$ ONB $\mathcal{B}$ mit $A_{\alpha}^{\mathcal{B}}$ Diagonalmatrix. 

          \par \medskip

          Wähle B als Basiswechselmatrix: $B^{-1} A B$ Diagonalmatrix. $\mathcal{B}, \mathcal{C}$ ONB $\Rightarrow$ B orthogonale Matrix (9.4), d.h. $B^{-1} = B^t$.
        \end{enumerate}

        \section{Bemerkung}

        Grund für die Bezeichnung von 9.13.

        \begin{enumerate}[label=(\alph*)]
          \item Eine \underline{Quadrik} im $\mathbb{R}^n$ besteht aus allen $\begin{pmatrix}x_1 \\ ... \\ x_n \end{pmatrix} \in \mathbb{R}^n$, die die Gleichung \[ \sum_{i,j=1; i \le j}^{n} b_{ij} x_i x_j + \sum_{i=1}^n b_i x_i + c = 0 (*)\]

          (quadr. Gleichung mit n Variablen $x_1, ..., x_n$ mit Koeff. $b_{ij}, b_j, c \in \mathbb{R}$)

          \par \medskip

          Setze $a_{ij} = a_{ji} = \frac{b_{ij}}{2}$ für $i < j$.

          Dann lässt sich (*) schreiben: \[ \sum_{i,j = 1}^n a_{ij} x_ix_j + \sum_{i=1}^n b_ix_i + c = 0(**)\]

          $A = (a_{ij})$ ist symmetrisch.

          \par \medskip

          $\alpha(x) = A \cdot x $ symm. Abbildung ($A = A_{\alpha}^{\mathcal{C}}$ bzgl. kan. Basis)

          \par \medskip

          $(**)$ $\Leftrightarrow (\alpha(v) | v) + (b|v) + c = 0$, $b = (b_1, ..., b_n)^t, v = (x_1, ..., x_n)^t (***)$

          \par \medskip

          9.13. besagt: $\exists$ ONB $\mathcal{B} = (v_1, ..., v_n)$, so dass $A_{\alpha}^{\mathcal{B}}$ Diagonalmatrix.

          \par \medskip

          Sei $v = \sum_{i=1}^n x_i' v_i$, so liest sich $(***)$ \[ \sum_{i=1}^n c_i x_i'^2 + \sum_{i=1}^n b_ix_i' + c = 0 \]

          $v_i$ sind die \underline{Hauptachsen} der Quadrik.

          \item n = 2: Quadriken sind \underline{Kegelschnitte}.

          $(****)$, Striche weglassen.

          Fall: $b_1 = b_2 = 0, c \neq 0$.

          O.B.d.A: \[ c_1x_1^2  + c_2x_2^2 = 1 \]

          $c_1 > 0$.

          \par \medskip

          \begin{itemize}
            \item $c_1, c_2 > 0$:  $c_1 = \frac{1}{a^2}$, $c_2 = \frac{1}{b^2}$ \[ \frac{x_1^2}{a^2} + \frac{x_2^2}{b^2} = 1, a,b > 0 \]

           \underline{Ellipse}.

           \item $c_1 > 0, c_2 < 0$

           $c_1 = \frac{1}{a^2}, c_2 = - \frac{1}{b^2}, a,b > 0$ \[ \frac{x_1^2}{a^2} - \frac{x_2^2}{b^2} = 1 \]

           \underline{Hyperbel}

           \item Parablen Typ Gleichung $c_1x_2^2 - x_2 = 0$

          \end{itemize}


        \end{enumerate}

        \chapter{Mehrdimensionale Analysis}

        Funktionen \[f: D \subseteq \mathbb{R}^n \rightarrow \mathbb{R}^m \] Vektoren im $\mathbb{R}^n$ sind \underline{Zeilenvektoren}. 

        \par \medskip

        \underline{Länge} von Vektoren: $x = (x_1, ..., x_n)$, $||x|| = + \sqrt{x_1^2 + ... + x_n^2}$

        \par \medskip

        \underline{Abstand} von $x$ und $y$: $||x-y|| = \sqrt{\sum_{i=1}^n (x_i - y_i)^2}$

        \section{Beispiele}

        \begin{enumerate}[label=(\alph*)]
          \item $f : D \subseteq \mathbb{R}^n \rightarrow \mathbb{R}$ \underline{skalare Funktionen}

          Z.B: $f(x_1, x_2, x_3) = x_1^2 + x_2^2 + x_3^2$, $D = \mathbb{R}^3$

          \par \medskip

          $f(x_1, x_2) = \frac{2_x1x_2}{x_1^2 + x_2^2}$, $D = \mathbb{R}^2 \setminus \{(0,0)\}$

          $x,y,z$ statt $x_1, x_2, x_3$.

          \par \medskip

          $n = 2$: $f: D \subseteq \mathbb{R}^2 \rightarrow \mathbb{R}$ lassen sich graphisch darstellen.

          \par \medskip

          Beispiel: 

          \begin{itemize}
            \item $f(x,y) = x+y$

            \includegraphics[width=0.3\textwidth]{Bilder/10_1_1.png}
            
            \item $f(x,y) = x \cdot y$


            \includegraphics[width=0.3\textwidth]{Bilder/10_1_2.png}
             
      
            \item $f(x,y) = x^2-y^2$

            \includegraphics[width=0.3\textwidth]{Bilder/10_1_3.png}

            \item $f(x,y) = sin(x) \cdot sin(y)$

              \includegraphics[width=0.5\textwidth]{Bilder/10_1_4.png}
          \end{itemize}

          \item $f: D \subseteq \mathbb{R}^n \rightarrow \mathbb{R}^m$, $m>1$ Vektorwertige Funktion.

          $x \in D : f(x) = (f_1(x), ..., f_m(x))$ \[ f_i : D \subseteq \mathbb{R}^n \rightarrow \mathbb{R} \] skalar

          \par \medskip

          Spezialfall in 10.1.b): $n = 1, m = 2/3$.
        \end{enumerate}

        \section{Beispiel (ebene Kurven / Raumkurven)}

        z.B $f : \begin{cases} [0, 2 \pi ] \Rightarrow \mathbb{R}^2 \\ t \mapsto (cos(t), sin(t)) \end{cases}$

        \par \medskip

        $n = 2$: Ebene Kurve

        $n = 3$:  Raumkurve

        \subsection*{n = 2}

        Physikalische Interpret.: 

        t Zeit $f(t)$ Ort

        \[ f'(t) = (f'_1(t), f_2'(t)) \]

        Geschwindigkeitsvektor.

        Betrag der Geschwindigkeit: \[ ||f'(t)|| = \sqrt{f_1'(t)^2 + f_2'(t)^2} \]

        Lönge der Kurve zwischen $t = a, t = b$?

        \par \medskip

        Konst. Geschwindigkeit v : $L = v \cdot (b-a)$.

        \par \medskip

        Bei variabler Geschwindigkeit: \[ v(t) = || f'(t)|| \] Approx. \[ L \approx \sum_{i=0}^k v(t_i) (t_{i+1} - t_i) \]

        Grenzwert: \[ L = \int_a^b ||f'(t)|| dt \]

        \section{Satz}

        Ist $f : D \subseteq \mathbb{R} \rightarrow \mathbb{R}^n$, $f(t) = (f_1(t), ..., f_n(t))$, $f_i$ stetig diffbar, $[a,b] \subseteq D$, so gilt für die \underline{Bogenlänge} L der Kurve zwischen $t = a$ und $t = b$: \[ L = \int_a^b ||f'(t)|| dt \]

        (1-dim. Analysis)

        \par \medskip

        (Exakter Beweis: Forster, Analysis 2)

        \section{Beispiel}

        Abrollen eines Kreises auf der x-Achse. Bahn eines Punktes auf der Peripherie des Kreises.

        \par \medskip

        $f(t) = (t-sin(t), 1-cos(t))$

        \par \medskip

        \underline{Zykloide}

        \par \medskip

        \[ L = \int_0^{2 \pi} ||f'(t)|| dt \] (Übungsaufgabe)

        \par \medskip

        Grenzwertbegriff für Funkt. $D \subseteq \mathbb{R}^n \rightarrow \mathbb{R}^m$

        \section{Definition}

        $f: D \subseteq \mathbb{R}^n \rightarrow \mathbb{R}^m$, $a$ Adhärenzpunkte (d.h. es ex. $d_i \in D$ mit $||d_i-a|| \underset{i \rightarrow \infty}{\longrightarrow} 0$)

        \begin{enumerate}[label=(\alph*)]
          \item $\lim \limits_{x \rightarrow a} f(x) = c \in \mathbb{R}^m$ $\Leftrightarrow \forall \epsilon > 0 \ \exists \ \delta > 0 \ \forall \ x \in D: ||x-a|| < \delta \Rightarrow ||f(x)-c|| < \epsilon$

          \item Ist $a \in D$, so heißt $f$ stetig in $a$, falls \[ \lim_{x \rightarrow a} f(x) = f(a) \]
        \end{enumerate}


         \section{Bemerkung}

          $f: D \subseteq \mathbb{R}^n \rightarrow \mathbb{R}^m$, $f(x) = (f_1(x), ..., f_m(x)), x \in D$.

          Dann gilt:

          \[ \lim_{x \rightarrow a} f(x) = (c_1, ..., c_n) \Leftrightarrow \lim_{x \rightarrow a} f_i(x) = c_i, i=1, ...,m \]

          Insb. $f$ stetig in $a \Leftrightarrow$ alle $f_i$ stetig in $a$.

          \par \medskip

          $f_i : D \rightarrow \mathbb{R}$ skalare Funktion.

          \section{Beispiel}

          Monom: $f(x_1, ..., x_n) = ax_1^{m_1} \cdot ... \cdot x_n^{m_n}$, $a \in \mathbb{R}, m_i \in \mathbb{N}_0$ fest.

          \par \medskip

          Polynom: Summe von endl. vielen Monomen.

          \par \medskip

          $f(x_1, x_2, x_3) = 3x_1^7 + 4x_1x_2x_3^5 - 12x_1^2x_2^4$

          Polynome sind stetig.

          \par \medskip

          Allgemein gilt (Beweis wie in Analysis einer Var):

          Summe, Produkt, Quotient von stetigen (Skalaren) Fkt. sind stetig.

          \par \medskip

          Hintereinanderausführung stetiger Funktionen ist stetig. 

          \section{Definition}

          \begin{enumerate}[label=(\alph*))]

            \item   $a \in \mathbb{R}^n, r \in \mathbb{R}, r > 0$.

            \[ K(a,r) = \{ x \in \mathbb{R}^n : ||x-a|| < r \} \] \underline{offene Kugeln} um $a$ von Radius $r$.

            \par \medskip

            $n = 1$: offenes Intervall $]a-r, a+r[$

            \par \medskip

            $n = 2$: $K(a,r)$ ist das Innere des Kreises ohne die Kreislinie

            \item $D \subseteq \mathbb{R}^n$ heißt \underline{offen}, falls es zu jedem $a \in D$ ein $r(a) > 0$ gibt mit $K(a, r(a)) \subseteq D$.

            \par \medskip

            $\{x \in \mathbb{R}^2 : || x-0|| \le 1 \}$ ist \underline{nicht} offen.

            \par \medskip

            $\{x \in \mathbb{R}^2 : || x-0 || < 1 \}$ ist offen.

          \end{enumerate}


          \section{Definition}

          $f : D \subseteq \mathbb{R}^n \rightarrow \mathbb{R}$, $D$ offen.

          \par \medskip

          $a = (a_1, ..., a_n) \in D$.

          \begin{enumerate}[label=(\alph*)]
            \item $f$ heißt \underline{partiell differenzierbar} nach $x_j$ an der Stelle $a$, falls \[ \lim_{x_j \rightarrow a_j} \frac{f(a_1, ..., a_{j-1}, x_j, a_{j+1}, ..., a_n) - f(a_1, ..., a_n)}{x_j-a_j} \] existiert.

            \par \medskip

            \underline{Partielle Ableitung} von $f$ nach $x_j$ an der Stelle $a$: \[ \frac{\partial f(a)}{\partial x_j} \]

            (n=1: $f'(a)$)

            \item Ist $f$ nach allen $x_j$ an der Stelle $a$ partiell differenzierbar, so heißt \[ (\frac{\partial f(a)}{\partial x_1}, ..., \frac{\partial f(a)}{\partial x_n}) \]

            \underline{Gradient} von $f$ an der Stelle a.

            \item Ist $f$ an allen $a \in D$ nach allen $x_j$ partiell diffbar, so sind \[ \frac{\partial f}{\partial x_j} : D \rightarrow \mathbb{R} \]

            Funktionen, die \underline{partielle Ableitungen} von $f$ auf $D$  zusammengefasst zu \[ Grad(f) := (\frac{\partial f}{\partial x_1}, ..., \frac{\partial f}{\partial x_n}) \]

            \underline{Gradient} von $f$.

            \par \medskip

            [ $f: D \subseteq \mathbb{R}^n \rightarrow \mathbb{R}^m$, $f(x) = (f_1(x), ..., f_m(x))$ skalar. Fkt. $D \rightarrow \mathbb{R}$ \[ \begin{pmatrix}\frac{\partial f_1}{\partial x_1} & ... & \frac{\partial f_1}{\partial x_n} \\ ... & ... & ... \\ \frac{\partial f_n}{\partial x_1} & ... & \frac{\partial f_n}{\partial x_n} \end{pmatrix} \] \underline{Jacobi-Matrix}]
          \end{enumerate}

          \section{Beispiel}

          $f(x_1, x_2) = sin(_1 \cdot x_2)$, $\mathbb{R}^2 \rightarrow \mathbb{R}$

          \par \medskip

          Partielle Abl. an Stelle $a = (\pi, 1)$

          \begin{align*}
            \frac{ \partial f(\pi, 1)}{\partial x_1} & = \lim_{x_1 \rightarrow \pi} \frac{f(x_1, 1) - f(\pi, 1)}{x_1- \pi} \\
            & = \lim_{x_1 \rightarrow \pi} \frac{sin(x_1) - sin(\pi)}{x_1 - \pi} \\
            & = sin'(\pi) \\
            & = cos(\pi) \\
            & = -1
          \end{align*}

          \begin{align*}
            \frac{\partial f(\pi, 1)}{\partial x_2} & = \lim_{x_2 \rightarrow 1} \frac{f(\pi, x_2) - f(\pi, 1)}{x_2-1} \\
            & = \lim_{x_2 \rightarrow 1} \frac{sin(\pi x_2) - sin(\pi)}{x_2-\pi} \\
            & = \pi \cdot cos(\pi \cdot 1) \\
            & = - \pi
          \end{align*}

          Allgemein: part. Ableitungen an beliebiger Stelle $(x_1, x_2)$:

          \begin{align*}
            \frac{\partial f}{\partial x_1} \underbrace{=}_{\text{halte $x_2$ konstant}} cos(x_1 \cdot x_2) \cdot x_2
          \end{align*}

          \[ \frac{\partial f}{\partial x_2} \underbrace{=}_{\text{halte $x_1$ konstant}} cos(x_1 \cdot x_2) \cdot x_1 \]

          \par \medskip
          
          $Grad(f) = (cos(x_1x_2) \cdot x_2, cos(x_1x_2) \cdot x_1)$


        

        \end{document} 



